\section{Objetivos espec\'ificos}
\label{objetivos}

Si bien GER es la tarea m\'as estudiada de GLN, la evaluaci\'on de los sistemas de GER es a\'un un problema abierto. 
El objetivo final de un sistema de GER es generar una expresi\'on referencial que pueda ser utilizada con \'exito por 
un usuario humano a fin de identificar el referente previsto. Y por lo tanto, los sistemas de GER deben ser evaluados 
por personas, o por m\'etodos de evaluaci\'on automatizados correlacionados con las evaluaciones humanas. 
El problema es que la evaluaci\'on humana es costosa, mientras que las evaluaciones autom\'aticas propuestas 
hasta al momento no han mostrado correlaci\'on con las evaluaciones humanas~\cite{Reiter09}.

Proponemos evaluar el algoritmo de generaci\'on de expresiones referenciales implementado en~\cite{Areces2008} 
utilizando dos metodolog\'ias. 

\begin{description}

\item[Integraci\'on en un sistema de generaci\'on de instrucciones:] El GIVE Challenge~\cite{KolByrCasDalStrMooObe09} es una
 competencia propuesta para evaluar sistemas de GLN situados en entornos virtuales. Su objetivo es disminuir el costo de la evaluaci\'on humana requerida para sistemas de GLN obteniendo testers en Internet. Con este objetivo en mente, 
la tarea que los testers tienen que realizar es seguir las instrucciones del sistema de GLN para as\'i alcanzar 
el objetivo de un juego en un entorno virtual en tres dimensiones. Esta tarea es desafiante para los algoritmos 
de generaci\'on de expresiones referenciales por dos razones. La primera es que el contexto de referencia cambia 
constantemente a medida que el jugador se mueve en el mundo virtual y lo modifica. Se investigar\'a la posibilidad 
de adaptar el algoritmo para considerar un modelo din\'amico. La segunda es que, dado que las instrucciones son 
ef\'imeras (no quedan registradas en un log) el sistema debe decidir como fragmentar la informaci\'on de las expresiones
 referenciales de forma tal de que sean m\'as efectivas. Por ejemplo, un sistema que se refiere a un objeto como 
``tenemos que presionar un bot\'on azul'' y espera que el jugador enfoque dos posibles referentes en el dominio 
para luego decir ``el de la izquierda'' probablemente ser\'a m\'as efectivo que un sistema que diga ``tenemos que 
presionar el bot\'on azul que est\'a atr\'as tuyo a la izquierda'' (dado que al girar, la persona puede haber 
olvidado si era el de la izquierda o el de la derecha). Se investigar\'a la posibilidad de que el algoritmo no retorne
 referencias un\'ivocas para cada elemento del contexto sino que retorne referencias parciales y luego refine 
esas referencias agregando propiedades incrementalmente. 
\end{description}

