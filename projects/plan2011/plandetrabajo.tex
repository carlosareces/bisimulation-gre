\section{Plan de Actividades}
\label{plandetrabajo}

En esta secci\'on detallaremos los objetivos para el primer a\~no de trabajo, cuyo calendario de actividades se describe en la tabla. 

\begin{enumerate}
\item Lectura de la bibliograf\'ia citada.
%\item Explorar las posibilidades de la aplicaci\'on de l\'ogica para resoluci\'on de expresiones referenciales. 
\item Aprendizaje y modificaci\'on del software disponible actualmente.
\item Generaci\'on de modelos correspondientes a corpus disponible.
\item Generaci\'ion de texto a partir de la f\'ormula.
\item Aplicaci\'on a corpus citados en el punto anterior. Evaluaci\'on de los resultados.
\item Proponer modificaciones tanto para el modelo, como para el algoritmo, que mejoren la GRE.
\item Publicaci\'on de resultados.
%\item 
\end{enumerate}

A continuaci\'on se sumarizan las tareas a desarrollar en los pr\'oximos 12 meses.

{\footnotesize
\begin{center}
\begin{tabular}{|p{7cm}||p{2mm}|p{2mm}|p{2mm}|p{2mm}||p{2mm}|p{2mm}|p{2mm}|p{2mm
}||p{2mm}|p{2mm}|p{2mm}|p{2mm}||}
\hline
 \rowcolor[rgb]{0.8,0.8,0.8}\hspace{3.5cm}Tarea & 1 & 2 & 3 & 4 & 5 & 6 & 7 & 8
& 9 & 10 & 11 & 12\\
\hline 1. Lectura de la bibliograf\'ia citada. & $\times$ & $\times$ & $\times$ & &&&&&&&&\\
\hline 2. Aprendizaje y modificaci\'on del software disponible actualmente. &  & $\times$ & $\times$ &  && &&&&&&\\
\hline 3. Generaci\'on de modelos correspondientes a corpus disponible.  & &&$\times$&$\times$&   & &&&&&&\\
\hline 4. Generaci\'ion de texto a partir de la f\'ormula.& &&&$\times$&   & &&&&&&\\
\hline 5. Aplicaci\'on a corpus citados en el punto anterior y an\'alisis de resultados. & & &  &$\times$&$\times$&&&&&&&\\
\hline 6. Proponer modificaciones tanto para el modelo, como para el algoritmo, que mejoren la GRE. && &  &&&$\times$&$\times$&&&&&\\
\hline 7. Aplicaci\'on y an\'alisis de resultados.&&&& & &&$\times$ &$\times$&&&&\\
\hline 8. Documentaci�n y publicaci\'on de resultados. &&&& & &&& &$\times$ &&&\\
\hline 9. Aplicaci\'on al GIVE challenge. &&&& & &&&&&$\times$&$\times$ &$\times$ \\
%\hline 8. Resultados, mejoras ??? &&&& & &&&&&$\times$&$\times$&$\times$\\
\hline
%\hline 6.  & & &  & &&$\times$&$\times$&$\times$&$\times$&&&\\
%\hline 7.  & & & $\times$ & $\times$&&&$\times$&$\times$&&&&\\
%\hline 7. Testing &&&&$\times$&&&&$\times$&&&&$\times$\\
%\hline 8.  &&&&&&&&&$\times$&$\times$&$\times$&\\
%\hline 9.  &&&&$\times$&$\times$&&&$\times$&$\times$&&$\times$&$\times$\\
%\hline 10.  &&&&&&&&&&&$\times$&$\times$\\
%\hline 11.  &&&&$\times$&$\times$&&&$\times$&$\times$&&&$\times$\\\hline
%\hline 12. Elaborac.\ y presentaci\'on de resultados aplicados &&&&$\times$&$\times$&$\times$&&$\times$&$\times$&$\times$&&$\times$\\\hline
\end{tabular}\end{center}
}


