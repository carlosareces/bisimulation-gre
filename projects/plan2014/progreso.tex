\section{Avances logrados a\~no 2013-2014 }
\label{progreso}

\begin{itemize}
\item \textbf{Tareas de investigaci\'on:}
\begin{itemize}
%\item Aplicamos el algoritmo realizado en el a\~no anterior a otro corpus de expresiones referenciales humanas~\cite{gatt-balz-kow:2008:ENLG}, de distinto tipo del corpus usado el a\~no anterior consiguiendo buenos resultados que fueron comparados con los mejores resultados de una competencia de generaci\'on de expresiones referenciales~\cite{KrahmerGRAPH}. Tambi\'en realizamos una evaluaci\'on humana con jueces y mostramos que en el 92\% de los casos la expresi\'on referencial generada por nuestro algoritmo es considerada mejor o igual a la hecha por humanos.

\item Conjuntamente con el Dr. Invandre Paraboni, University of S\~ao Paulo - USP 
School of Arts, Sciences and Humanities - EACH, hemos recolectado un corpus de descripciones de mapas. El corpus recolectado es de expresiones referenciales dichas por personas a las cuales se les mostraron 20 mapas con uno o dos lugares se\~nalados en cada mapa. La recolecci\'on se realiz\'o en dos idiomas en espa\~nol y en portugu\'es, siendo diferentes los mapas usados teniendo en cuenta el idioma, es decir se eligieron lugares en el mismo idioma de la persona. Se recolectaron aproximadamente 4000 expresiones referenciales. 

El proceso de anotar un corpus es un proceso por el cual la informaci\'on desestructurada en languaje natural (cada persona puede nombrar cosas de diferente manera), se unifica a fin de agrupar las diferentes formas de decir las cosas, el objetivo es obtener un archivo XML con la informaci\'on formalizada.
Para esto hemos creado un sistema de anotaci\'on, en el cual ponemos los mapas, los identificadores de las referencias en dichos mapas (calles, restaurantes, etc.). Además incluye digitalizar las frases de cada persona al sistema, estandarizar nombres (ejemplo: bar, restaurant), identificar relaciones a usar en cada mapa (ejemplo: en la esquina-de, a la derecha-de). Finalmente, para cada expresi\'on referencial, evaluamos si es una expresi\'on v\'alida, en caso afirmativo se anota lo que la persona dijo, en caso negativo desecharla como inv\'alida. Esto se deber\'a realizar para cada una de las 100 expresiones referenciales recolectadas por cada uno de los 20 mapas en cada idioma.
 
\item Present\'e un art\'iculo cient\'ifico~\cite{benotti-altamirano-context2013} en la conferencia \emph{International and Interdisciplinary Conference on Modelling and Using Context (Context 2013)}. Fue publicado en modo \emph{lecture notes in Artificial Intelligence}, editorial Springer, dicha conferencia se realiz\'o desde el 28 de octubre de 2013 al 1 de noviembre de 2013, Annecy, Francia. En la presentaci\'on que realic\'e en CONTEXT 2013 conoci varios investigadores del \'area los cuales se mostraron interesados en mi trabajo Dra. Roberts Craige de Ohio	State	University, Dr. Stone Matthew, Rutgers, USA.
\end{itemize}
\item \textbf{Publicaciones:} 
\begin{itemize}
\item Present\'e un art\'iculo cient\'ifico~\cite{benotti-altamirano-context2013} en la conferencia \emph{International and Interdisciplinary Conference on Modelling and Using Context (Context 2013)}. Fue publicado en modo \emph{Lecture Notes in Artificial Intelligence}, editorial Springer, dicha conferencia se realiz\'o desde el 28 de octubre de 2013 al 1 de noviembre de 2013, Annecy, Francia.
\item Present\'e un poster de un art\'iculo cient\'ifico ``Frequencies of occurrence in LGLex lexicon with IRASubcat''~\cite{tolone-altamirano} en JAIIO 2013 en el \emph{Simposio Argentino de Inteligencia Artificial (ASAI 2013)}, FAMAF, C\'ordoba Argentina que se realiz\'o desde el 16 al 20 de septiembre de 2013.
%\item Conjuntamente con mi directora Dra. Luciana Benotti hemos escrito un art\'iculo cient\'ifico~\cite{benotti-altamirano-context2013} con los resultados de esta investigaci\'on. Este art\'iculo fue aceptado en la conferencia \emph{International and Interdisciplinary Conference on Modelling and Using Context (Context 2013)}. Ser\'a publicado en modo \emph{lecture notes in Artificial Intelligence}, editorial Springer. 
%\item Conjuntamente con Elsa Tolone hemos escrito un art\'iculo cient\'ifico ``Frequencies of occurrence in LGLex lexicon with IRASubcat''~\cite{tolone-altamirano} que fue aceptado en JAIIO 2013 en el \emph{Simposio Argentino de Inteligencia Artificial (ASAI 2013)} que se llevar\'a a cabo en FAMAF, C\'ordoba, Argentina desde el 16 al 20 de Septiembre de 2013.
\item Con mi directora Dra. Luciana Benotti y el Dr. Carlos Areces continuamos con la escritura de un art\'iculo~\cite{benotti-altamirano-jair} de journal que ser\'a submitido a \emph{Journal of Artificial Intelligence Research (JAIR)}.
%\item Continuamos con la anotaci\'on del corpus recolectado con el Dr. Ivandre Paraboni.
\end{itemize}

\item \textbf{Seminarios:}
\begin{itemize}
\item Presente mi trabajo en el seminario de alumnos de Famaf 2013.
\end{itemize}
\item \textbf{Idioma:}
\begin{itemize}
\item Aprob\'e el curso de \textbf{ingl\'es} requerido en mi doctorado.

%\item \textbf{``Graph-based Representation and Reasoning in Artificial Intelligence''} dictado por Madalina Croitoru (University of Montpellier II, Francia), curso de 15 horas de duraci\'on en la \emph{Escuela de Ciencias Inform\'aticas 2013 (ECI 2013)}.

%\item \textbf{``Runtime Verification: From Theory to Practice and Back''} dictado por Gordon Pace (Department  of  Computer  Science,  Faculty  of  ICT,  University  of  Malta), curso de 15 horas de duraci\'on en la \emph{Escuela de Ciencias Inform\'aticas 2013 (ECI 2013)}.
\end{itemize}

\item  \textbf{Cr\'editos:} 
A la fecha he cumplimentado con todos los cr\'editos, las presentaciones de seminarios y la aprobaci\'on del idioma ingl\'es requeridos en mi doctorado.

\item  \textbf{Licencia:}
Estuve de licencia por maternidad desde el 15 de febrero al 15 de agosto de 2014. 

\end{itemize}




