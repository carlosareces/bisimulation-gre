\section{Avances logrados a\~no 2012-2013 }
\label{progreso}

\begin{itemize}
\item \textbf{Tareas de investigaci\'on:}
\begin{itemize}
\item Aplicamos el algoritmo realizado en el a\~no anterior a otro corpus de expresiones referenciales humanas~\cite{gatt-balz-kow:2008:ENLG}, de distinto tipo del corpus usado el a\~no anterior consiguiendo buenos resultados que fueron comparados con los mejores resultados de una competencia de generaci\'on de expresiones referenciales~\cite{KrahmerGRAPH}. Tambi\'en realizamos una evaluaci\'on humana con jueces y mostramos que en el 92\% de los casos la expresi\'on referencial generada por nuestro algoritmo es considerada mejor o igual a la hecha por humanos.

\item Conjuntamente con el Dr. Invandre Paraboni, University of S\~ao Paulo - USP 
School of Arts, Sciences and Humanities - EACH, estamos trabajando desde noviembre de 2012 en la investigaci\'on necesaria para la recolecci\'on de un corpus, actualmente recolectado, y en proceso de anotaci\'on. Dicha investigaci\'on incluy\'o una visita a la Universidad de S\~ao Paulo desde el 17 de febrero al 1 de marzo de 2013.
\end{itemize}
\item \textbf{Publicaciones:} 
\begin{itemize}
\item Conjuntamente con mi directora Dra. Luciana Benotti hemos escrito un art\'iculo cient\'ifico~\cite{benotti-altamirano-context2013} con los resultados de esta investigaci\'on. Este art\'iculo fue aceptado en la conferencia \emph{International and Interdisciplinary Conference on Modelling and Using Context (Context 2013)}. Ser\'a publicado en modo \emph{lecture notes in Artificial Intelligence}, editorial Springer. 
\item Conjuntamente con Elsa Tolone hemos escrito un art\'iculo cient\'ifico ``Frequencies of occurrence in LGLex lexicon with IRASubcat''~\cite{tolone-altamirano} que fue aceptado en JAIIO 2013 en el \emph{Simposio Argentino de Inteligencia Artificial (ASAI 2013)} que se llevar\'a a cabo en FAMAF, C\'ordoba, Argentina desde el 16 al 20 de Septiembre de 2013.
\item Con mi directora Dra. Luciana Benotti y el Dr. Carlos Areces estamos escribiendo un art\'iculo~\cite{benotti-altamirano-jair} de journal que ser\'a submitido a \emph{Journal of Artificual Intelligence Research (JAIR)}.
\end{itemize}

\item \textbf{Conferencias:}
\begin{itemize}
\item Presente el art\'iculo~\cite{arec:2012:coling12} en la conferencia \emph{24th International Conference on Computational Linguistics IIT Bombay, Mumbai, India (Coling 2012)} del 8 al 15 de diciembre de 2012.
\end{itemize}
\item \textbf{Cursos:}
\begin{itemize}
\item \textbf{``Positional games''} dictado por Milos Stojakovic (Department of Mathematics and Informatics, University of Novi Sad, Serbia), curso de 15 horas de duraci\'on en la \emph{Escuela de Ciencias Inform\'aticas 2013 (ECI 2013)}.

\item \textbf{``Graph-based Representation and Reasoning in Artificial Intelligence''} dictado por Madalina Croitoru (University of Montpellier II, Francia), curso de 15 horas de duraci\'on en la \emph{Escuela de Ciencias Inform\'aticas 2013 (ECI 2013)}.

\item \textbf{``Runtime Verification: From Theory to Practice and Back''} dictado por Gordon Pace (Department  of  Computer  Science,  Faculty  of  ICT,  University  of  Malta), curso de 15 horas de duraci\'on en la \emph{Escuela de Ciencias Inform\'aticas 2013 (ECI 2013)}.
\end{itemize}

\item  \textbf{Cr\'editos:} 
A la fecha tengo 9 cr\'editos y la aprobaci\'on de los cursos mencionados m\'as los cursos que realic\'e el a\~no anterior en la misma escuela me dar\'ian los 3 cr\'editos que me faltan para completar los 12 cr\'editos requeridos en mi doctorado.


\end{itemize}




