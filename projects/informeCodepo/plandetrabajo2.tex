\section{Plan de Actividades 2012-2013}
\label{plandetrabajo2}

%En los \'ultimos a\~nos se han propuesto diferentes algoritmos para la generaci\'on de expresiones referenciales sobreespecificadas (Por ejemplo,~\cite{ruudINLG2012}. Tanto el algoritmo~\cite{delucenaENLG} como el de \cite{ruudINLG2012} han sido evaluados con el corpus TUNA-AR~\cite{gattENLG} y obtienen 33\% y 40\% de exactitud respectivamente. Nuestro algoritmo obtiene un promedio de 75\% en el corpus~\cite{viethen-dale:2011:UCNLG+Eval}. Una tarea interesante ser\'{i}a evaluar nuestro algoritmo con el corpus TUNA-AR para poder comparar nuestros resultados con los de ellos.

%Tambi\'en nos gustar\'{i}a evaluar nuestro algoritmo con dominios m\'as complejos como los provistos por dominios abiertos Folksonomies~\cite{pachecoDuboue} y con corpus de dominios de interacci\'on humana como el GIVE Corpus~\cite{GarGarKolStr10} donde es com\'un observar que las expresiones referenciales se dan por partes. Bajo presi\'on de tiempo, las personas producen una expresi\'on que no alcanza para identificar al objeto y que incluye las propiedades sobresalientes del objeto (por ejemplo ``el bot\'on rojo''). Y luego en la siguiente oraci\'on, ellos dan propiedades adicionales (por ejemplo ``a la izquierda de la l\'ampara'') para hacer la expresi\'on referencial una expresi\'on que identifique univocamente al objeto. Para realizar esto, deber\'iamos hacer cambios en el algoritmo, para poder dar una expresiones parciales tambi\'en ser\'{i}a interesante no tener que ejecutar el algoritmo completo de nuevo por un cambio en el modelo, estos cambios har\'{i}an que nuestro algoritmo se pueda usar en dominios din\'amicos.

A continuaci\'on se listan las tareas a desarrollar en los doce meses comprendidos 
entre el 1 de Septiembre 2012 y el 31 de Agosto 2013.

{\footnotesize
\begin{center}
\begin{tabular}{|p{7cm}||p{2mm}|p{2mm}|p{2mm}|p{2mm}||p{2mm}|p{2mm}|p{2mm}|p{2mm
}||p{2mm}|p{2mm}|p{2mm}|p{2mm}|}
\hline
 \rowcolor[rgb]{0.8,0.8,0.8}\hspace{3.5cm}Tarea & 1 & 2 & 3 & 4 & 5 & 6 & 7 & 8
& 9 & 10 & 11 & 12\\
\hline 1. Lectura de bibliograf\'ia  del corpus TUNA-AR & $\times$ & & & &&&&&&&&\\
\hline 2. Adaptaciones necesarias, aprendizaje autom\'atico de probabilidades & $\times$ & & & &&&&&&&&\\
\hline 3. Ejecuci\'on del algoritmo y comparaci\'on de resultados & $\times$ & $\times$ & & &&&&&&&&\\
\hline 4. Obtenci\'on de los datos de corpus necesarios para armar un modelo el cual se intentar\'a realizar autom\'aticamente con datos de la web (Folksonomies). &  & $\times$ & $\times$ &  && &&&&&&\\
\hline 5. Obtenci\'on de las probabilidades de uso usando aprendizaje autom\'atico a partir de datos de la web.& &&$\times$&$\times$&   & &&&&&&\\
\hline 6. Ejecuci\'on del algoritmo, an\'alisis de resultados y comparaci\'on con resultados similares. & &&&$\times$&  $\times$ & &&&&&&\\
\hline 7. Adaptaci\'on a dominio din\'amico. & & & & & &$\times$&$\times$&$\times$&$\times$&&&\\

\hline 8. Integraci\'on del algoritmo en una aplicaci\'on con dominio din\'amico (por ejemplo GIVE). &&&& & &&& & &$\times$&$\times$&\\
\hline 9. Documentaci�n y publicaci\'on de resultados. &&&& & &&& & &&&$\times$\\
\hline

\end{tabular}\end{center}
%}


