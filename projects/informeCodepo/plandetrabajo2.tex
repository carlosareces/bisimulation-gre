\section{Plan de Actividades 2012-2013}
\label{plandetrabajo2}

Una de las tareas que nos proponemos para el siguiente per\'{\i}odo es evaluar nuestro algoritmo con otro corpus el TUNA-AS que es un corpus que se us\'o para una competencia en el a\~no 2008, en la cual los participantes obtuvieron una precisi\'on del 33\% y 40\% de exactitud, para poder comparar nuestros resultados con los de ellos. Con los resultados de esta tarea y sumando los resultados de lo realizado el a\~no anterior tenemos pensado escribir un paper de revista.

Tambi\'en nos gustar\'{i}a evaluar nuestro algoritmo con dominios m\'as complejos como los provistos por dominios abiertos Folksonomies y con corpus de dominios de interacci\'on humana como el GIVE Corpus donde es com\'un observar que las expresiones referenciales se dan por partes. Bajo presi\'on de tiempo, las personas producen una expresi\'on que no alcanza para identificar al objeto y que incluye las propiedades sobresalientes del objeto (por ejemplo ``el bot\'on rojo''). Y luego en la siguiente oraci\'on, ellos dan propiedades adicionales (por ejemplo ``a la izquierda de la l\'ampara'') para hacer la expresi\'on referencial una expresi\'on que identifique univocamente al objeto. Para realizar esto, deber\'iamos hacer cambios en el algoritmo, para poder dar una expresiones parciales tambi\'en ser\'{i}a interesante no tener que ejecutar el algoritmo completo de nuevo por un cambio peque\~no en el modelo, estos cambios har\'{i}an que nuestro algoritmo se pueda usar en dominios din\'amicos.



