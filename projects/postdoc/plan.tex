\documentclass[10.9pt,a4paper]{article}

\usepackage[spanish]{babel}
\usepackage[draft]{fixme}
\usepackage{url,latexsym,amssymb,color}
\usepackage{a4wide}
\usepackage{xcolor}
\usepackage{natbib}
\usepackage{xspace}
\usepackage{ifpdf}
\usepackage{colortbl}
\usepackage{enumerate}
\usepackage{setspace}
\usepackage[margin=1.25in]{geometry}

\title{Plan de Trabajo y Actividades}

\date{}

\renewcommand{\refname}{\normalfont\large\textbf{Referencias Bibliogr\'aficas}}

\begin{document}

\maketitle

\begin{description}
	\item [T\'itulo del Trabajo:] Generaci\'on de lenguaje natural para sistemas de di\'alogo
	\item [Alumna:] Ivana Romina Altamirano
	\item [Director:] Dr. Carlos Eduardo Areces
\end{description}



\section{Objetivos}
%Referidos al proposito de la investigacion. El objetivo especifico refleja el 
%resultado esperable en el plazo previsto para la realización del plan. El 
%objetivo general, al cual contribuye el objetivo especifico, se orienta
%hacia resultados posibles de obtener en el largo plazo en la linea de 
%investigacion mediante trabajos ulteriores no incluidos en el plan.

Como objetivo general de este proyecto proponemos investigar 
sistemas de di\'alogo humano-computadora con soporte para interacci\'on 
en entornos de tres dimensiones (3D). Concretamente, estos sistemas 
generan autom\'aticamente instrucciones en lenguaje natural (castellano, 
ingl\'es, alem\'an, etc.) para guiar a un usuario durante la 
ejecuci\'on de una tarea determinada. Un ejemplo
cl\'asico son las instrucciones generadas por un GPS. En forma mas general,
esta clase de sistemas se utilizan como sistemas de asistencia en la 
web (e.g., recomendaciones de compra en un sitio de comercio electr\'onico,
instrucciones t\'ecnicas para usuarios no expertos, etc.), sistemas de control 
de dispositivos por voz (e.g., veh\'iculos, smartphones, smart-tv, etc.), interfaces multimodales (e.g., videojuegos interactivos con reconocimiento 
de movimientos y gestos), etc.

Como objetivo espec\'ifico proponemos dise\~nar modelos 
computacionales y algoritmos para mejorar la generaci\'on de 
instrucciones del sistema y la interpretaci\'on de las respuestas del 
usuario. 
Un sistema de di\'alogo del tipo que nos interesa 
debe ser capaz de entablar una comunicaci\'on coherente, y debe 
adaptarse a los posibles errores de interpretaci\'on de las instrucciones 
produciendo nuevas instrucciones cuando sea necesario. Para ello deber\'a 
ser capaz de analizar el contexto de interacci\'on del di\'alogo con el 
objetivo de intentar inferir las intenciones del usuario.
El usuario se encontrar\'a en un entorno virtual en tres dimensiones
en el que podr\'a interactuar con objetos y explorar el ambiente provisto. 
En este entorno, el usuario tendr\'a que ejecutar las instrucciones 
provistas por el sistema por medio de acciones f\'isicas y tendr\'a la 
posibilidad de solicitar ayuda mediante mensajes en lenguaje natural.

En las etapas iniciales de la investigaci\'on, nos concentraremos en un 
modelo que permita la comunicaci\'on de informaci\'on ling\"u\'istica de 
manera unidireccional, esto es, mensajes en lenguaje natural desde el 
sistema hacia el usuario. La restricci\'on al caso unidireccional tiene por 
objetivo simplificar la representaci\'on y el manejo del contexto de
interacci\'on. En posteriores etapas se extender\'a el modelo unidireccional 
para permitir intercambio ling\"u\'istico bidireccional.




\section{Antecedentes}
%Indicar el marco teorico de la investigacion y las hipotesis de trabajo 
%propuestas consignando, sobre que otros trabajos de investigacion 
%propios o de contribuciones de terceros, se basan.

% Marco teorico del problema.
El objetivo principal de un sistema de di\'alogo generador de instrucciones 
situadas es desempe\~nar el rol de gu\'ia o asistente virtual para ayudar a 
un usuario a llevar a cabo una tarea en un entorno f\'isico (ya sea virtual o 
real). Estos sistemas tienen la capacidad de comunicarse con el 
usuario mediante lenguaje natural y poseen sensores y mecanismos para 
obtener informaci\'on sobre el entorno f\'isico en el cual est\'an situados. 
Un aspecto crucial de la habilidad comunicacional de estos sistemas es la 
capacidad de detectar la evoluci\'on del usuario en la tarea y de poder 
recuperarse de malos entendidos o errores de interpretaci\'on.
Por esta raz\'on, los sistemas generadores de instrucciones deben ser 
sensibles al contexto, llevando registro del estado del entorno de 
interacci\'on para poder interpretar correctamente las acciones y mensajes 
que los usuarios le env\'ian al sistema.

% Pragmatica: Es por el lado en donde el trabajo de Lu y Carlos se 
% relaciona
Esta propuesta de investigaci\'on se enmarca entonces en el \'area 
de la pragm\'atica del lenguaje natural desde un punto de vista computacional. 
Esta es un \'area interdisciplinaria a la que contribuyen por un lado distintas
\'areas de Ciencias de la Computaci\'on (e.g., dise\~no de algoritmos, 
complejidad, representaci\'on del conocimiento), y por otro teor\'ias 
ling\"u\'isticas (e.g., implicaturas conversacionales~\citep{grice75}, 
an\'alisis gramatical del contexto conversacional~\citep{ginzburg09}), 
sociol\'ogicas (e.g., an\'alisis conversacional~\citep{schegloff87b}, 
\emph{joint actions} y \emph{grounding}~\citep{clark96}) y 
filos\'oficas (e.g., teor\'ia de los actos del habla~\citep{austin62}). Su 
objetivo es estudiar c\'omo el contexto, en el que la conversaci\'on est\'a 
situada, contribuye al significado de cada enunciado de la conversaci\'on.
La transmisi\'on de significado depende no s\'olo de la informaci\'on 
ling\"u\'istica (e.g., reglas gram\'aticales y morfol\'ogicas), sino tambi\'en 
de informaci\'on extraling\"u\'istica (e.g., situaci\'on f\'isica donde la 
comunicaci\'on est\'a situada, experiencias previas de los hablantes u 
objetivo comunicacional).
Una misma oraci\'on puede significar cosas diferentes en
distintos contextos. El \'area de la pragm\'atica estudia entonces el
proceso por el cual una oraci\'on es desambiguada usando su contexto. 
%En pragm\'atica se distingue entre oraci\'on (forma gramatical que toma 
%el acto ling\"u\'istico) y enunciado (oraci\'on m\'as su contexto). La 
%habilidad de entender un enunciado se denomina competencia 
%pragm\'atica. Explicar la competencia pragm\'atica implica explicar 
%c\'omo una persona hace inferencias sobre una oraci\'on y su contexto
%para interpretar adecuadamente el enunciado que el emisor intenta 
%transmitir.

Para que un sistema de di\'alogo establezca una comunicaci\'on natural 
con los usuarios, debe demostrar habilidad pragm\'atica. Para ello el 
sistema debe definir el tipo de informaci\'on contextual que se debe 
representar y las tareas de inferencia sobre la oraci\'on y contexto 
que son necesarias para interpretar un enunciado.

% Inferencia: Necesario para linkear con el trabajo de Carlos.
Podemos entender como inferencia toda operaci\'on que transforme 
informaci\'on impl\'icita en expl\'icita. Esta definici\'on es lo 
suficientemente general como para cubrir tareas que van desde la 
inferencia l\'ogica (e.g., tareas de inferencia en l\'ogicas para la 
descripci\'on~\citep{DBLP:conf/dlog/2003handbook,ARECES:thesis}), hasta tareas de 
inferencia habituales en inteligencia artificial (e.g., 
\emph{Planning}~\citep{nau04,beno:mus11}). 
Un sistema de di\'alogo realiza continuamente dos tipos de operaciones de 
inferencia:
\begin{enumerate}[(1)]
    \item  Por un lado, necesita inferencia para interpretar la 
    informaci\'on recibida e incorporarla al repositorio de datos del sistema.
    \item Por otro lado, necesita inferencia para decidir que parte de la 
    informacion disponible se debe transmitir al usuario.
\end{enumerate}

Una de las tareas de inferencia principales relacionadas con (2), que es 
particularmente importante en conversaciones situadas, es la generaci\'on 
de expresiones referenciales. Un sistema de di\'alogo situado en un 
entorno f\'isico debe ser capaz de producir expresiones referenciales no 
ambiguas que le permitan al usuario poder identificar los objetos
correctos. Recientemente, se han investigado formas de utilizar 
teor\'ias de l\'ogica computacional para inferir las propiedades de los 
objetos a utilizar en las expresiones 
referenciales~\citep{arec:refe08,arec:usin11,altamirano-areces-benotti:2012:POSTERS}.




\section{Actividades y Metodolog\'ia}
%Enumerar las tareas a desarrollar y las metodologias experimentales y 
%tecnicas a emplear en el plan de trabajo propuesto para la obtencion 
%de resultados y la demostracion de hipotesis.

La primer actividad a realizar ser\'a el relevamiento y estudio del material 
bibliogr\'afico de metodolog\'ias existentes para el desarrollo de sistemas 
de di\'alogo en entornos 3D, enfatizando principalmente el estudio de la
interacci\'on entre el sistema y el usuario, y las tareas de inferencia 
propuestas en (1) y (2).

Una vez realizado el estudio de la bibliograf\'ia b\'asica, el siguiente 
objetivo consistir\'a en dise\~nar e implementar los componentes que 
conformar\'an el sistema encargado de la generaci\'on de instrucciones y 
de la interpretaci\'on de respuestas del usuario.
Este proceso se dividir\'a en dos etapas: primero se construir\'an los 
componentes y algoritmos necesarios para obtener un modelo de 
comunicaci\'on unidireccional entre el sistema y el usuario. Dicho modelo 
s\'olo permitir\'a la comunicaci\'on de mensajes en lenguaje 
natural desde el sistema hacia el usuario. En este caso, el usuario s\'olo 
podr\'a comunicarse con el sistema mediante la ejecuci\'on de 
acciones f\'isicas en el entorno 3D. Para el desarrollo de este modelo en 
particular nos concentraremos en estudiar el tipo de inferencia dado 
en (2), esto es, inferencia para determinar el contenido de las 
contribuciones ling\"u\'isticas seg\'un el contexto de interacci\'on.
Con respecto al tipo de inferencia (2), dise\~naremos algoritmos de
selecci\'on de oraciones usando \emph{planning} a partir de un corpus 
ling\"u\'istico de interacci\'on humano-humano~\citep{beno:acl11,beno:hci12}.
Inicialmente estos algoritmos tendr\'an en cuenta la posici\'on actual del 
usuario y pr\'oxima acci\'on a realizar, como informaci\'on contextual. 
Luego los extenderemos con informaci\'on contextual discursiva y de 
comportamiento previo.

Luego de estudiar el modelo unidireccional, se extender\'a
el mismo al caso bidireccional. 
El modelo de comunicaci\'on bidireccional le permitir\'a al usuario 
comunicarse con el sistema utilizando lenguaje natural.
%, adem\'as de la interacci\'on f\'isica con el entorno.
Para ello, se pondr\'a en pr\'actica el tipo de inferencia que hemos mencionado
en (1). En relaci\'on a esto, hemos investigado y evaluando la 
utilizaci\'on de modelos computacionales de 
\emph{Grounding}~\citep{traum99} en sistemas generadores de 
instrucciones situados~\citep{racc:ENLG11,raccathesis}. Para 
comparar los algoritmos y modelos definidos en los mencionados trabajos 
hemos implementado un sistema generador de instrucciones y participado en 
la competencia internacional denominada \emph{Generating Instructions 
in Virtual Environments} 
(GIVE)\footnote{\url{http://www.give-challenge.org}. La competencia 
posee el respaldo de los grupos de inter\'es SIGSEM, SIGDIAL y SIGGEN de 
la \emph{Association for Computational Linguistics 
(ACL}).}.
Dicho sistema obtuvo resultados positivos en la mayor\'ia de las 
m\'etricas de evaluaci\'on con diferencias estad\'isticamente significativas 
sobre los dem\'as participantes~\citep{stri:ENLG11}.
Utilizando los datos de comportamiento de usuario de la competencia,
proponemos dise\~nar una arquitectura de aprendizaje por
refuerzos~\citep{sutton98} para privilegiar el uso de instrucciones
que, en contextos similares, hayan logrado que el usuario se acerque a
la meta.

% REG
En trabajo previo~\citep{altamirano-areces-benotti:2012:POSTERS} hemos mejorado un algoritmo basado en 
bisimulaci\'on~\citep{arec:refe08} para la generaci\'on de expresiones referenciales, hemos evaluado el algoritmo con varios corpus~\citep{altamirano-areces-benotti:2012:POSTERS, beno:context2013}. Se planea investigar la adaptaci\'on de estos algoritmos basados en para la generaci\'on de expresiones referenciales multi-oraci\'on en entornos din\'amicos, usando si es necesario la partici\'on de las oraciones.

Resumiendo entonces, el trabajo de investigaci\'on abarcar\'a en principio tres grandes l\'ineas 
de investigaci\'on en el \'area de generaci\'on de lenguaje natural para interfaces conversacionales 
situadas.
\begin{enumerate}

    \item Modelado del terreno com\'un (\emph{common ground}) 
    discursivo y de los operadores de \emph{grounding} usados en 
    conversaci\'on situada~\citep{benotti-EtAl:2012:ACL2012short}.

    \item Dise\~no de un modelo de optimizaci\'on basado en aprendizaje 
    por refuerzo y \emph{planning} para la generaci\'on de instrucciones 
    causalmente apropiadas~\citep{beno:acl11}.

    \item Utilizaci\'on de t\'ecnicas de minimizaci\'on de aut\'omatas para la 
    producci\'on de expresiones referenciales, prestando especial atenci\'on 
    al caso multi-oraci\'on.

\end{enumerate}

Durante todo el proyecto, los resultados te\'oricos y aplicados ser\'an 
presentados en conferencias locales e internacionales, y en revistas 
cient\'ificas pertinentes a las \'areas de investigaci\'on afectadas.
Adem\'as, se planea participar en cursos de posgrado y seminarios afines
al \'area de trabajo. 
%Cabe mencionar que el alumno ha asistido recientemente
%a la Escuela de Ciencias Inform\'aticas ECI 2012, aprobando el curso
%\emph{Advanced Speech Processing Techniques} dictado por el Dr. John Hansen
%de la Universidad de Dallas, USA.

A continuaci\'on se muestra un resumen de las actividades a desarrollar y 
el cronograma que se propone para los primeros dos a\~nos del postdoctorado.
Las columnas del cronograma representan trimestres.

{\footnotesize
\begin{center}
\begin{tabular}{|p{7cm}||p{2mm}|p{2mm}|p{2mm}|p{2mm}||p{2mm}|p{2mm}|p{2mm}|p{2mm
}||p{2mm}|p{2mm}|p{2mm}|p{2mm}||}
\hline
 \rowcolor[rgb]{0.8,0.8,0.8}\hspace{3.5cm}Tarea 
& 1 & 2 & 3 & 4 & 5 & 6 & 7 & 8 \\

\hline 1. Relevamiento bibliogr\'afico y estudio del material 
& $\times$ & $\times$ && $\times$ & $\times$ & $\times$ &&\\

%\hline 2. Comparaci\'on de t\'ecnicas de aprendizaje autom\'atico
%&$\times$ & $\times$ & $\times$ & $\times$ &&&&&&&&\\

\hline 3. Modelado del \emph{common ground} discursivo
& $\times$ & $\times$ & $\times$ & $\times$ & $\times$ & $\times$ &&\\

\hline 4. T\'ecnicas de aprendizaje por refuerzo
&& $\times$ & $\times$ & $\times$ & $\times$ &&&\\

\hline 5. Expresiones referenciales multi-sentencia
&&&& $\times$ & $\times$ & $\times$ & $\times$ &\\

\hline 6. Diseminaci\'on de resultados
&&& $\times$ &&& $\times$ && $\times$\\

\hline
\end{tabular}\end{center}
}



\section{Factibilidad}
%Indicar si el lugar de trabajo cuenta con la infraestructura, los servicios y 
%el equipamiento a emplear. Detallar el origen de los recursos financieros 
%requeridos para la realizacion del plan propuesto. Enumerar los equipos 
%mas importantes a ser utilizados en el desarrollo de su plan de trabajo 
%en la institucion propuesta como lugar de trabajo para la beca o en otra.

% El lugar de trabajo cuenta con la infraestructura y servicios necesarios.
El postdoctorado se realizar\'a en la Facultad de Matem\'atica, Astronom\'ia y 
F\'isica (FaMAF) de la Universidad Nacional de C\'ordoba que cuenta con el 
equipamiento e infraestructura necesaria para su desarrollo: espacio 
f\'isico para el postdoctorando, biblioteca y computadoras con acceso a 
internet.
% Origen de los recursos financieros para la realizacion del proyecto.
El trabajo se realizar\'a en el marco del proyecto PICT-PRH 
\emph{``Dialogue Systems for Virtual Environments''} de Famaf
que tiene al Dr. Areces como investigador responsable. Se cuenta 
adem\'as con los recursos del proyecto \emph{``Natural Processing for Virtual Asistants''} que recibi\'o un premio
\emph{SUR AWARD} de IBM. 
Los mencionados proyectos cuentan tambi\'en con recursos para financiar 
vi\'aticos para asistir a conferencias locales e internacionales, y para 
visitar colaboradores en pa\'ises extranjeros.

% Vinculacion del director/co-director con colaboradores y equipos externos
El Dr. Areces ha sido miembros del grupo 
TALARIS, parte del \emph{Laboratoire Lorrain de Recherche en 
Informatique et ses Applications} (LORIA) cuyo principal tema de 
investigaci\'on es la ling\"u\'istica computacional con \'enfasis en 
sem\'antica e inferencia.

% Colaboracion local
% Dar informacion sobre el grupo de PLN de famaf.
%Esta propuesta de investigaci\'on se desarrollar\'a en el marco del 
%\emph{Grupo de Procesamiento de Lenguaje Natural}
%(PLN) de FaMAF. 
Esta propuesta de investigacion se desarrollar\'a en el marco del \emph{
Grupo de L\'ogica, Interacci\'on y Sistemas Inteligentes (LIIS)} de FaMAF.
A nivel nacional dicho grupo colabora con GLyC, \emph{Grupo de L\'ogica, 
Lenguaje y Computabilidad}, parte del Departamento de Computaci\'on de 
la Universidad de Buenos Aires, en el \'area de representaci\'on del 
conocimiento e inferencia. En particular, ambos grupos han colaborado en 
la investigaci\'on de t\'ecnicas de gesti\'on del conocimiento en sistemas de 
di\'alogo~\citep{arec:usin11}. 

%\nocite{kehler04,devault08}
\begin{spacing}{0.77}
\bibliographystyle{named}
\bibliography{plan}
\end{spacing}

\end{document}
