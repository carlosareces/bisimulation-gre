\section{Avances logrados hasta el momento}
\label{progreso}

\begin{itemize}
\item Tareas del doctorado:
Estudi\'e un algoritmo existente de generaci\'on de expresiones referenciales que usa l\'ogica modal para dar la 
expresi\'on referencial~\cite{Areces2008} y lo modifique para dar expresiones referenciales m\'as humanas, para hacer 
esto inclu\'imos las probabilidades de uso de las palabras al algoritmo y estudiamos las posibilidades de agregar 
sobreespecificaci\'on a una expresi\'on referencial, estudi\'e la documentaci\'on de un corpus de expresiones 
referenciales~\cite{viethen-dale:2011:UCNLG+Eval} hecho por humanos e intentamos aprender con aprendizaje autom\'atico 
las probabilidades de uso para las figuras de ese corpus. Se evaluaron 8 figuras de las 32 figuras que aparec\'{\i}an en 
el corpus consiguiendo una buena precisi\'on con nuestro algoritmo. Conjuntamente con mi directora Dra. Luciana Benotti y el Dr. Carlos Areces hemos escrito un paper con los resultados de esta investigaci\'on que est\'a en proceso de revisi\'on.

\item Materias realizadas y pr\'acticos en las materias:
\begin{itemize}
\item Aprob\'e la materia {\bf Generaci\'on de lenguaje natural} En el proyecto de esta materia realic\'e un realizador 
ling\"u\'{\i}stico para el espa\~{n}ol el cual intentar\'e integrar en otras etapas de mi doctorado.
\end{itemize}
\item Conferencias o colaboraciones:
\begin{itemize}
\item Asist\'{\i} al internacional workshop de usar informaci\'on ling\"u\'{\i}stica en sistemas h\'{i}bridos de traducci\'on autom\'atica.
\item Colaboraci\'on con grupo de investigaci\'on (Instituto Nacional Franc\'es de Investigaci\'on en Inform\'atica y Automatizaci\'on)
\item Asist\'{\i} a la Escuela de Ciencias Inform\'aticas 2012 (ECI 2012) en donde realic\'e 2 cursos: ``Java para software de tiempo real'' 
y ``T\'ecnicas avanzadas para procesamiento del habla'' (aprob\'e el curso).
\end{itemize}
\item Otros cursos:
\begin{itemize}
\item Aprob\'e el curso on-line ``Introducci\'on a la inteligencia artificial'' dictado por Dr. Sebastian Thrun y Dr. Peter Norving de la 
universidad de Stanford.
\item Aprob\'e el curso on-line ``Aprendizaje autom\'atico'' dictado por Dr. Andrew Ng de la universidad de Stanford. 
\end{itemize}

\end{itemize}




