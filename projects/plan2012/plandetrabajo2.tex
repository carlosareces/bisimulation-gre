\section{Plan de Actividades 2012-2013}
\label{plandetrabajo}

A continuaci\'on se listan las tareas a desarrollar en los pr\'oximos doce meses comprendidos entre el 1 de Septiembre 2012 y el 31 de Agosto 2013.

En los \'ultimos a\~nos se han propuesto diferentes algoritmos para la generaci\'on de expresiones referenciales sobreespecificadas (Por ejemplo,~\cite{ruudINLG2012}. Tanto el algoritmo~\cite{delucenaENLG} como el de \cite{ruudINLG2012} han sido evaluados con el corpus TUNA-AR~\cite{gattENLG} y obtienen 33\% y 40\% de exactitud respectivamente. Una tarea interesante ser\'{i}a evaluar nuestro algoritmo con este corpus para poder comparar nuestros resultados con los de ellos.

Tambi\'en nos gustar\'{i}a evaluar nuestro algoritmo con dominios m\'as complejos como los provistos por dominios abiertos Folksonomies~\cite{pachecoDuboue} y a corpus de dominios de interacci\'on como el GIVE Corpus~\cite{GarGarKolStr10} donde es comun observar que las expresiones referenciales se dan por partes. Bajo presi\'on de tiempo, las personas producen una expresi\'on poco espec\'{i}fica que incluye las propiedades sobresalientes del objeto (por ejemplo ``el bot\'on rojo''). Y luego en la siguiente oraci\'on, ellos dan propiedades adicionales (por ejemplo ``a la izquierda de la l\'ampara'') para hacer la expresi\'on referencial una expresi\'on que identifica univocamente al target.


{\footnotesize
\begin{center}
\begin{tabular}{|p{7cm}||p{2mm}|p{2mm}|p{2mm}|p{2mm}||p{2mm}|p{2mm}|p{2mm}|p{2mm
}||p{2mm}|p{2mm}|p{2mm}|p{2mm}|}
\hline
 \rowcolor[rgb]{0.8,0.8,0.8}\hspace{3.5cm}Tarea & 1 & 2 & 3 & 4 & 5 & 6 & 7 & 8
& 9 & 10 & 11 & 12\\
\hline 1. Lectura de bibliograf\'ia  del corpus TUNA-AR & $\times$ & & & &&&&&&&&\\
\hline 2. Adaptaciones necesarias, aprendizaje autom\'atico de probabilidades & $\times$ & & & &&&&&&&&\\
\hline 3. Ejecuci\'on de nuestro algoritmo y comparaci\'on de resultados & $\times$ & $\times$ & & &&&&&&&&\\
\hline 4. Obtenci\'on de los datos de corpus necesarios para armar un modelo el cual se intentar\'a realizar autom\'aticamente con datos de la web (Folksonomies). &  & $\times$ & $\times$ &  && &&&&&&\\
\hline 5. Obtenci\'on de las probabilidades de uso usando aprendizaje autom\'atico a partir de datos de la web.& &&$\times$&$\times$&   & &&&&&&\\
\hline 6. Ejecuci\'on del algoritmo, an\'alisis de resultados y comparaci\'on con los resultados de~\cite{pachecoDuboue}. & &&&$\times$&  $\times$ & &&&&&&\\
\hline 7. Integraci\'on a una plataforma din\'amica. & & & & & &$\times$&$\times$&$\times$&$\times$&&&\\

\hline 8. Generaci\'on de expresiones referenciales parciales. &&&& & &&& & &$\times$&$\times$&\\
\hline 9. Documentaci�n y publicaci\'on de resultados. &&&& & &&& & &&&$\times$\\
\hline

\end{tabular}\end{center}
%}


