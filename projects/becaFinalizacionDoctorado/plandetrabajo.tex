\section{Plan de Actividades}
\label{plandetrabajo}

En esta secci\'on detallamos los objetivos para el primer a\~no de trabajo, cuyo calendario de actividades se describe a continuaci\'on. 

El proceso de anotar un corpus es un proceso por el cual la informaci\'on desestructurada en languaje
natural (cada persona puede nombrar cosas de diferente manera), se unifica a fin de agrupar las
diferentes formas de decir las cosas, el objetivo es obtener un archivo XML con la informaci\'on
formalizada. Para esto hemos creado un sistema de anotaci\'on, en el cual ponemos los mapas,
los identificadores de las referencias en dichos mapas (calles, restaurantes, etc.). Adem\'as
incluye digitalizar las frases de cada persona al sistema, estandarizar nombres
(ejemplo: bar, restaurant), identificar relaciones a usar en cada mapa (ejemplo:
en la esquina-de, a la derecha-de). Finalmente, para cada expresi\'on referencial,
evaluamos si es una expresi\'on v\'alida, en caso afirmativo se anota lo que la persona dijo,
en caso negativo desecharla como inv\'alida. Esto se deber\'a realizar
para cada una de las 100 expresiones referenciales recolectadas por cada uno de
los 20 mapas en cada idioma. Esta tarea fue realizada parcialmente el a\~{n}o anterior, quedar\'{i}a
la finalizaci\'on para este a\~{n}o. Estos recursos son muy interesantes ya que provienen de datos
mucho m\'as naturales que otros existentes en el \'area, adem\'as quedar\'an disponibles para que 
otros investigadores puedan usarlos.
Ejecutar nuestro algoritmo con el nuevo corpus y evaluar la salida del algoritmo tanto con m\'etricas 
autom\'aticas como con m\'etricas manuales es una tarea muy interesante que nos afianzar\'ia 
la utilidad de nuestro algoritmo en dominios m\'as naturales. La revisi\'on de la bibliograf\'ia, 
escritura de la tesis, y finalizaci\'on de un art\'iculo de revista en el cual inclu\'iriamos 
la mayor parte de los resultados de mi trabajo.


{\footnotesize
\begin{center}
\begin{tabular}{|p{7cm}||p{2mm}|p{2mm}|p{2mm}|p{2mm}||p{2mm}|p{2mm}|p{2mm}|p{2mm
}||p{2mm}|p{2mm}|p{2mm}|p{2mm}|}
\hline
 \rowcolor[rgb]{0.8,0.8,0.8}\hspace{3.5cm}Tarea & 1 & 2 & 3 & 4 & 5 & 6 & 7 & 8
& 9 & 10 & 11 & 12\\
\hline 1. Re-lectura de la bibliograf\'ia citada. & $\times$ & $\times$ & $\times$ & $\times$ & $\times$ &&&&&&&\\
\hline 2. Finalizar anotaci\'on del corpus ZOOM &  & $\times$ & $\times$ &  && &&&&&&\\
\hline 3. Evaluaci\'on del algoritmo sobre ZOOM &  &          &          & $\times$ & $\times$ & &&&&&&\\
\hline 4. Escritura de la tesis & &$\times$&$\times$& $\times$ & $\times$ & $\times$ & $\times$ & $\times$ & $\times$ &&&\\
\hline 5. Preparaci\'on y defensa de la tesis. & & &  &&&&&&&$\times$&$\times$&$\times$\\
\hline 6. Finalizar la escritura del art\'iculo para JAIR && &  &&&$\times$&$\times$&$\times$&$\times$&&&\\
%\hline 8. Resultados, mejoras ??? &&&& & &&&&&$\times$&$\times$&$\times$\\
\hline
%\hline 6.  & & &  & &&$\times$&$\times$&$\times$&$\times$&&&\\
%\hline 7.  & & & $\times$ & $\times$&&&$\times$&$\times$&&&&\\
%\hline 7. Testing &&&&$\times$&&&&$\times$&&&&$\times$\\
%\hline 8.  &&&&&&&&&$\times$&$\times$&$\times$&\\
%\hline 9.  &&&&$\times$&$\times$&&&$\times$&$\times$&&$\times$&$\times$\\
%\hline 10.  &&&&&&&&&&&$\times$&$\times$\\
%\hline 11.  &&&&$\times$&$\times$&&&$\times$&$\times$&&&$\times$\\\hline
%\hline 12. Elaborac.\ y presentaci\'on de resultados aplicados &&&&$\times$&$\times$&$\times$&&$\times$&$\times$&$\times$&&$\times$\\\hline
\end{tabular}\end{center}
}


