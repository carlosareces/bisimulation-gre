\chapter{Selecci\'on de contenido de expresiones referenciales}
\label{sec:seleccion}

\section{Generaci�n autom�tica de expresiones referenciales}

\section{Algoritmos de REG}

\subsection{Incremental}
\subsection{Graph}
\subsection{Bisimulaci�n}

En este cap�tulo daremos una introducci�n a l�gicas de descripci�n (DL del acr�nimo en ingl�s description logic) y luego daremos el algoritmo en el cual nos basamos en esta tesis.

Hay distintas l�gicas, y entre ellas se diferencian por como se generan, por ejemplo ALC y EL, la diferencia es que ALC incluye negaciones y EL no.

La gram�tica de ALC es:

\todo{Agregar formula aca}

y la de EL es:

\todo{Agregar formula aca}

donde p es un s�mbolo proposicional, y R esta en el conjunto de s�mbolos relacionales.

Las f�rmulas son interpretadas en modelos de primer orden M=(Delta, Intep) donde Delta es un conjunto no vac�o y la Intep es la siguiente funci�n:

Cada f�rmula denota un conjunto de entidades del dominio, as� podemos usar las f�rmulas para denotar conjuntos.

Llamaremos similaridad a la noci�n de preservaci�n de f�rmulas

Dada una DL diremos que una entidad i es similar a i' en el modelo M si para la cualquien f�rmula Fi de L (l�gica considerada) i pertenece a la Inter(Fi) tenesmos que i' tambi�n pertenece. Por lo tanto no hay f�rmula que pueda distinguirlas. La similaridad no es una relaci�n sim�trica

por ejemplo f1 es EL-similar a f2 pero f2 no es EL-similar a f1 

En ALC la similaridad es una realci�n sim�trica porque el lenguaje contiene la negaci�n.

La idea es transformar el problema de GRE al problema de computar una f�rmula de DL cuya extensi�n es el elemento target (o los elementos targets) ya que una f�rmula describe un conjunto.



\section{Aproximaciones emp�ricas a la soluci�n de REG}

\subsection{Corpus existente}
\subsection{Jette y otros trabajos emp�ricos}
\subsection{M�tricas de evaluaci�n/comparaci�n con corpus}



