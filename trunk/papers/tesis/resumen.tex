\thispagestyle{empty}

\begin{itemize}
	\item \textbf{\textsc{Clasificaci�n de Biblioteca:}} \\ \\  \\ \\

	\item \textbf{\emph{\textsc{Palabras Clave:} \\ expresiones referenciales, aprendizaje autom�tico, bisimulaci�n, evaluaci�n.}}
\end{itemize}

\begin{center}

%{ 
\vspace*{1cm} 
%}
\huge{\textbf{\textsc{\textmd{Resumen}}}}\\[1cm]
%\Large{\textsc{\textmd{Facultad de Matem�tica, Astronom\'ia y F\'isica}}}\\[1cm]

\end{center}

\normalsize{
Se propone un algoritmo para la generaci�n de expresiones referenciales (ERs) que adapta la aproximaci�n 
de~\cite{arec2:2008:Areces,arec:usin11} para incluir sobre-especificaci�n y probabilidades aprendidas a partir de corpus. Luego de introducir el algoritmo discutimos como las probabilidades requeridas pueden ser computadas para cualquier dominio para el cual existe un corpus de ERs y como las probabilidades pueden ser ajustadas para nuevas escenas en el dominio usando aprendizaje autom�tico. Ejemplificamos como computar las probabilidades sobre el corpus GRE3D7 de \cite{viet:gene11} y el TUNA corpus de \cite{gatt-balz-kow:2008:ENLG}. El algoritmo resultante es capaz de generar diferentes expresiones referenciales para el mismo target con una frecuencia similar a aquella observada en el corpus. Evaluamos emp�ricamente el nuevo algoritmo sobre el corpus GR3D7, y mostramos que la distribuci�n de probabilidad de las expresiones referenciales generadas por el algoritmo machean las encontradas en el corpus con alta precisi�n. Tambi�n comparamos nuestros resultados con los resultados the \cite{graph08}, los ganadores de la competencia NLG Challenge sobre selecci�n de atributos para generaci�n de expresiones referenciales (ASGRE). Proveemos un an�lisis de error y conclusiones para ambos corpus. Se cre� un nuevo corpus de ER sobre el cual se describe y se dan los resultados obtenidos, este nuevo corpus realizado sobre un entorno mucho m�s natural nos d� la certeza de que nuestro algoritmo se puede aplicar a sistemas de la vida real.
}


\newpage
\begin{center}

{ \vspace*{1cm} }
\huge{\textbf{\textsc{\textmd{Abstract}}}}\\[1cm]
%\Large{\textsc{\textmd{Facultad de Matem\'atica, Astronom\'ia y F\'isica}}}\\[1cm]

\end{center}

\normalsize{
We propose an algorithm for the generation of referring expressions (REs) that adapts the approach of~\cite{arec2:2008:Areces,arec:usin11} 
to include overspecification and probabilities learned from corpora.  After introducing the algorithm, we discuss how probabilities required as 
input can be computed for any given domain for which a suitable corpus of REs is available, and how the probabilities can be adjusted for new scenes in the domain using a machine learning approach.  
We exemplify how to compute probabilities over the GRE3D7 corpus of~\cite{viet:gene11} \textit{and the TUNA corpus of ~\cite{gatt-balz-kow:2008:ENLG}}.
The resulting algorithm is able to generate different referring expressions for the same target with a frequency similar to that observed in corpora. 
Moreover, the most frequently generated referring expressions not in the corpus are also semantically natural, indicating that the algorithm 
can generalize from the learning data. \\
We empirically evaluate the new algorithm over the GRE3D7 corpus, and show that the probability distribution of the generated referring expressions 
matches the one found in the corpus with high accuracy. We also compare our results with the results of~\cite{graph08} the winners of the NLG Challenge on Attribute Selection for Generating Referring Expressions (ASGRE). We provide error analysis and conclusion for both corpus. 
In this thesis we create a new corpora or REs but the new corpora is much more natural, the REs create by the algorithm for this new corpora are shown with those results we can say that our algorithm is usefull in systems of the real life.
}
