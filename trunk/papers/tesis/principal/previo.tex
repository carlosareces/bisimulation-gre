La generación de expresiones referenciales (GER) es una tarea clave para muchos sistemas sistemas de procesamiento de lenguaje natural (PLN) \cite{reiter-dale-2000}. Dependiendo de la información disponible al sistema de PLN, ciertos objetos pueden no ser asociados a un  identificador el cual puede ser fácilmente reconocido por el usuario. En estos casos el sistema    tendrá que generar una posiblemente compleja descripción que contenga información suficiente para que el interlocutor sea capaz de identificar el referente.

La generación de expresiones referenciales es un campo muy estudiado en generación de lenguaje natural. Construído sobre el trabajo fundacional de \cite{winograd-1972}, dale-1989, dale-reiter-1995, varias propuestas han


