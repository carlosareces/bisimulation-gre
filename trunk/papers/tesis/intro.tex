
\subsection{Descripci\'on del problema}


En linguistica una expresi\'on referencial (RE) es una expresi\'on que identifica un\'{ì}vocamente a un objeto de para un interlocutor, 
desde un conjunto de posibles distractores. Por ejemplo si nosotros queremos identificar a un cierto animal d de un conjunto de mascotas, 
la expresi\'on ``el perr'' ser\'a ER si d es el \'unico perro en el conjunto, y si nosotros estamos seguros que nuestro interlocutor 
identificar\'a a d como un perro.


\subsection{Contribuciones}

\begin{itemize}
 \item Un algoritmo para la generaci\'on de expresiones referenciales independiente de dominio.
 \item
\end{itemize}

\subsection{Mapa de la tesis}
Esta tesis se divide en 6 cap\'{i}tulos, en el primer cap\'{i}lo ...

