\section{Generation of referring expressions}\label{sec:gre}

\fxnote{Carlso working here}

Automatically generate referential expresion is a goal of a lot's of systems, but the expresion that they generate in the most of cases are deterministics and not overspecified, or sobrespecified when they tell every attributes that the thing has.
The algorithm of~\cite{areces08} is based in modal logics for adquisition of referential expresion...

In the first part of this paper we resume the previos work in the area, after that we present the novell algorithm, in the tirth part we apply the algorithm to a manual corpus, generate with the algorithm and compare results, after that we generate for another picture and show the very good results. Two metrics are studied for the test, precision and recall. We also give some future lines of investigation about this topic.  


\cite{paig:thre87}

The generation of referring expressions (GRE) is one of the most
active and successful research areas in natural language generation.
Building upon Dale and Reiter's work \cite{dale89cooking,Dale1995},
various researchers have added extensions such as reference to sets
\cite{Stone2000}, more expressive logical connectives
\cite{deemter02:_gener_refer_expres}, and relational expressions
\cite{dale91:_gener_refer_expres_invol_relat}.

Referring expressions (REs) involving relations, in particular, have
received increasing attention recently; especially in the context of
spatial referring expressions in situated generation (e.g.\
\cite{kelleher06:_increm_gener_of_spatial_refer}), where it seems
particularly natural to use expressions such as ``the book on the
table''.  However, the classical algorithm by
\cite{dale91:_gener_refer_expres_invol_relat} was recently shown to
be unable to generate satisfying REs in practice
\cite{viethen06:_algor_for_gener_refer_expres}.  Furthermore, the Dale
and Haddock algorithm and most of its successors (such as
\cite{kelleher06:_increm_gener_of_spatial_refer}) are vulnerable to
the problem of ``infinite regress'', where the algorithm jumps back
and forth between generating descriptions for two related individuals
infinitely, as in ``the book on the table which supports a book on the
table \ldots''.


In this paper, we propose to view GRE as the problem of computing a
formula of description logic (DL) that denotes exactly the set of
individuals that we want to refer to.  This very natural idea has been
mentioned in passing before
\cite{Krahmer2003,gardent07:_gener_bridg_defin_descr}; however, we
take it one step further by proposing DL as an interlingua for
comparing the REs produced by different approaches to GRE.  In this
way, we can organize existing GRE approaches in an expressiveness
hierarchy.  For instance, the classical Dale and Reiter algorithms
compute purely conjunctive formulas;
\cite{deemter02:_gener_refer_expres} extends this language by
adding the other propositional connectives, whereas
\cite{dale91:_gener_refer_expres_invol_relat} extends it by
allowing existential quantification.

Furthermore, the view of GRE as a problem of computing DL formulas
with a given extension allows us to apply existing algorithms for the
latter problem to obtain efficient algorithms for GRE.  We present
algorithms that compute such formulas for the description logics \el\
(which allows only conjunction and existential quantification).  These algorithms effectively
compute REs for all individuals in the domain at the same time, which
allows them to systematically avoid the infinite regress problem.  The
\el\ algorithm is capable of generating 67\% of the relational REs in
the \cite{viethen06:_algor_for_gener_refer_expres} dataset, in about
15 milliseconds.  


The paper is structured as follows.  In Section~\ref{sec:overspecification}, we
TALK ABOUT NON DETERMINISM AND OVERSPECIFICATION.  We will then show how to
generate REs by computing DL similarity sets for  \el\ in
Section~\ref{sec:algorithm}.  In Section~\ref{sec:corpus}, we SHOW HOW TO COMPUTE
PROBABILITY OF USE AND PROBABILITY OF DISCERNIBILITY FROM CORPUS.  Section~\ref{sec:usecases}
WE DISCUSS INTERESTING EXAMPLES. Section~\ref{sec:conclusion} concludes and points to future work.

