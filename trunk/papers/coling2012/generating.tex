\section{Generation of referring expressions}\label{sec:gre}

In linguistics, a \emph{referring expression} (RE) is an expression that 
unequivocally identifies the intended target to the interlocutor, from a set of possible distractors.  
For example, if we intend to identify a certain animal $d$ from a set of pets, the expression 
``the dog'' will be an RE if $d$ is the only dog in the set, and if we are confident
that our interlocutor will identify $d$ as a dog. 

The generation of referring expressions (GRE)  is a key task of most natural 
language generation (NLG) systems (e.g.,~\cite[Section 5.4]{dale2000}). 
Depending on the information available to the NLG system, certain objects might 
not be associated with an identifier which can be easily recognized by the user. 
In those cases, the system will have to generate a, possibly complex, description that contains 
enough information so that the interlocutor will be able to identify the intended referent.

The generation of referring expressions is a well developed field in automated natural language generation.
Building upon Dale and Reiter's foundational work~\cite{dale89cooking,Dale1995},
various proposals have investigated the generation of different kinds of referring expressions 
such as relational expressions (``the blue ball next to the box''~\cite{dale91:_gener_refer_expres_invol_relat}),
reference to sets (``the two small boxes''~\cite{Stone2000}), or more expressive logical connectives (``the 
blue ball not in the box''~\cite{deemter02:_gener_refer_expres}).

Referring expressions involving relations, in particular, have
received increasing attention recently; especially in the context of
spatial referring expressions in situated generation (e.g.~\cite{kelleher06:_increm_gener_of_spatial_refer}), where it seems
particularly natural to use expressions such as ``the ball next to the box''.  However, the classical algorithm by~\cite{dale91:_gener_refer_expres_invol_relat} was shown to be unable to generate satisfying REs in practice~\cite{viethen06:_algor_for_gener_refer_expres}.  Furthermore, the Dale
and Haddock algorithm and most of its successors (such as~\cite{kelleher06:_increm_gener_of_spatial_refer}) are vulnerable to
the problem of ``infinite regress'', where the algorithm jumps back
and forth between generating descriptions for two related individuals
infinitely, as in ``the book on the table which supports a book on the
table \ldots''.

In~\cite{arec:refe08,arec:usin11} new algorithms have been proposed for the generation 
of relational REs (including references to sets) that eliminate the risk of infinite regression. 
These algorithms are based on different variations of the partition refinement algorithms of~\cite{paig:thre87}.
Given a knowledge base containing the information about a situation, the information is 
interpreted as a labeled graph.  The algorithm, then, partitions the graph nodes into sets that can 
be described by the same description.  This partition is successively refined (while possible) till the target 
is the only element fitting the description of its part.  The existence of a RE for the target will 
depend, on the one hand, on the information available in the input knowledge base, and on the other 
on the expressive power of the formal language used to describe the elements of the different 
parts in the partition (e.g., whether the REs can contain negations, or relational information). 

The idea of using a formal language to describe the information that an RE should convey has been mentioned
already in~\cite{Krahmer2003,gardent07:_gener_bridg_defin_descr}.  In~\cite{arec:refe08,arec:usin11} the 
combination of partition refinement algorithms and different formal languages are used to organize existing 
GRE approaches in an expressiveness hierarchy.  For instance, the classical Dale and Reiter algorithms
compute purely conjunctive formulas; \cite{deemter02:_gener_refer_expres} extends this language by
adding the other propositional connectives, whereas\cite{dale91:_gener_refer_expres_invol_relat} extends it by
allowing existential quantification.


============================

THESIS DE JETTE


Furthermore, the view of GRE as a problem of computing DL formulas
with a given extension allows us to apply existing algorithms for the
latter problem to obtain efficient algorithms for GRE.  We present
algorithms that compute such formulas for the description logics \el\
(which allows only conjunction and existential quantification).  These algorithms effectively
compute REs for all individuals in the domain at the same time, which
allows them to systematically avoid the infinite regress problem.  The
\el\ algorithm is capable of generating 67\% of the relational REs in
the \cite{viethen06:_algor_for_gener_refer_expres} dataset, in about
15 milliseconds.  


The paper is structured as follows. We will show how to generate REs by computing DL similarity sets for  \el\ in Section~\ref{sec:algorithm} preferring properties that are more comonly found in corpora.  In Section~\ref{sec:learning}, we SHOW HOW TO estimate the 
PROBABILITY OF USE of a property by generalizing from training data. Section~\ref{sec:overspecification}
WE propose a modified algorithm that generate the kind of overspecification found in corpora. Section~\ref{sec:evaluation} evaluation presents a cuantitative evaluation of our REG algorithm and we discusses INTERESTING EXAMPLES. Section~\ref{sec:discussion} discusses previous work, applications and motivations of our proposal. Section~\ref{sec:conclusion} concludes and points to future work.

