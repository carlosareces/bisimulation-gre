\section{Discussion}
\label{sec:discussion}

In this paper we extend Areces et al.~\shortcite{arec2:2008:Areces} algorithm to generate REs similar to those produced by humans. The extensions done to the algorithm are based on two observations, on the one hand, there is no ordering of properties that is able to generate all the REs produced by humans and humans frequently overspecify their REs. We obtain an algorithm that is able to generate a large proportion of the overspecified REs found in corpora without generating bad referring expressions for our domain.

In previous work~\cite{delucena-paraboni:2008:ENLG,delucena-paraboni:2009:ENLG}, algorithms have been proposed for using redundant information following the redundancy observed in corpora. The algorithm proposed by De Lucena and Parabony can only handle propositional information, no relational REs can be produced by it. The authors report that their accuracy on the TUNA-AR corpus~\cite{gatt-balz-kow:2008:ENLG} is 33s\%. The descriptions generated by this algorithm are complete referring expressions that uniquely identify the target and potentially include redundant information. 

In~\cite{viet:gene11} trains decision trees that are able to achieve a 65\% average accuracy on the GRE3D7 corpus, the same corpus that we use for our experiments. The decision tress are able to generate overspecified relational descriptions. Viethen proposal has the problem that the generated descriptions may not be referring expressions, that is, the decision trees do not consider a complete model of the scene that is being described and hence cannot make sure that the generated description are distinguishing.  

To the best of our knowledge, our algorithm is the first to be proposed that is able to generate relational referring expressions which may include redundant information in a similar fashion that redundant information is used in corpora in the domain. Our algorithm achieves a XX\% average accuracy on the GRE3D7 corpora. 

Our short term plans of future work include evaluating our algorithm in another domain, such as the TUNA-AR corpus. 


