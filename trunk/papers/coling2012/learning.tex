\section{Learning to describe new objects from corpora}
\label{sec:learning}

In the previous section we presented an algorithm that assumes that each property used in the referring expressions has a known probability of use. In this section, we describe how to calculate these probabilities from corpora when there is one available, and how to estimate them if there is no such corpora. This section starts by describing the corpora that we use for calculating the probabilities of use. Then we present the method used in the calculation. And finally we propose a methodology for estimating the probability of use of properties when describing objects for which no corpora is available. 

\subsection{A corpus of referring expressions}

The corpus we used is known as the GRE3D7 and consists of 4480 referring expressions that describe 16 object in a 3D scene. Each scene contained a small number of simple objects (cubes and balls), and the individual descriptions were elicited in the absence of a preceding discourse. The stimulus scenes were designed in a way that encourage the use of relations between objects, but did not require them. For a detailed description of the collection produre see~\cite[Chapter 5]{viet:gene11}. A sample stimulus used in the corpus collection is shown in Figure~\ref{GRE3D7-stimulus}. 


\subsection{Calculating the probability of use}



\subsection{Learning to describe new objects}
