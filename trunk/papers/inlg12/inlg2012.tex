%
% File inlg2012.tex
%

\documentclass[11pt,letterpaper]{article}
\usepackage{inlg2012}
\usepackage{times}
\usepackage{latexsym}
\usepackage[draft]{fixme}
\usepackage{xspace}
\usepackage{graphicx}
\usepackage[algoruled, linesnumbered,noend]{algorithm2e}

\setlength\titlebox{6.5cm}    % Expanding the titlebox

\newcommand{\pdisc}{\small \textsf{p\_disc}\xspace}
\newcommand{\puse}{\small \textsf{p\_use}\xspace}

\newcommand{\RE}{\textsf{RE}\xspace}
\newcommand{\REL}{\textsf{REL}\xspace}
\newcommand{\IR}{\textrm{I}\!\textrm{R}}
\newcommand{\gM}{\mathcal{M}}
\newcommand{\el}{\mathcal{EL}}
\newcommand{\interp}[1]{|\!|#1|\!|}

\title{Enriching a referential expresion with overspecification and nondeterminism}



%%Instructions for INLG 2012 Proceedings\Thanks{This
%%    document has been adapted from the instructions for earlier ACL
%%    and NAACL proceedings, including those for NAACL-HLT-12, 
%%    NAACL-HLT-10 by Claudia Leacock and Richard Wicentowski,
%%    NAACL-HLT-09 by Joakim
%%    Nivre and Noah Smith, for ACL-05 by Hwee Tou Ng and Kemal Oflazer,
%%    for ACL-02 by Eugene Charniak and Dekang Lin, and earlier ACL and
%%    EACL formats.  Those versions were written by several people,
%%    including John Chen, Henry S. Thompson and Donald Walker.
%%    Additional elements were taken from the formatting instructions of
%%    the {\em International Joint Conference on Artificial
%%      Intelligence}.}}

\author{Ivana Romina Altamirano 
	  \ \ \ \ \ 
	Carlos Areces
	  \ \ \ \ \ 
	Luciana Benotti\\
	    FaMAF, Universidad Nacional de C\'ordoba\\
	    Haya de la Torre s/n\\
	    C\'ordoba, 5000, Argetina\\
	    	    {\tt \{ialtamir, careces, benotti\}@famaf.unc.edu.ar}	    }

\date{}

\begin{document}
\maketitle
\begin{abstract}
  This paper present an algorithm with a new form of adquire automatically reference expresion with overspecification and nondeterminism. We modify the algorithm described in~\cite{arec2} adding probabilities of use and discernibility extracted from a corpora in order to generate referential expresion like we see in a manual corpus of referential expresion. We will show how the algorithm can learn probabilities from the manual corpus of some picture make by~\cite{viethen-dale} and use those probabilities in another pictures with similarity vocabulary with successfull results. We achieve \% precision and \% recall. 
\end{abstract}

\paragraph{Main contributions}
\begin{itemize}
\item We address needs discussed in deemter et al.\ article on non determinism and overspecification
\item We show how to implement the ideas as a modification of areces et al.\ algorithm
\item We show how to compute \pdisc y \puse from corpus
\item We test in interesting examples
\end{itemize}

\documentclass[11pt]{article}

\usepackage{acl08}
\usepackage{times}
\usepackage{url}
\usepackage{graphicx}
\usepackage{xspace}
\usepackage{amsthm}
\usepackage{bm}
\usepackage{multirow}

\usepackage[algoruled, linesnumbered,noend]{algorithm2e}

\newcommand{\gM}{\ensuremath{\mathcal{M}}}
\newcommand{\gL}{\ensuremath{\mathcal{L}}}

\newcommand{\C}{\ensuremath{\mathcal{C}}\xspace}
\newcommand{\M}{\ensuremath{\mathcal{M}}\xspace}
\newcommand{\RE}{\ensuremath{\mathit{RE}}\xspace}

\newtheorem{definition}{Definition}[section]
\newtheorem{theorem}{Theorem}[section]

\newcommand{\todo}[1]{\textbf{(#1)}}
\newcommand{\ignore}[1]{}

\newcommand{\el}{\ensuremath{\mathcal{EL}}\xspace}
\newcommand{\alc}{\ensuremath{\mathcal{ALC}}\xspace}
\newcommand{\form}{\mathsf{form}\xspace}
\newcommand{\prop}{\ensuremath{\mathsf{prop}}\xspace}
\newcommand{\simm}{\textsf{sim}}
\newcommand{\rel}{\ensuremath{\mathsf{rel}}\xspace}
\newcommand{\propm}{\ensuremath{\mathsf{prop}^\gM\xspace}}

\newcommand{\interp}[1]{|\!|#1|\!|}

\title{}
\author{
Romina Altamirano \url{}
\And
Carlos Areces \url{}
\And
Luciana Benotti \url{}
}

\date{}

%\setlength\titlebox{2.25in}    % Expanding the titlebox

\begin{document}

\maketitle

%\input abstract
%\input introduction
%\input bisim

\newcommand{\REL}{\textsf{REL}\xspace}
\newcommand{\IR}{\textrm{I}\!\textrm{R}}
\newcommand{\puse}{\textit{p}\_\textit{use}}
\newcommand{\pdisc}{\textit{p}\_\textit{disc}}

\REL is the set of relation symbols in the signature. 

For each $R \in \REL$ we asume defined two values $R.\puse \in \IR$, the 
probability of using relation $R$ in a RE; and $R.\pdisc \in \IR$, the 
probability that the hearer recognizes relation $R$. 

\begin{algorithm}[h]
\dontprintsemicolon
\caption{Computing the $\mathcal{L}$-similarity sets}
\label{algo:bisim-l}
\KwIn{A model $\gM = (\Delta, \interp{\cdot})$ and a list $Rs \in \REL^*$ ordered by $R.\puse$}
\KwOut{A set \RE of pairs of formulas such that
$\{\interp{\varphi_R} \mid (\varphi_O,\varphi_R) \in \RE\}$ is the set of
$\mathcal{L}$-similarity 
sets of $\gM$.}

$\RE \leftarrow \{(\top,\top)\}$

\While{$\exists (\varphi_O,\varphi_R) \in \RE. |\interp{\varphi_O}|>1$}{
    \RE'.\emph{Obs} $\leftarrow$ \RE.\emph{Obs} \;
    \For{$R \in Rs$}{
        \If{random() $\le R.\puse$}{
            \For{$P \in \RE$}{
                add$_\mathcal{L}(R, P, \RE)$}
                }\;
            \If{\RE.Obs $\not =$ \RE'.Obs}{exit}
            }
   \If{\RE.Obs $=$ \RE'.Obs}{exit}
}
\end{algorithm}




%\begin{algorithm}[h]
%\caption{add$_\alc(R, (\varphi_O, \varphi_R), \RE)$}
%\label{algo:bisim-add-alc}
%\If{random() $\le R.\pdisc$}{
%\For{$(\psi_O,\psi_R) \in \RE$ with $|\interp{\psi_O}| > 1$}{
%         remove $(\psi_O,\psi_R)$ from \RE \;
%         add $(\psi_O \sqcap \exists R.\varphi_O,\psi_R \sqcap \exists R.\varphi_R)$ 
%         and $(\psi_O \sqcap \neg \exists R.\varphi_O,\psi_R \sqcap \neg \exists R.\varphi_R)$ to \RE
%   }
%}
%\Else{
%\For{$(\psi_O,\psi_R) \in \RE$ with $|\interp{\psi_O}| > 1$}{
%         remove $(\psi_O,\psi_R)$ from \RE \;
%         add $(\psi_O,\psi_R \sqcap \exists R.\varphi_R)$ 
%         and $(\psi_O,\psi_R \sqcap \neg \exists R.\varphi_R)$ to \RE
%   }

%}
%\end{algorithm}

\begin{algorithm}[h]
\dontprintsemicolon
\caption{add$_\el(R, (\varphi_O, \varphi_R), \RE)$}
\label{algo:bisim-add-el}
\If{random() $\le R.\pdisc$}{
\For{$(\psi_O,\psi_R) \in \RE$ with $|\interp{\psi_O}| > 1$}{
  \If{$\psi_O \sqcap \exists R.\varphi_O$ is not subsumed in $\RE$ {\bf and}
    $\interp{\psi_O \sqcap \exists R.\varphi_O} \neq \emptyset$ {\bf and}
    $\interp{\psi_O \sqcap \exists R.\varphi_O} \neq \interp{\psi_O}$}{
    add $(\psi_O \sqcap \exists R.\varphi_O, \psi_R \sqcap \exists R.\varphi_R)$ to $\RE$ \;
    remove subsumed formulas from $\RE$\;
  }
}
}
\Else{
\For{$(\psi_O,\psi_R) \in \RE$ with $|\interp{\psi_O}| > 1$}{
  \If{$\psi_O \sqcap \exists R.\varphi_O$ is not subsumed in $\RE$ {\bf and}
    $\interp{\psi_O \sqcap \exists R.\varphi_O} \neq \emptyset$ {\bf and}
    $\interp{\psi_O \sqcap \exists R.\varphi_O} \neq \interp{\psi_O}$}{
    add $(\psi_O, \psi_R \sqcap \exists R.\varphi_R)$ to $\RE$ \;
    remove subsumed formulas from $\RE$\;}
  }
}
\end{algorithm}


%\input discussion
%\input related
%\input conclusion

%\bibliographystyle{acl}
%\bibliography{bibliography}


\end{document}



\section{Non determinism and Overspecification}\label{sec:overspecification}

\fxnote{Lu working here}

Both in psycholinguistics and computational linguistics, researchers have addressed aspects of the problem of generation of referring expressions, but we still have an incomplete understanding of how the human production of referring expressions works. This paper is part of a larger effort which rests on the assumption that for a better, more complete understanding of this process, it is important to bridge the gap between the two disciplines. Such a bridge has both theoretical and methodological advantages. Psycholinguistics has important insights to offer in the human production of referring expressions and a methodology involving carefully constructed and controlled experiments; computational linguistics a well-established approach involving corpus analysis and computational modeling.

Recent studies have suggested that redundant information is frequent in the referring expressions produced by people (e.g., Arts, 2004; P. E. Engelhardt, Bailey,  Ferreira, 2006), that is people generate \emph{overspecified} referring expressions. This is consistent with the Incremental Algorithm~\cite{dale95}, which predicts overspecification in specific situations. In fact, the IA makes some precise predictions concerning overspecification: assuming that colour is preferred over size, then a generated description can include colour as an overspecified property, but not size (after all, if colour were suficient to distinguish a target, the IA would select colour and then terminate, so it would not consider less preferred properties). This is an interesting prediction that has never been
tested experimentally

Perhaps the most striking property of most computational algorithms that is problematic from a psycholinguistic point of view is their \emph{determinism}: they always generate the same referring expression in a particular situation or condition. For example, in a situation where there is no other object of the same category as the target object---say, a single car---most algorithmic models either always generate minimally specified expressions---the car---or always generate overspecified expressions---the red car. But given this specific situation, they would not generate a minimally specified expression in some cases and an overspecified expression in others. This contrasts with the results from experiments with human speakers, which show that they produce various types of referring expressions in a specific condition. For example,~\cite{dale10} showed that even when referring to simple objects in simple scenes, different speakers used a large variety of referring expressions to refer to the same object, while the same speaker was likely to vary their choice of referring expression considerably in very similar (or even isomorphic) scenarios. The results of experimental studies are normally reported averaged across participants, so they do not report whether individual human speakers are deterministic. However, closer examination of the data of individual participants reveals that their responses vary substantially, even within a single experimental condition. For example,~\cite{deemter12} examined the data of~\cite{Fukumura10}, who conducted experiments that investigated the choice between a pronoun and a name for referring to a previously mentioned discourse entity. The clear majority (79\%) of participants in their two main experiments behaved non-deterministically, that is, they produced more than one type of referring expression (i.e., both a pronoun and a name) in at least one of the conditions.

Consider the Incremental Algorithm (IA) once again~\cite{dale95}. The original, deterministic version always generates the cup to refer to a black cup in the
presence of a blue ashtray and yellow candle. The reason is that it assumes a fixed preference order, causing it to check the category of the object (cup) before its colour (red), and since cup rules out both distractors, colour is not tried. Research by~\cite{pechmann89} suggests that speakers do produce overspecified expressions
such as the red cup in this situation. To account for this, the IA could be revised slightly, so that colour is selected first when it is a discriminating feature. But this would still not fully account for Pechmann's results, because he showed that although overspecified expressions (e.g., the black cup) are produced most frequently, minimally specified expressions are produced on one quarter of the trials. To account for this, the ia would need to incorporate some form of non-determinism. One possibility would be to include a random process by which the algorithm checks colour before type three-quarters of the time and type before colour in the remaining quarter (both across speakers and within a single speaker). If we assume that the decision about which property is checked first is a probabilistic, non-deterministic process, then the algorithm makes interesting predictions that are relevant to psycholinguists. For example, a non-deterministic version of the Incremental Algorithm makes exact, quantitative predictions about when overspecification occurs. Although several psycholinguistic studies have shown that overspecification is common, it remains unclear under exactly what conditions it occurs and psycholinguistic models do not make clear predictions concerning this issue. We therefore believe that the algorithm provides an important step towards a better understanding of the possible psychological mechanisms involved in overspecification.

In this paper we address the concrete questions: ``Why and how do speakers overspecify their references?'' and ``Why and how do speakers produce different referring expressions in isomorphic scenarios?''. In other words, in this paper we propose and test a non-deterministic algorithm that is able to generate overspecified referring expressions.  


%%%% p_use p_disc

The probability of use of a property p\_i is the probability that p\_i is included in a referring expression without considering how discriminating this property is in the referring context. This property is related with ‘salience’; in general, if the salience of a property is high, its use probability is high due to the fact that is a property that ‘comes to mind’ easily and the person may decide to include it in the referring expression before even considering its discriminatory power in context. For instance, bright colors or unusual sizes are properties with high probability of use because they catch the speaker attention very easily. However, as we define it, the probabilty of use is not the same as salience, since probability of use is defined independently of the particular context in which the referring expression is produced, while salience is in general defined with respect to the context. 

\begin{verbatim}
p_use (p_i) = (# uses of p_i) / (# of RE)
\end{verbatim}
 
The probability of discrimination of a property p\_i is the probability that p\_i is understood by the audience (i.e., the interpreter of the referring expression). By understood we mean that the interpreter is able to identify the target as an object that exhibits the property p\_i. The property of discrimination is affected by different factors: The first factor is the certainty that the property p\_i is true of the target (for instance, gradable properties are not certain because people might have different opinions of whether something is small or middle sized). The second factor is the certainty that the audience will know about the property p\_i (for instance, a technical property will have a low probability of discrimination). The third factor is lexical, does the audience how the property p\_i, is realized in the target language?

\begin{verbatim}
p_disc = \sum ( 1 \ length RE) \ #RE donde aparece p_i
\end{verbatim}

\section{The algorithm}\label{sec:algorithm}

\fxnote{Romina workign here}

\section{Computing \pdisc and \puse}\label{sec:corpus}

\fxnote{Lu working here}

\section{Use cases}\label{sec:usecases}

\fxnote{Carlos y Romina working here}

FIND INTERESING EXAMPLES (e.g., failures of overspecification of the IA algorithm). 

\begin{enumerate}

\item Incluir relaciones: comparacion con el algoritmo de EL anterior que priorizaba 
propiedades. 

%\item Overspecification mediante \puse.  Example where with one fix order, the IA algorithm cannot generate two overspecified expressions. Ejemplo: orden "type, color, size" modelo con dos esferas una grande una chica una azul, una verde" Dos expresiones overspecified "ball, blue" y "ball, big" pero cualquier orden va a priorizar color sobre size o vice-versa, por lo que puede generar solo una de ellas.  Buscar en el corpus un ejemplo similar. 

%\item Overspecification mediante \pdisc: "ball, blue, big" con modelo igual al anterior.  El algoritmo IA generaria solo "ball, blue" o "ball, big" pero no puede seguir overspecifing una vez que el target se alcanzo.  Nuestro algoritmo puede hacerlo si falla \pdisc. 

%\item Comparar No-determinismo random vs. no-determinismo guiado por corpus.  Experimento, tomar modelo de 7 elementos, hacer 20 corridas con el algoritmo usando \pdisc y \puse del corpus, y 20 corridas con \pdisc y \puse al azar, y comparar con el corpus.  

\item Comparation with Gatt et al. 2011
\begin{itemize} 
  \item \pdisc corpus\pdisc
  \item \pdisc model\pdisc
  \item \pdisc AVG(corpus\pdisc, model\pdisc)
\end{itemize}
\item Jette's corpus non determinism random vs. non determinism with probabilities

\end{enumerate}

\section{Conclusions}\label{sec:conclusions}

%\fxnote{\tiny  I would like to rewrite the conclusions also. It reads to weak
%and too vague. Empezar repitiendo cual es el punto principal del paper y cual
%fue la contribucion.  Lo que hay que decir ya esta dicho en lo que esta escrito,
%pero falta decirlo more 'to the point' y sin dar tantas vueltas.}

The content determination phase during the generation of referring
expressions identifies which `properties' will be used to refer to a
given target object or set of objects. What is considered as a
`property' is specified in different ways by each of the many
algorithms for content determination existing  in the literature. In
this article, we put forward that this issue can be addressed by
deciding when two elements should be considered to be equal, that
is, by deciding which discriminatory power we want to use. Formally,
the discriminatory power we want to use in a particular case can be
specified syntactically by choosing a particular formal language, or
semantically, by choosing a suitable notion of simulation.  It is
irrelevant whether we choose first the language (and obtain the
associated notion of simulation afterwards) or vice versa.

We maintain that having both at hand is extremely useful. Obviously,
the formal language will come handy as representation language for
the output to the content determination problem.  But perhaps more
importantly, once we have fixed the expressivity we want to use, we
can  rely on model theoretical results defining the adequate notion
of sameness underlying each language, which indicates what can and
cannot be said (as we discussed in \sect{technical}). Moreover, we
can transfer general results from the well-developed fields of
computational logics and graph theory as we discuss
in~\sect{simulation} and~\sect{krahmer}, where we generalized known
algorithms into \emph{families} of GRE algorithms for different
logical languages. %These notions were crucial also in
%\sect{combining} to devise new heuristics.


%Each language has its own expressive power, and
%this induces an adequate notion of sameness. Naturally this last
%notion becomes crucial when trying to uniquely refer to an element
%in a model.


%We have discussed making the notion of expressiveness involved an
%explicit parameter of the GRE problem, unlike usual practice. Hence
%we talk of ``$\+L$-GRE problem'', for a logical language $\+L$. We
%considered various possible choices for $\+L$, though we did not
%argue for any of them. Instead, we tried to make explicit the
%trade-off involved in the selection of a particular $\+L$. This, we
%believe, depends heavily on the given context.


An explicit notion of expressiveness also provides a
cleaner interface, either between the content determination and
surface realization modules or between two collaborating content
determination modules. An instance of the latter was exhibited in
\sect{combining}.

As a future line of research, one may want to avoid sticking to a
fixed $\+L$ but instead favor an incremental approach in which
features of a more expressive language $\+L_1$ are used only when
$\+L_0$ is not enough to distinguish certain element.

%\fixme{Podemos generalizar el algoritmo para conjuntos? Mencionar algoritmo de Piazza}

%To finish this section, observe that the algorithms introduced here
%can also be used to compute referring expressions for \emph{sets} of
%elements. Let $\+L$ be \EL or \ELAN. For any $v\in\Delta$ we define
%the $\+L$-class of $v$ as
%%\fixme{$[v]_{\+L}$ pide demasiado. alcanza con que sean todos los $u$ tq $v \simul{\+L} u$, no?}
%$$
%[v]_{\+L}=\{u\in\Delta\mid u\in\simset_{\+L}(v)\wedge
%v\in\simset_{\+L}(u)\}.
%$$
%A set $T\subseteq\Delta$ has an $\+L$-RE iff $T=[u]_{\+L}$ for some
%$u\in\Delta$. In case $T=[u]_{\+L}$ for some $u$ then for any
%$v\in[u]_{\+L}$, $F(v)$ is a $\+L$-RE for $T$.
%
%Since computing the $\+L$-classes of $\Delta$ is polynomial in
%$\size{\Delta}$, Theorem \ref{thm:complexity-EL-GRE} implies the
%following:


%\fixme{ARREGLAR} To finish this section, observe that the algorithms
%introduced here can also be used to compute referring expressions
%for some \emph{sets} of elements. Let $\+L$ be \EL or \ELAN. A set
%$T\subseteq\Delta$ has an $\+L$-RE iff $T=\simset_{\+L}(u)$ for some
%$u\in\Delta$. In case $T=\simset_{\+L}(u)$ for some $u$ then $F(u)$
%is a $\+L$-RE for $T$. In fact, $F(v)$ also is, for any $v$ such
%that $u\in\simset_{\+L}(v)\wedge v\in\simset_{\+L}(u)$.

%Hence by Theorem \ref{thm:complexity-EL-GRE} we have:

%\begin{cor}
%The problem of generating the \EL and \ELAN referring expressions of
%sets of elements given a finite model
%%$\gM=\tup{\Delta,\interp{\cdot}}$
%can be solved in polynomial time.
%\end{cor}



\section*{Acknowledgments}

\bibliographystyle{acl}
\bibliography{bibliography}


\end{document}
