\section{Background} \label{background}

REs involving relations have
received increasing attention recently; especially in the context of
spatial referring expressions in situated generation (e.g., \cite{kelleher06:_increm_gener_of_spatial_refer}), where it is
particularly natural to use expressions involving spatial relations such as ``the ball on top of the cube.''  However, the classical algorithm by~\shortcite{dale91:_gener_refer_expres_invol_relat} was shown to be unable to generate satisfying REs in practice (see the 
analysis over the \emph{cabinet corpus} in~\cite{viethen06:_algor_for_gener_refer_expres}).  Furthermore, the Dale
and Haddock algorithm and many of its successors (such as~\cite{kelleher06:_increm_gener_of_spatial_refer}) are vulnerable to
the problem of \emph{infinite regress}, where the algorithm enters an infinite loop, jumping back
and forth between descriptions for two related individuals, as in ``the book on the table which supports a book on the
table \ldots''

\shortcite{arec2:2008:Areces,arec:usin11} have proposed low complexity algorithms for the generation 
of relational REs 
%(including references to sets) 
that eliminate the risk of infinite regression. 
These algorithms are based on variations of the partition refinement algorithms of~\shortcite{paig:thre87}.
The information provided by a given scene is interpreted as a relational model whose 
objects are classified into sets that fit the same description.  
This classification is successively \emph{refined}  till the target 
is the only element fitting the description of its class.  The existence of an RE 
depends on the information available in the input scene, and on the expressive power of the formal 
language used to describe elements of the different classes in the refinement. 

Refinement algorithms %presented in~\cite{arec2:2008:Areces,arec:usin11} 
effectively compute REs for all individuals in the domain, at the same time. The algorithms always terminate returning a formula of 
the formal language chosen that uniquely describes the target (if the 
formal language is expressive enough to identify the target in the input model). 
%\shortcite{arec2:2008:Areces}
%show that the refinement algorithm using the description language \el  is capable of generating 67\% of 
%the relational REs in the~\cite{viethen06:_algor_for_gener_refer_expres} dataset, when all possible orders of the relations in the domain are considered. This is in sharp contrast with the analysis 
%done in~\cite{viethen06:_algor_for_gener_refer_expres} over the cabinet corpus, of algorithms based in Dale and Reiter's original proposal.    


