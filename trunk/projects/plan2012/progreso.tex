\section{Actividades de investigaci\'on desarrolladas}
\label{progreso}

\begin{itemize}
\item \textbf{Tareas de investigaci\'on:}
Estudi\'e un algoritmo existente de generaci\'on de expresiones referenciales que usa l\'ogica modal para dar la 
expresi\'on referencial~\cite{Areces2008} y lo modifique para dar expresiones referenciales m\'as humanas, para hacer 
esto inclu\'imos las probabilidades de uso de las palabras al algoritmo y estudiamos las posibilidades de agregar 
sobreespecificaci\'on a una expresi\'on referencial, estudi\'e la documentaci\'on de un corpus de expresiones 
referenciales~\cite{viethen-dale:2011:UCNLG+Eval} hecho por humanos e intentamos aprender con aprendizaje autom\'atico 
las probabilidades de uso para las figuras de ese corpus. Se evaluaron 8 figuras de las 32 figuras que aparec\'{\i}an en 
el corpus consiguiendo una buena precisi\'on con nuestro algoritmo. 
\item \textbf{Publicaciones:} 
Conjuntamente con mi directora Dra. Luciana Benotti y el Dr. Carlos Areces hemos escrito un art\'iculo cient\'ifico con los resultados de esta investigaci\'on. Submitimos este art\'iculo a \emph{The 24th International Conference on Computational Linguistics (COLING 2012)}. COLING 2012 es una conferencia de tipo A en el CORE ranking con una tasa de aceptaci\'on menor al 25\%. Sin embargo hay altas chances de que nuestro trabajo sea aceptado dado que los resultados que tenemos hasta el momento mejoran el estado del arte.    
\item \textbf{Software:} El c\'odigo fuente y la documentaci\'on del software realizado est\'an disponibles bajo licencia GNU Lesser GPL en \url{http://code.google.com/p/bisimulation-gre}.

\item Cursos realizados:
\begin{itemize}
\item Agosto del 2012 - Aprob\'e la materia de postgrado {\bf Generaci\'on de lenguaje natural}, la cual me da 3 cr\'editos para el doctorado.
\item A septiembre del 2011 ten\'{\i}a 6 cr\'editos; al d\'ia de la fecha tengo un total de 9 cr\'editos.
\end{itemize}
\item Asistencia a conferencias y escuelas:
\begin{itemize}
\item Asist\'{\i} al \emph{OpenMT-2 Workshop on Using Linguistic Information for Hybrid Machine Translation} el 18 de Noviembre del 2011 en  Barcelona Espa\~na.
\item Asist\'{\i} a la Escuela de Ciencias Inform\'aticas 2012 (ECI 2012) en donde realic\'e 2 cursos: ``Java para software de tiempo real'' 
y ``T\'ecnicas avanzadas para procesamiento del habla'' (aprobado).
\end{itemize}
\item Otros cursos:
\begin{itemize}
\item Aprob\'e el curso on-line ``Introducci\'on a la inteligencia artificial'' dictado por Dr. Sebastian Thrun y Dr. Peter Norvig de la 
universidad de Stanford.
\item Aprob\'e el curso on-line ``Aprendizaje autom\'atico'' dictado por Dr. Andrew Ng de la universidad de Stanford. 
\end{itemize}

\end{itemize}




