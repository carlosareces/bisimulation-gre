\section{Objetivos espec\'ificos}
\label{objetivos}

Si bien ya hemos hecho una evaluaci\'on de como funciona nuestro sistema de generaci\'on de expresiones referenciales, la evaluaci\'on realizada 
fue hecha en un dominio muy acotado, el cual es dif\'icil encontrar aplicacioens en la vida real. Ahora proponemos
extender la evaluaci\'on a un dominio muy amplio que es el de generar expresiones referenciales que identifiquen personas, extendiendo el
trabajo de \cite{pacheco-duboue-dominguez:2012:NAACL-HLT} e intentando demostrar que nuestra aproximaci\'on conseguir\'a mejores resultados.

Este mismo trabajo se integrar\'a al chatbot que esta realizando la Universidad en colaboraci\'on con la Fundaci\'on Sadosky.
\begin{itemize}

\item Obtenci\'on de los datos de corpus necesarios para armar un modelo el cual se intentar\'a realizar autom\'aticamente con datos de la web.
\item Obtenci\'on de las probabilidades de uso usando machine learning a partir de datos de la web.
\item Ejecusi\'on del algoritmo y an\'alisis de resultados.
\item Integraci\'on al chatbot.
\end{itemize}
