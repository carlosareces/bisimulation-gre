\section{Plan de Actividades 2011-2012}
\label{plandetrabajo}

En esta secci\'on detallamos el nivel de cumplimiento de los objetivos del plan de trabajo del a\~no anterior. 

{\footnotesize
\begin{center}
\begin{tabular}{|p{7cm}|p{7cm}|}
\hline
 \rowcolor[rgb]{0.8,0.8,0.8}\hspace{3.5cm}Tarea & Nivel de realizaci\'on\\
\hline 1. Lectura de la bibliograf\'ia citada. & Parcialmente realizado\\
\hline 2. An\'alisis de la implementaci\'on actual del algoritmo de Areces et al.~\cite{Areces2008}. & Completado\\
\hline 3. Extensi\'on a m\'ultiples expresiones referenciales por referente.  & Completado\\
\hline 4. Extensi\'on para expresiones referenciales sobre-especificadas.& Completado\\
\hline 5. Consideraci\'on de un orden de preferencias de propiedades. & Completado\\
\hline 6. Integraci\'on de probabilidades al algoritmo. & Completado\\
\hline 7. Aplicaci\'on a corpora y an\'alisis de resultados.& Completado\\
\hline 8. Adaptaci\'on a dominios din\'amicos. & No realizado\\
\hline 9. Integraci\'on del algoritmo en una aplicaci\'on con dominio din\'amico (por ejemplo GIVE). & No realizado\\
\hline 10. Generaci\'on de expresiones referenciales parciales. & Completado\\
\hline 11. Documentaci�n y publicaci\'on de resultados. & Completado (en evaluaci\'on)\\
\hline
\end{tabular}\end{center}
}

En lugar de realizar las tareas 8 y 9 este a\~no, se decidi\'o trabajar en aprender el orden de preferencia de propiedades desde corpora, adaptando algoritmos de aprendizaje autom\'atico para esta tarea. Esta decisi\'on se tom\'o con el objetivo de cerrar un a\~no de trabajo en resultados publicables y aplicable seg\'un los est\'andares del \'area de mi doctorado: la ling\"u\'istica computacional y, en particular, la generaci\'on de lenguaje natural. Las tareas 8 y 9 se abordar\'an el pr\'oximo a\~no. 


