\section{Avances logrados a\~no 2011-2012 }
\label{progreso}

Estudi\'o un algoritmo existente de generaci\'on de expresiones referenciales que usa l\'ogica modal para dar la 
expresi\'on referencial y lo modific\'o para dar expresiones referenciales m\'as naturales a nivel sem\'antico, para hacer 
esto incluy\'o las probabilidades de uso de las palabras al algoritmo y agreg\'o 
sobreespecificaci\'on a las expresiones referenciales. Estudi\'o un corpus de expresiones 
referenciales hecho por humanos y con aprendizaje autom\'atico infiri\'o las probabilidades de uso 
que son par\'ametros del algoritmo. Realiz\'o la evaluaci\'on del sistema compar\'andolo con el corpus 
y varios baselines consigui\'endose una muy buena precisi\'on. 
Conjuntamente con su directora Dra. Luciana Benotti y el Dr. Carlos Areces escribieron un art\'iculo cient\'ifico 
con los resultados de esta investigaci\'on el cual fue aceptado en la conferencia ``24th International Conference on Computational Linguistics IIT Bombay, Mumbai, India'' (COLING 2012).\\
En agosto de 2012 aprob\'o la materia de postgrado ``Generaci\'on de lenguaje natural'', la cual le d\'a 3 cr\'editos para el doctorado.
Teniendo un total de 9 cr\'editos.\\
Asisti\'o a la Escuela de Ciencias Inform\'aticas 2012 (ECI 2012) en donde realiz\'o 2 cursos: ``Java para software de tiempo real'' 
(rindi\'o el ex\'amen pero todav\'{\i}a no tiene el resultado del ex\'amen)
y ``T\'ecnicas avanzadas para procesamiento del habla'' (aprobado). Estos 2 cursos le dar\'{\i}an 1 cr\'edito m\'as.\\
Se cumpli\'o con casi todas las tareas previstas para el a\~no excepto por por integraci\'on del algoritmo a una aplicaci\'on con dominio din\'amico y la generaci\'on de expresiones referenciales parciales, las cuales fueron reemplazadas por aprendizaje de orden de preferencia de propiedades desde corpora, adaptando algoritmos de aprendizaje autom\'atico para esta tarea. Esta decisi\'on se tom\'o con el objetivo de cerrar un a\~no de trabajo en resultados publicables y aplicables seg\'un los est\'andares del \'area del doctorado de la alumna en ling\''{\i}stica computacional. Las tareas que no se hicieron se incluyen en el nuevo plan de trabajo.




