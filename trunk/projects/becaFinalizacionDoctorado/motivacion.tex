\section{Motivaci\'on del plan de trabajo}
\label{motivacion}

La generaci�n autom\'atica de texto (citar libro) es un desaf\'io en el \'area de la inteligencia artificial, en particular la generaci\'on de expresiones referenciales es un \'area en la que hay mucho trabajo hecho, pero a\'un queda mucho por hacer, para acercar la salida producida por la m\'aquina a las expresiones generadas por humanos.


A modo ilustrativo, supongamos que tenemos una escena y queremos conseguir la expresi\'on referencial de cada objeto de la imagen, tambi\'en tenemos acceso a una base de datos que contiene todas las propiedades relevantes de los objetos en la escena, necesitamos encontrar una combinaci�n de propiedades que se aplica a un objeto particular, y no a los otros, para poder generar para ese objeto una expresi\'on referencial que lo identifique un\'ivocamente. Sin embargo no s\'olo se trata de conseguir una expresi\'on ya que podr\'iamos conseguir una expresi\'on que lo identifique, pero que no sea la m\'as natural en el sentido de como lo dir\'ia un humano, por lo tanto, habr\'ia que generar multiples opciones y luego decidir cu�l de ellas es la \'optima en el contexto dado. 


Otras subtareas de la GRE son: la generaci\'on de corpus de expresiones referenciales, la identificaci\'on de propiedades relevantes, el orden de las mismas...


Una de las formas de generar expresiones referenciales es dado un modelo usar l\'ogica para conseguir una formula que satisfaga a cada uno de los objetos del modelo (citar paper Carlos)
Actualmente hay un algoritmo que genera una expresi\'on referencial para cada elemento del modelo, este algoritmo ser\'a modificado para poder generar muchas expresiones referenciales y mediante otras t\'ecnicas de inteligencia artificial se decidir\'a cu\'al o cuales son las m\'as parecidas a la que un humano generar\'ia. Notar que esto es interesante ya que nuestro generador podr\'ia dar distinta expresiones en distintos momentos o para distintos objetivos.
Agregar algoritmo aqui

Entre las aplicaciones a las cual ser\'ia \'util agregar generaci\'on de expresiones referenciales tenemos agentes que dan instrucciones generadas autom\'aticamente en entornos virtuales (LU:citar-estudio previo, agregar mas... GIVE) en los cuales se le da instrucciones a un usuario para que cumpla una tarea, y estas instrucciones se generan autom\'aticamente, un tutor de aprendizaje de segundas lenguas(YO: voy a buscar algumas citas en la web...) que genere autom\'aticamente el texto a aprender, usando el contexto de las cosas que el alumno ya ha aprendido, un ejemplo simple ser\'ia ense�ar las preposiciones mostrando una imagen y dando una expresi\'on referencial al objeto, o ense�ar palabras nuevas diciendole al alumno cual es la palabra nueva a aprender y d\'andole una expresi\'on referencial para identificarla en un contexto dado.
