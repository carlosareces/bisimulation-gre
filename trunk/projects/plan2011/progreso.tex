\section{Avances logrados hasta el momento}
\label{progreso}

He aprobado algunas materias y regularizado otras materias que me han dado herramientas te\'oricas y pr\'acticas de gran utilidad para la realizaci\'on de mi doctorado. As\'i mismo, he asistido a seminarios en el \'area de mi doctorado.

\begin{itemize}

\item Materias realizadas y pr\'acticos en las materias:\\

\begin{itemize}
\item Aprob\'e la materia {\bf Procesamiento de lenguaje natural} en calidad de materia optativa en la licenciatura (2do cuatrimestre 2007). En el proyecto de esta materia realic\'e miner\'ia de opiniones de las elecciones presidenciales con corpus extra\'ido de blogs de la web. El libro principal de la materia fue \cite{Manning2000}
Los conocimientos aprendidos pueden ser usados para realizar b\'usquedas en corpus de expresiones referenciales.
\item Aprob\'e la materia {\bf Traducci\'on autom\'atica y Evaluaci\'on}. En el proyecto de la materia mejor\'e un sistema de traducci\'on autom\'atica en Cunei (traducci\'on autom\'atica h\'ibrida - por ejemplos y estad\'istica).
En la materia aprend\'i t\'ecnicas de evaluaci\'on que pueden ser usadas en otras \'areas de la inteligencia artificial. 
\item Regularic\'e {\bf Redes Neuronales} (2do cuatrimestre 2010). El libro de la materia fue \cite{Hertz1991}\\
Una de las aplicaciones de las redes neuronales es aprendizaje autom\'atico, por lo que ser\'ia \'util en mi doctorado.
\item Aprob\'e la materia {\bf Inteligencia Artificial} (marzo 2011)\\
En la materia hice un sistema de resoluci\'on de correferencias para los pronombres personales tercera persona el cual apliqu\'e a tres idiomas, el espa\~{n}ol, el catal\'an y el italiano.\\
La correferencia tambi\'en es una forma de expresi\'on referencial. En esta materia tambi\'en aprend\'i sobre entornos virtuales lo que va a ser aplicado en mi doctorado.
\item Curs\'e la materia {\bf Generaci\'on de Lenguaje Natural} (primer cuatrimestre de 2011)\\
En esta materia estoy realizando un surface realizer para el espa\~{n}ol, un surface realizer es un m\'odulo de generaci\'on de lenguaje natural que toma informaci\'on sem\'antica y retorna texto en lenguaje natural que corresponde a la sem\'antica dada. El libro principal de la materia fue \cite{Reiter2000}\\
\end{itemize}
\item Asistencia a seminarios:\\
\begin{itemize}
\item Present\'e el estudio realizado sobre correferencias en el Grupo de Procesamiento de Lenguaje Natural - Famaf\\
\item Asist\'{\i} a la Escuela de Ling\"u\'{\i}stica Computacional 2011 y recib\'{\i} el diploma del curso "Transducers"\\
\item Asist\'{\i} a las jornadas de j\'ovenes investigadores desarrolladas el 18 y 19 de agosto de 2011
\end{itemize}
\end{itemize}
