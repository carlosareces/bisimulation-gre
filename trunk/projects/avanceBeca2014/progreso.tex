\section{Resumen de la investigaci\'on realizada}
\label{progreso}

A continuaci\'on se resumen los principales temas investigados y resultados obtenidos. 

\begin{itemize}
\item Aplicamos el algoritmo realizado en el a\~no anterior a otro corpus de expresiones referenciales humanas~\cite{gatt-balz-kow:2008:ENLG}, de distinto tipo del corpus usado el a\~no anterior consiguiendo buenos resultados que fueron comparados con los mejores resultados de una competencia de generaci\'on de expresiones referenciales~\cite{KrahmerGRAPH}. Tambi\'en realizamos una evaluaci\'on humana con jueces y mostramos que en el 92\% de los casos la expresi\'on referencial generada por nuestro algoritmo es considerada mejor o igual a la hecha por humanos.

\item Conjuntamente con el Dr. Invandre Paraboni, University of S\~ao Paulo - USP 
School of Arts, Sciences and Humanities - EACH, la doctoranda ha recolectado un corpus de descripciones de mapas. El corpus recolectado es de expresiones referenciales dichas por personas a las cuales se les mostraron 20 mapas con uno o dos lugares se\~nalados en cada mapa. La recolecci\'on se realiz\'o en dos idiomas en espa\~nol y en portugu\'es, siendo diferentes los mapas usados teniendo en cuenta el idioma, es decir se eligieron lugares en el mismo idioma de la persona. Se recolectaron aproximadamente 4000 expresiones referenciales. 

El proceso de anotar un corpus es un proceso por el cual la informaci\'on desestructurada en languaje natural (cada persona puede nombrar cosas de diferente manera), se unifica a fin de agrupar las diferentes formas de decir las cosas, el objetivo es obtener un archivo XML con la informaci\'on formalizada.
Para esto hemos creado un sistema de anotaci\'on, en el cual ponemos los mapas, los identificadores de las referencias en dichos mapas (calles, restaurantes, etc.). Adem\'as incluye digitalizar las frases de cada persona al sistema, estandarizar nombres (ejemplo: bar, restaurant), identificar relaciones a usar en cada mapa (ejemplo: en la esquina-de, a la derecha-de). Finalmente, para cada expresi\'on referencial, evaluamos si es una expresi\'on v\'alida, en caso afirmativo se anota lo que la persona dijo, en caso negativo desecharla como inv\'alida. Esto se deber\'a realizar para cada una de las 100 expresiones referenciales recolectadas por cada uno de los 20 mapas en cada idioma.
 
\item La doctoranda present\'o un art\'iculo cient\'ifico~\cite{benotti-altamirano-context2013} en la conferencia \emph{International and Interdisciplinary Conference on Modelling and Using Context (Context 2013)}. Fue publicado en modo \emph{lecture notes in Artificial Intelligence}, editorial Springer, dicha conferencia se realiz\'o desde el 28 de octubre de 2013 al 1 de noviembre de 2013, Annecy, Francia. En la presentaci\'on que realiz\'o en CONTEXT 2013 conoci\'o varios investigadores del \'area los cuales se mostraron interesados en el trabajo: Dra. Roberts Craige de Ohio	State	University, Dr. Stone Matthew, Rutgers, USA.
\end{itemize}


\section{Publicaciones}
 
La doctoranda cuenta con cuatro publicaciones internacionales, una de ellas en una revista indexada, y una publicaci\'on nacional. Actualmente est\'a trabajando en la redacci\'on de otra publicaci\'on en revista indexada que incluir\'a los resultados finales de su trabajo de tesis doctoral.

\begin{itemize}
\item Agregar ac\'a el art\'iculo con Laura para NAACL
\item Present\'o un art\'iculo a \emph{The 24th International Conference on Computational Linguistics (COLING 2012)}. \textbf{Software:} El c\'odigo fuente y la documentaci\'on del software realizado est\'an disponibles bajo licencia GNU Lesser GPL en \url{http://code.google.com/p/bisimulation-gre}.
\item Present\'o un art\'iculo cient\'ifico~\cite{benotti-altamirano-context2013} en la conferencia \emph{International and Interdisciplinary Conference on Modelling and Using Context (Context 2013)}. Fue publicado en modo \emph{Lecture Notes in Artificial Intelligence}, editorial Springer, dicha conferencia se realiz\'o desde el 28 de octubre de 2013 al 1 de noviembre de 2013, Annecy, Francia.
\item Present\'o un poster de un art\'iculo cient\'ifico ``Frequencies of occurrence in LGLex lexicon with IRASubcat''~\cite{tolone-altamirano} en JAIIO 2013 en el \emph{Simposio Argentino de Inteligencia Artificial (ASAI 2013)}, FAMAF, C\'ordoba Argentina que se realiz\'o desde el 16 al 20 de septiembre de 2013.
\item Conjuntamente con Elsa Tolone hemos escrito un art\'iculo cient\'ifico ``Frequencies of occurrence of entries and subcategorization frames in LGLex lexicon with IRASubcat''~\cite{tolone-altamirano-2} que fue aceptado en \emph{Actes du 32\`eme Colloque international sur le Lexique et la Grammaire (LGC'13)}, Faro, Portugal, que se realiz\'o desde el 10 al 14 de septiembre de 2013.
\item Escribi\'o un art\'iculo cient\'ifico ``Frequencies of occurrence of entries and subcategorization frames in LGLex lexicon with IRASubcat''~\cite{tolonealtamirano2} que fue publicado en \emph{Actes du 32\`eme Colloque international sur le Lexique et la Grammaire (LGC'13)}, Faro, Portugal, que se realiz\'o desde el 10 al 14 de septiembre de 2013.
\item Con el Dr. Carlos Areces continuamos con la escritura de un art\'iculo~\cite{benotti-altamirano-jair} de journal que ser\'a submitido a \emph{Journal of Artificial Intelligence Research (JAIR)}.
%\item Continuamos con la anotaci\'on del corpus recolectado con el Dr. Ivandre Paraboni.
\end{itemize}


\section{Cursos y seminarios requeridos por el doctorado}

La doctoranda ha completado los 12 cr\'editos requeridos por el Doctorado en Ciencias de la Computaci\'on de FAMAF-UNC. Adem\'as ha dictado los 2 seminarios y ha aprobado el ex\'amen de ingl\'es requerido. A continuaci\'on se detallan los cursos aprobados que suman 12 cr\'editos.  

\begin{itemize}
\item {\bf ``Traducci\'on autom\'atica y Evaluaci\'on''}. En el proyecto de la materia la doctoranda mejor\'o un sistema de traducci\'on autom\'atica  usando traducci\'on autom\'atica h\'ibrida - por ejemplos y estad\'istica. Curso de posgrado de FAMAF-UNC de duraci\'on de 120 horas dictado por la Dra.~Paula Estrella. 
\item {\bf ``Inteligencia Artificial''} (2011). En la materia la doctoranda hizo un sistema de resoluci\'on de correferencias para los pronombres personales tercera persona el cual aplic\'o a tres idiomas. Las correferencias son expresiones referenciales que refieren a objetos previamente mencionados en un texto. Curso de posgrado de FAMAF-UNC de duraci\'on de 120 horas dictado por la Dra.~Laura Alonso Alemany. 
\item {\bf ``Generaci\'on de Lenguaje Natural''} (2011). En esta materia la doctoranda realiz\'o un {\it surface realizer} para el espa\~{n}ol. Un {\it surface realizer} es un m\'odulo de generaci\'on de lenguaje natural que toma informaci\'on sem\'antica y retorna texto en lenguaje natural que corresponde a la sem\'antica dada. Curso de posgrado de FAMAF-UNC de duraci\'on de 120 horas dictado por el Dr.~Pablo Duboue. 
\item \textbf{``Java para software de tiempo real''}, curso de 15 horas de duraci\'on en la \emph{Escuela de Ciencias Inform\'aticas 2012 (ECI 2012)}.
\item \textbf{``T\'ecnicas avanzadas para procesamiento del habla''}, curso de 15 horas de duraci\'on en la \emph{Escuela de Ciencias Inform\'aticas 2012 (ECI 2012)}.
\item \textbf{``Graph-based Representation and Reasoning in Artificial Intelligence''} dictado por Madalina Croitoru (University of Montpellier II, Francia), curso de 15 horas de duraci\'on en la \emph{Escuela de Ciencias Inform\'aticas 2013 (ECI 2013)}.
\item \textbf{``Runtime Verification: From Theory to Practice and Back''} dictado por Gordon Pace (Department  of  Computer  Science,  Faculty  of  ICT,  University  of  Malta), curso de 15 horas de duraci\'on en la \emph{Escuela de Ciencias Inform\'aticas 2013 (ECI 2013)}.
\end{itemize}

\textbf{Licencia:}
La doctoranda estuvo de licencia por maternidad desde el 15 de febrero al 15 de agosto de 2014. 




