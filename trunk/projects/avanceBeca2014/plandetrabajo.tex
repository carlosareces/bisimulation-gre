\section{Plan de Actividades 2012-2013}
\label{plandetrabajo}

En esta secci\'on detallamos el nivel de cumplimiento de los objetivos del plan de trabajo del a\~no anterior. 

{\footnotesize
\begin{center}
\begin{tabular}{|p{7cm}|p{7cm}|}
\hline
 \rowcolor[rgb]{0.8,0.8,0.8}\hspace{3.5cm}Tarea & Nivel de realizaci\'on\\
\hline 1. Lectura de bibliograf\'ia  del corpus TUNA-AR & REALIZADA \\
\hline 2. Adaptaciones necesarias, aprendizaje autom\'atico de probabilidades & REALIZADA  \\
\hline 3. Ejecuci\'on de nuestro algoritmo y comparaci\'on de resultados & REALIZADA   \\
\hline 4. Obtenci\'on de los datos de corpus necesarios para armar un modelo el cual se intentar\'a realizar autom\'aticamente con datos de la web (Folksonomies). &  NO REALIZADA  \\
\hline 5. Obtenci\'on de las probabilidades de uso usando aprendizaje autom\'atico a partir de datos de la web.& NO REALIZADA \\
\hline 6. Ejecuci\'on del algoritmo, an\'alisis de resultados y comparaci\'on con los resultados de~\cite{pachecoDuboue}. & NO REALIZADA \\
\hline 7. Adaptaci\'on a dominio din\'amico. & NO REALIZADA  \\

\hline 8. Integraci\'on del algoritmo en una aplicaci\'on con dominio din\'amico (por ejemplo GIVE). & NO REALIZADA \\
\hline 9. Documentaci�n y publicaci\'on de resultados. & REALIZADA \\
\hline

\end{tabular}\end{center}
}

En lugar de realizar las tareas 4 a 8 este a\~no, se decidi\'o trabajar en la recolecci\'on de un corpus de expresiones referenciales. Esta decisi\'on se tom\'o con el objetivo de brindar a la comunidad un corpus mucho m\'as natural de los que existen actualmente y poder mostrar que nuestra generaci\'on de expresiones referenciales se puede usar en situaciones de la vida real.


