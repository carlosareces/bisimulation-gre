\thispagestyle{empty}


%\newpage
\begin{center}

{ \vspace*{1cm} }
\huge{\textbf{\textsc{\textmd{Resumen}}}}\\[1cm]
%\Large{\textsc{\textmd{Facultad de Matem\'atica, Astronom\'ia y F\'isica}}}\\[1cm]

\end{center}

\normalsize{


En esta tesis investigamos la generaci\'on autom\'atica de rankings de
expresiones referenciales en contextos con incertidumbre. Las
posibles aplicaciones de la generaci\'on de expresiones referenciales
que deben referirse al mundo real (e.g. software para robots, sistemas
gps, etc.) sufren de incertidumbre por datos ruidosos de sensores y
modelos incompletos de la realidad. Extendemos t\'ecnicas y algoritmos
de teor\'ia de modelos y simulaciones integrando una distribuci\'on finita
de probabilidades que representa esta incertidumbre. El objetivo es
generar un ranking de las expresiones referenciales ordenado por la
probabilidad de ser correctamente interpretada en el contexto. \\
En primer lugar, se desarrollaron t\'ecnicas y algoritmos de generaci\'on de
expresiones referenciales que extienden algoritmos cl\'asicos de
minimizaci\'on de aut\'omatas. Los algoritmos de minimizaci\'on se aplicaron a la caracterizaci\'on de modelos de
primer orden. Dichos algoritmos fueron extendidos usando
probabilidades aprendidas de corpora con t\'ecnicas de aprendizaje
autom\'atico. Los algoritmos resultantes fueron evaluados usando
t\'ecnicas autom\'aticas y evaluaciones de jueces humanos sobre datos de
benchmarks del \'area. Finalmente se recolect\'o un nuevo corpus de
expresiones referenciales de puntos de inter\'es en mapas de ciudades
con distintos niveles de zoom. Se evalu\'o el desempe\~no del algoritmo en
este corpus relevante a aplicaciones sobre mapas del mundo real.
}

\begin{itemize}
	\item \textbf{\textsc{Clasificaci\'on de Biblioteca: CCS, Computing methodologies, Artificial intelligence, Natural language processing, Natural language generation}}
	\item \textbf{\emph{\textsc{Palabras Clave:} \\ expresiones referenciales, aprendizaje autom\'atico, bisimulaci\'on, evaluaci\'on, corpus, teor\'ia de modelos.}}
\end{itemize}

\newpage

\begin{center}

{ \vspace*{1cm} }
\huge{\textbf{\textsc{\textmd{Abstract}}}}\\[1cm]
%\Large{\textsc{\textmd{Facultad de Matem\'atica, Astronom\'ia y F\'isica}}}\\[1cm]

\end{center}

\title{Generation of referring expressions under uncertainty using model theory}

\normalsize{In this thesis we investigate the automatic generation of referring expression rankings in contexts under uncertainty. The potential applications of automatic generation of referring expressions that need to refer to the real world (e.g. robot software, gps systems, etc) suffer from uncertainty due to noisy sensor data and incomplete models. We extend tecniques and algorithms from model theory with finite probability distributions that represent these uncertainties. Our goal is to generate rankings of referring expressions ordered by the probability of being interpreted successfully.\\
First, we developed techniques and algorithms for generating referring expressions that extend classical algorithms for automata minimization applied to first order model characterization. Such algorithms were extended using probabilities learned from corpora applying machine learning techniques. The resulting algorithms were evaluated using automatic metrics  and human judgements with respect to benchmarks from the area. Finally, we collected a new corpus of referring expressions of interest points in city maps with different zoom levels. The algorithms were evaluated on this corpus which is relevant to applications that use maps of the real world.
}

\begin{itemize}
	%\item \textbf{\textsc{Clasificaci\'on de Biblioteca:}} \\ \\  \\ \\

	\item \textbf{\emph{\textsc{Keywords:} \\ referring expressions, machine learning, bisimulation, evaluation, corpora, model theory.}}
\end{itemize}

