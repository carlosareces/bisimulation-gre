\chapter{Introducci\'on a la L-simulaci\'on}

Exactamente debido a su generalidad, los grafos no definen por s\'i mismos, una unica noci\'on de igualdad. Cu\'ando decimos que dos nodos en el grafo pueden o no pueden ser
referenciados de forma \'unica en t\'erminos de sus propiedades? Esta pregunta s\'olo tiene sentido una vez que ya hemos fijado un cierto nivel de expresividad el cual determina cuando dos grafos, o dos elementos en el mismo gr\'afico, son equivalentes.
La expresividad se puede formalizar usando las relaciones estructurales de los grafos (isomorfismos, etc.) o, alternativamente, lenguajes l\'ogicos. 

Ambas formas se presentan en Secci\'on \ref{sec:estructurales}, donde tambi\'en discutimos c\'omo arreglar la noci\'on de impacto de expresividad 
en los casos que el problema GRE tiene soluci\'on; la complejidad computacional de los algoritmos GRE involucrados; 
y la complejidad computacional del problema de la realizaci\'on. Luego investigamos el problema GRE en t\'erminos de diferentes 
nociones de expresividad. 

\section{Estructuras}
\label{sec:estructurales}

%\newcommand{\nBall}{\mathit{ball}\xspace}
%\newcommand{\nCube}{\mathit{cube}\xspace}
%\newcommand{\aSmall}{\mathit{small}\xspace}
\newcommand{\aLarge}{\mathit{large}\xspace}

%\newcommand{\nOntop}{\mathit{ontop}\xspace}
%\newcommand{\nRightof}{\mathit{right of}\xspace}
\newcommand{\nLeftOf}{\mathit{leftof}\xspace}
\newcommand{\aRed}{\mathit{red}\xspace}
\newcommand{\aYellow}{\mathit{yellow}\xspace}

Las estructuras relacionales son muy adecuadas para la representaci\'on de situaciones o escenas. La estructura relacional (tambi\'en llamado ``el modelo relacional'') es un conjunto no vac\'io de objetos -el dominio- junto con una colecci\'on de las relaciones, cada uno con una aridad fija.
Formalmente, asumimos un vocabulario fijo y finito (pero arbitrario) vocabulario de
s\'imbolos de relaci\'on n-aria (las constantes y s\'imbolos de funci\'ones pueden ser representados como relaciones de aridad adecuada). 


Un modelo relacional $\+M$ es una tupla 
$\tup{\Delta,\interp{\cdot}}$ donde $\Delta$ es un conjunto no vac\'io, y
$\interp{\cdot}$ es una funci\'on de interpretaci\'on, esto es,
$\interp{r} \subseteq \Delta^n$ para todo s\'imbolo de relaci\'on $n$-aria tal que
$r$ est\'a en el vocabulario. Decimos que $\+M$ es \emph{finite} cuando
$\Delta$ es finito.  El \emph{tama\~no} de un modelo $\+M$ es la suma
$\#\Delta + \#\interp{\cdot}$, donde $\#\Delta$ es la cardinalidad
de $\Delta$ y $\#\interp{\cdot}$ es la suma de todas las aridades de las
relaciones en $\interp{\cdot}$.

La Figura~\ref{grafo-GRE3D7-stimulus_b} abajo muestra como nosotros podemos representar una escena
como un modelo relacional. Intuitivamente, $e_1$,$e_3$,$e_5$, son 'ball', $e_2$, $e_6$,y $e_7$, son 'cube', mientras que 
$e_1$, $e_3$ y $e_6$ son 'yellow', $e_2$, $e_4$, $e_5$, $e_7$ son 'red';
%Para las relaciones tenemos $\Ontop(e_3,e_2)$ como ``{\em $e_3$ on top of $e_2$}''.
%$\Rightof(e_4,e_5)$, $\Rightof(e_6,e_7)$, $\Lefttof(e_5,e_4)$,$\Leftof(e_7,e_6)$. $e_4$ y $e_5$ son 'large', los dem\'as 'small'.

 \begin{figure}
 \begin{left}
 \begin{tabular}{rcl}
$\Delta$              & = & $\cset{e_1,e_2,e_3,e_4,e_5,e_6,e_7}$\\
$\interp{\aRed}$      & = & $\cset{e_2,e_4,e_5,e_7}$\\
$\interp{\aYellow}$   & = & $\cset{e_1,e_3,e_6}$\\
$\interp{\nBall}$     & = & $\cset{e_1,e_3,e_6}$\\
$\interp{\nCube}$     & = & $\cset{e_2,e_4,e_7}$\\

$\interp{\aSmall}$    & = & $\cset{e_1,e_2,e_3,e_6,e_7}$\\
$\interp{\aLarge}$    & = & $\cset{e_4,e_5}$\\

$\interp{\nRightof}$   & = & $\cset{(e_4,e_5),(e_6,e_7)}$\\
$\interp{\nLeftof}$    & = & $\cset{(e_5,e_4),(e_7,e_6)}$\\
$\interp{\nOntop}$     & = & $\cset{(e_3,e_2)}$\\

 \end{tabular}
\begin{picture}(120,50)
\put(0,-50){\begin{tikzpicture}
  [
    n/.style={circle,fill,draw,inner sep=1.5pt,node distance=1.5cm},
    aSniffing/.style={->, >=stealth, semithick, shorten <= 3pt, shorten >= 3pt},
  ]
% \node[n,label=above:$a$,label=below:{\relsize{-1}$\begin{array}{c}\nDog\end{array}$}] (a) {};

% \node[n,label=above:$b$,label=below:{\relsize{-1}$\begin{array}{c}\nDog\\ \aSmall \end{array}$}, right of=a] (b) {};

% \node[n,label=above:$c$,label=below:{\relsize{-1}$\begin{array}{c}\nCat\\ \aSmall\end{array}$}, right of=b] (c) {};

% \node[n,label=above left:$d$,label=below:{\relsize{-1}$\begin{array}{c}\nDog\\ \nBreed\\  \aSmall \end{array}$}, below of=a,xshift=25pt,yshift=-10pt] (d) {};

% \node[n,label=above right:$e$,,label=below:{\relsize{-1}$\begin{array}{c}\nCat\end{array}$},right of=d] (e) {};

% \draw [aSniffing,loop left] (a) to node[above,xshift=-5pt]{\relsize{-1}$\aSniffing$} (a);

% \draw [aSniffing,bend right=40] (b) to node[auto,swap]{\relsize{-1}$\aSniffing$} (a);

% \draw [aSniffing,bend right=40] (c) to node[auto,swap]{\relsize{-1}$\aSniffing$} (b);

% \draw[aSniffing, bend left=40] (d) to node[auto]{\relsize{-1}$\aSniffing$} (e);
% \draw[aSniffing, bend left=40] (e) to node[auto,swap]{\relsize{-1}$\aSniffing$} (d);


%\begin{figure}
%\centering
%\begin{tikzpicture}
%  [
%    n/.style={circle,fill,draw,inner sep=3pt,node distance=2cm},
%    aArrow/.style={->, >=stealth, semithick, shorten <= 1pt, shorten >= 1pt},
%  ]
 \node[n,label=above:$e_1$,label=below:{
    \relsize{-2}$\begin{array}{c}
      \nSmall\\[-3pt] 
      \nYellow \\[-3pt] 
      \nBall\end{array}$}] (a) {};
 \node[n,label=above:$e_2$,label=below:{
    \relsize{-2}$\begin{array}{c}     
      \nSmall\\[-3pt] 
      \nRed\\[-3pt] 
      \nCube\end{array}$}, right of=a] (b) {};
 \node[n,label=below:$e_3$,label=above:{
    \relsize{-2}$\begin{array}{c}
      \nTop\\[-3pt]      
      \nSmall\\[-3pt] 
      \nYellow\\[-3pt] 
      \nBall\end{array}$}, above of=b] (c) {};
 \node[n,label=right:$e_4$,label=left:{
    \relsize{-2}$\begin{array}{c}
      \nLarge\\[-3pt] 
      \nRed\\[-3pt] 
      \nCube\end{array}$}, right of=b] (d) {};
 \node[n,label=left:$e_5$,label=below:{
    \relsize{-2}$\begin{array}{c}
      \nLarge\\[-3pt] 
      \nRed\\[-3pt] 
      \nBall\end{array}$}, right of=d] (e) {};
 \node[n,label=right:$e_6$,label=left:{
    \relsize{-2}$\begin{array}{c}
      \nSmall\\[-3pt] 
      \nYellow\\[-3pt] 
      \nCube\end{array}$}, right of=e] (f) {};
 \node[n,label=left:$e_7$,label=below:{
    \relsize{-2}$\begin{array}{c}
      \nSmall\\[-3pt]
      \nRed\\[-3pt] 
      \nCube\end{array}$},  right of=f] (g) {};
 \draw [aArrow,bend right=40] (b) to node[auto,swap]{\relsize{-3}$\nBelow$} (c);
 \draw [aArrow,bend right=40] (c) to node[auto,swap]{\relsize{-3}$\nOntop$} (b);
 \draw [aArrow,bend right=40] (d) to node[auto,swap]{\relsize{-3}$\nLeftof$} (e);
 \draw [aArrow,bend right=40] (e) to node[auto,swap]{\relsize{-3}$\nRightof$} (d);
 \draw [aArrow,bend right=40] (f) to node[auto,swap]{\relsize{-3}$\nLeftof$} (g);
 \draw [aArrow,bend right=40] (g) to node[auto,swap]{\relsize{-3}$\nRightof$} (f);
 \draw[dotted] (-0.5,-1.1) rectangle (8,2.6);

% \end{tikzpicture}
%\caption{Grafo del contexto \ref{GRE3D7-stimulus}}
%\label{grafo-GRE3D7-stimulus_b}
%\end{figure}
 \end{tikzpicture}}
 \end{picture}

 \end{left}
 \caption{Representaci\'on de escenas de Graph $\+S$.\label{fig:cat-dog-1}}
 \end{figure}


\subsection{Grafos}


\subsection{L\'ogicas modales}

