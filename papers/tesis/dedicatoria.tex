% dedicatoria.tex
% Dedicatoria de la tesis

\vspace*{\fill}

%\begin{flushright}
%    \emph{Gracias Lu, Carlos \ldots}
%\end{flushright}

En este camino aprend\'i muchas cosas, conoc\'i gente maravillosa, conoc\'i lugares hermosos. Y quiz\'as todo eso empez\'o aquel dia en que la Dra. Laura Alonso i Alemany me dijo ``Romina, queres ir a una conferencia en Los Angeles, EEUU?'', y yo respond\'i ``yo?, a Los Angeles?, no no'', y ella dijo ``Bueno, pensalo, despu\'es me decis''. Luego de 2 d\'ias le dije que s\'i. Gracias Laura por abrirme las puertas del mundo, y permitirme pensar que yo tambi\'en puedo hacer esas cosas. Y ah\'i conoc\'i a la Dra. Luciana Benotti, nunca pens\'e que recorrer\'iamos este largo camino juntas. O quiz\'as empez\'o aquel d\'ia en que decid\'i anotarme en el doctorado con la Dra. Paula Estrella. Gracias Paula, lamento que no hayamos podido progresar en la traducci\'on autom\'atica. Pero eso me di\'o un tiempito hasta que lleg\'o Luciana de Europa y me di\'o una oportunidad con las expresiones referenciales!. Gracias Lu, por dedicarme tanto tiempo!, por guiarme y conseguir nuevos desaf\'ios para m\'i. Gracias Dr. Carlos Areces por ayudarnos con el marco te\'orico. Gracias Luli!, por las correcciones!. 
Y en ese camino conoc\'i al Dr. Ivandr\'e Paraboni. Gracias Ivandr\`e disfrut\'e mucho la estadia en Brasil, hubo mucho trabajo, pero tambi\'en hubo playa!. Haciendo la materia Generaci\'on de Lenguaje Natural conoc\'i al Dr. Pablo Duboue. Gracias Pablo me gust\'o mucho hacer el proyecto de la materia, me di\'o la sensaci\'on que puedo hacer algo serio. Y en el 2012 fui a la India ah\'i conoc\'i a Dr. Manish Shrivastava. Gracias Manish por interesarte en mi trabajo, adem\'as gracias por tratarme como a una reina!. 

Gracias Laura, Ivandr\`e y Dr. Raul Fervari por aceptar ser los jurados de mi tesis.
A mis compa\~neros de trabajo en la SPGI que supieron tenerme muchos a\~nos trabajando en horario restringido. A mi jefe el Sr. Juan Montoya que sin conocerme, confi\'o en mi desde el primer momento.

A mi hija que supo esperar el momento adecuado para llegar al mundo, y darme un recreo en la tesis. A Boris, a mi familia. 


\vspace{\fill}
