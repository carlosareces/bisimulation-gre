\thispagestyle{empty}


%\newpage
\begin{center}

{ \vspace*{1cm} }
\huge{\textbf{\textsc{\textmd{Agradecimientos}}}}\\[1cm]
%\Large{\textsc{\textmd{Facultad de Matem\'atica, Astronom\'ia y F\'isica}}}\\[1cm]

\end{center}

\normalsize{

En este camino aprend\'i muchas cosas, conoc\'i gente maravillosa, conoc\'i lugares hermosos. Y quiz\'as todo eso empez\'o aquel d\'ia en que la Dra. Laura Alonso i Alemany me dijo ``Romina, quer\'es ir a una conferencia en Los Angeles, EEUU?'', y yo respond\'i ``yo?, a Los Angeles?, no no'', y ella dijo ``Bueno, pensalo, despu\'es me dec\'is''. Luego de 2 d\'ias le dije que s\'i\'i\'i. Gracias Laura por abrirme las puertas del mundo, y permitirme pensar que yo tambi\'en puedo hacer esas cosas. Y ah\'i conoc\'i a la Dra. Luciana Benotti, nunca pens\'e que recorrer\'iamos este largo camino juntas. Tambi\'en conoc\'i a Jerry Hobbs y con \'el a mis primeras expresiones referenciales. O quiz\'as empez\'o aquel d\'ia en que decid\'i anotarme en el doctorado con la Dra. Paula Estrella. Gracias Paula, lamento que no hayamos podido progresar en la traducci\'on autom\'atica. Fui a una conferencia en Barcelona y ah\'i conoc\'i a la Dra. Irene Castell\'on, y a su alumna de doctorado, divinas!, y hasta me saqu\'e una foto con Alon Lavie un capo de la traducci\'on autom\'atica. Despu\'es lleg\'o Luciana de Europa y me dio una oportunidad con las expresiones referenciales!. Muchas gracias Lu, por dedicarme tanto tiempo!, por guiarme y conseguir nuevos desaf\'ios para m\'i cada d\'ia. Muchas gracias Dr. Carlos Areces por ayudarnos con el marco te\'orico y hasta con el c\'odigo. Gracias Luli!, por las correcciones! (solo para entendidos). Y las aventuras continuaron, con Luciana presentamos un paper y fui a la India en el 2012 ah\'i conoc\'i a Dr. Manish Shrivastava. Gracias Manish por interesarte en mi trabajo, gracias a todos por tratarme como a una reina!, en la India hasta habl\'e en ruso, s\'i, ah\'i conoc\'i a Sonun Karabeva y a Devendra Sing y con ellos hablabamos en ruso y fuimos al Taj Mahal y a Jaipur. En el 2013 fui a Francia, ah\'i conoc\'i a la Dra. Craige Roberts, yo estaba embarazada en esa \'epoca, compartimos largas charlas en la cena de gala y camino de vuelta al hotel, ella me dijo refiri\'endose a mi hija, ``ojal\'a sea bella!, muy bella, porque sana seguro que va a ser'' y as\'i es. Tambi\'en conoc\'i a Mattew Stone, que personaje!. Conoc\'i al Dr. Ivandr\`e Paraboni y su alumno Thiago Ferreyra Castro, muy trabajadores!. Gracias Ivandr\`e disfrut\'e mucho la estad\'ia en Brasil, hubo mucho trabajo, pero tambi\'en hubo playa!.

Gracias Laura, Ivandr\`e y Dr. Raul Fervari por aceptar ser los jurados de mi tesis. Para mi es un honor que sean ustedes.

A mis compa\~neros de trabajo en la SPGI que supieron tenerme muchos a\~nos trabajando en horario reducido. A mi jefe el Sr. Juan Montoya que sin conocerme, confi\'o en mi desde el primer momento, por alentarme y decirme te quiero ver Doctora!. Al Dr. Sergio Obeide y a la SeCyT quienes consiguieron una excepci\'on extraordinaria que me permiti\'o hacer el doctorado siendo no-docente en la Secretar\'ia de Planificaci\'on y Gesti\'on Institucional de la UNC.

A mi hija que supo esperar el momento adecuado para llegar al mundo, y darme un recreo en la tesis. A Boris, a mi familia.

Y a todos los que olvid\'e mencionar!. Gracias a todos!.
}

\vspace{\fill}
