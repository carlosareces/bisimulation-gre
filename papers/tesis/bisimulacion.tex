
\chapter{Bisimulaci\'on}
\label{bisimulacion}

%Este algoritmo fue propuesto por %\cite{}

La idea es transformar el problema de GER al problema de computar una f\'ormula de l\'ogicas para la descripci\'on (DL) cuyos elementos que satisfagan la f\'ormula sea el target.% (o los elementos targets, en el caso de querer dar una ER para plurales aprovechando que una f\'ormula describe un conjunto que puede contener m\'as de un elemento).\\

A continuaci\'on daremos una introducci\'on a las l\'ogicas para la descripci\'on \alc y \el. Explicaciones m\'as detalladas sobre \alc y \el se ver\'an en el Cap\'itulo \ref{sec:intro_logica}.

\emph{F\'ormulas} (o \emph{conceptos}) $\varphi$ de $\alc$ son generadas por la siguiente gram\'atica:
$$
\varphi,\varphi' ::= \top \mid p \mid \neg \varphi \mid \varphi \sqcap \varphi'
\mid \exists R. \varphi
$$
donde $p$ es el conjunto de los s\'imbolos proposicionales \prop y $R$ es el de los s\'imbolos relacionales \rel. $\el$ es la parte sin negaci\'on de $\alc$.\\

Las f\'ormulas de ambos $\alc$ y $\el$ son interpretadas en modelos relacionales de primer orden $\gM = (\Delta,\interp{\cdot})$ donde
$\Delta$ es un conjunto no vac\'io y $\interp{\cdot}$ es una funci\'on de interpretaci\'on tal que:
$$
\begin{array}{ccl}
\interp{p} & \subseteq & \Delta  \mbox{ para $p \in \prop$}\\
\interp{R} & \subseteq & \Delta \times \Delta  \mbox{ para $R \in \rel$}\\
\interp{\neg \varphi} & = & \Delta - \interp{\varphi}\\
\interp{\varphi \sqcap \varphi'} & = & \interp{\varphi} \cap \interp{\varphi'}\\
\interp{\exists R.\varphi} & = & \{i \mid \mbox{para alg\'un } i', (i,i') \in \interp{R}\\
& & \mbox{ e } i' \in \interp{\varphi} \}.\\
\end{array}
$$

\begin{figure}[ht]
\begin{center}
\includegraphics[width=8.5cm]{figures/pic-dale-haddock.pdf}\\[0pt]
\caption{}
\label{fig:dale-haddock}
\end{center}
\end{figure}


Cada f\'ormula de una descripci\'on l\'ogica denota un conjunto de objetos del dominio; por lo tanto podemos usar tales f\'ormulas para describir conjuntos. Por ejemplo en el modelo de la Figura.~\ref{fig:dale-haddock}b, la f\'ormula
$\mathsf{flower}$ denota el conjunto $\{f_1,f_2\}$; La f\'ormula
$\mathsf{flower} \sqcap \exists \mathsf{in}.\mathsf{hat}$ denota
$\{f_2\}$; y la f\'ormula $\mathsf{flower} \sqcap \neg
\exists \mathsf{in}.\mathsf{hat}$ denota $\{f_1\}$.\\

\textcolor{blue}{no se si dejar este ejemplo, o cambiarlo por uno es espa\~nol}

Hay muchas otras l\'ogicas de descripci\'on (DL) en la literatura por ejemplo 

$\mathcal{CL}$ (\el\ sin el cuantificador existencial, es decir solo conjunciones at\'omicas); $\mathcal{PL}$ (\alc\ l\'ogica propocisional); y
$\mathcal{ELU}_{(\neg)}$ (\el\ m\'as disjunci\'on y negaci\'on at\'omica).\\

Usaremos una noci\'on de preservaci\'on de f\'ormulas que llamaremos
\emph{similaridad}. Para cualquier DL $\gL$, diremos que un individual $i$ es \emph{\gL-similar} a $i'$ en un modelo dado $\gM$
si para cualquier f\'ormula $\varphi \in \gL$ tal que $i \in
\interp{\varphi}$, tambi\'en tenemos que $i' \in \interp{\varphi}$.\\
Equivalentemente, no hay $\gL$-f\'ormula que se mantenga para $i$ pero no para
$i'$.  Diremos que el \emph{\gL-conjunto de similaridad} de alg\'un individual
$i$ es el conjunto de todos los individuales a los cuales $i$ es \gL-similar.\\

Notar que la similaridad no es necesariamente una relaci\'on sim\'etrica: Por ejemplo:$f_1$ es \el-similar a $f_2$ en
Figura~\ref{fig:dale-haddock}b, pero $f_2$ no es \el-similar a $f_1$
(satisface la f\'ormula $\exists \mathsf{in}.\mathsf{hat}$ y $f_1$
no la satisface).  De todas maneras, \alc-similaridad es una relaci\'on sim\'etrica porque
el languaje contiene negaci\'on; y en consecuencia, $f_1$ no es \alc-similar
a $f_2$ porque este tampoco satisface $\neg \exists
\mathsf{in}.\mathsf{hat}$.  Porque \alc\ es m\'as expresivo que \el,
esto es, para alg\'un individual $a$ es posible ser \el-similar pero
no \alc-similar a alg\'un individual $b$, pero no viceversa.\\



%\textcolor{blue}{SACAR ESTO... poner quizas en la introduccion o en primer capitulo. Las ER que involucran relaciones han recibido m\'as atenci\'on recientemente;
%especialmente en el contexto de las expresiones referenciales espaciales en generaci\'on (por ejemplo,~\cite{kelleher06:increm}), donde es particularmente natural utilizar expresiones que implican relaciones espaciales, tales como ``la esfera en la parte superior del cubo''. Sin embargo, el algoritmo cl\'asico por~\cite{dale91:gener} ha demostrado ser incapaz de generar ER satisfactorias en la pr\'actica (v\'ease el an\'alisis sobre el~\emph{cabinet corpus} en~\cite{viethen06:_algor_for_gener_refer_expres}). Adem\'as, el Dale y Haddock algoritmo y muchos de sus sucesores (tales como~\cite{kelleher06:increm}) son vulnerables a el problema de la \emph{regresi\'on infinita}, donde el algoritmo entra en un bucle infinito, saltando hacia atr\'as y hacia adelante entre las descripciones para dos individuos emparentados, como en `` el libro sobre la mesa que soporta una libro sobre la mesa \ldots ''}

%REs involving relations have received increasing attention recently;
%especially in the context of spatial referring expressions in situated
%generation (e.g., \cite{kelleher06:increm}),
%where it is particularly natural to use expressions involving spatial
%relations such as ``the ball on top of the cube.''  However, the
%classical algorithm
%by~\cite{dale91:gener} was shown to be
%unable to generate satisfying REs in practice (see the analysis over
%the \emph{cabinet corpus}
%in~\cite{viethen06:_algor_for_gener_refer_expres}).  Furthermore, the
%Dale and Haddock algorithm and many of its successors (such
%as~\cite{kelleher06:increm}) are vulnerable to
%the problem of \emph{infinite regress}, where the algorithm enters an
%infinite loop, jumping back and forth between descriptions for two
%related individuals, as in ``the book on the table which supports a
%book on the table \ldots''

%\cite{arec2:2008:Areces,arec:usin11} have proposed low complexity
%algorithms for the generation of relational REs
%%(including references to sets) 
%that eliminate the risk of infinite regression.  These algorithms are
%based on variations of the partition refinement algorithms
%of~\cite{paig:thre87}.  The information provided by a given scene
%is interpreted as a relational model whose objects are classified into
%sets that fit the same description.  This classification is
%successively \emph{refined} till the target is the only element
%fitting the description of its class.  The existence of an ER depends
%on the information available in the input scene, and on the expressive
%power of the formal language used to describe elements of the
%different classes in the refinement.


\cite{arec2:2008:Areces,arec:usin11} han propuesto algoritmos de baja complejidad
 para la generaci\'on de ER relacionales que eliminan el riesgo de regresi\'on infinita. Estos algoritmos son
basados en variaciones de los algoritmos de refinamiento de particiones
de~\cite{paig:thre87}. La informaci\'on proporcionada por una escena dada
se interpreta como un modelo relacional cuyos objetos se clasifican en
conjuntos que se adaptan a la misma descripci\'on. Esta clasificaci\'on es
sucesivamente \emph{refinada} hasta que el target es el \'unico elemento
en la clase. La existencia de una ER depende
de la informaci\'on disponible en la escena de entrada, y del poder expresivo
del lenguaje formal utilizado para describir los elementos de las
diferentes clases en el refinamiento.\\

%Refinement
%algorithms %presented in~\cite{arec2:2008:Areces,arec:usin11}
%effectively compute REs for all individuals in the domain, at the same
%time. The algorithms always terminate returning a formula of the
%formal language chosen that uniquely describes the target (if the
%formal language is expressive enough to identify the target in the
%input model).
%\cite{arec2:2008:Areces}
%show that the refinement algorithm using the description language \el  is capable of generating 67\% of 
%the relational REs in the~\cite{viethen06:_algor_for_gener_refer_expres} dataset, when all possible orders of the relations in the domain are considered. This is in sharp contrast with the analysis 
%done in~\cite{viethen06:_algor_for_gener_refer_expres} over the cabinet corpus, of algorithms based in Dale and Reiter's original proposal.    

%Los algoritmos de refinamiento
% presenta en~\cite{arec 2:2008:Areces, arec:usin11}
%calculan efectivamente ER para todos los objetos en el dominio, al mismo
%tiempo. Los algoritmos siempre terminan devolviendo una f\'ormula del
%lenguaje formal elegido que describe un\'{i}vocamente el target (si el
%lenguaje formal es suficientemente expresivo para identificar el target en el
%modelo de entrada).\\

%Refinement algorithms for GER are based on the following basic idea:
%given a scene $S$, the objects appearing in $S$ are successively
%classified according to their properties into finer and finer
%classes. A description (in some formal language $\mathcal{L}$) of each
%class is computed every time a class is refined. The procedure always
%stops when the set of classes stabilizes, i.e., no further refinement
%is possible with the information available in the scene\footnote{Of
%  course, if we are only interested in a referring expression for a
%  given target we can stop the procedure as soon as the target is the
%  only element of some of the classes.}.  If the target element is in
%a singleton class, then the formal description of that class is a
%referring expression; otherwise the target cannot be unequivocally
%described (in 


%It is clear that a scene can be encoded in different ways as a
%relational model (for example in \ref{figure22}, we could argue that
%$e_1$ is also \emph{leftof} $e_2$, not considered because they are no
%touching). The algorithm assumes that these issues have been resolved
%and that the model encodes a suitable representation of the scene we
%want to describe.  Moreover, we will assume that all relations are
%\emph{binary}.  We will not consider relations of arity greater than
%two (relations of higher arity can be encoded as binary relations via
%reification, if necessary).

\begin{figure}[ht]
%\begin{minipage}[b]{0.45\linewidth}
%\centering
\begin{center}
\includegraphics[width=0.5\textwidth]{images/3b.png}
\caption{Contexto de ejemplo}\label{GRE3D7-stimulus-cap2}
%\end{minipage}
\end{center}
\end{figure}

%\textcolor{blue}{hice mas grande la figura porque no se leian, como la pase a espaniol y se me fue al pie, no se porque}
%\hspace*{-0.35cm}
\begin{figure}[ht]
\begin{center}
%\begin{minipage}[b]{0.5\linewidth}
%\centering
\begin{tikzpicture}
  [
    n/.style={circle,fill,draw,inner sep=3pt,node distance=2.6cm},
    aArrow/.style={->, >=stealth, semithick, shorten <= 2pt, shorten >= 2pt},
  ]
 \node[n,label=above:$e_1$,label=below:{
    \relsize{-1}$\begin{array}{c}
      \nLeft\\[-2pt]
      \nSmall\\[-2pt] 
      \nBlue \\[-2pt] 
      \nBall\end{array}$}] (a) {};

 \node[n,label=above:$e_2$,label=below:{
    \relsize{-1}$\begin{array}{c}
      \nLeft\\[-2pt]
      \nBig\\[-2pt] 
      \nBlue\\[-2pt] 
      \nCube\end{array}$}, right of=a] (b) {};

 \node[n,label=below:$e_3$,label=above:{
    \relsize{-1}$\begin{array}{c}
      \nTop\\[-2pt]
      \nLeft\\[-2pt]
      \nSmall\\[-2pt] 
      \nGreen\\[-2pt] 
      \nBall\end{array}$}, above of=b] (c) {};

 \node[n,label=above:$e_4$,label=below:{
    \relsize{-1}$\begin{array}{c}
      \nSmall\\[-2pt] 
      \nGreen\\[-2pt] 
      \nCube\end{array}$}, right of=b] (d) {};

 \node[n,label=above:$e_5$,label=below:{
    \relsize{-1}$\begin{array}{c}
      \nBig\\[-2pt] 
      \nBlue\\[-2pt] 
      \nBall\end{array}$}, right of=d] (e) {};

 \node[n,label=above:$e_6$,label=below:{
    \relsize{-1}$\begin{array}{c}
      \nBig\\[-2pt] 
      \nGreen\\[-2pt] 
      \nCube\end{array}$}, right of=e] (f) {};

 \node[n,label=below:$e_7$,label=above:{
    \relsize{-1}$\begin{array}{c}
      \nTop\\[-2pt]
      \nSmall\\[-2pt] 
      \nBlue\\[-2pt] 
      \nCube\end{array}$}, above of=f] (g) {};

 \draw [aArrow,bend right=90] (b) to node[auto,swap]{\relsize{-1}$\nBelow$} (c);
 \draw [aArrow,bend right=90] (c) to node[auto,swap]{\relsize{-1}$\nOntop$} (b);

 \draw [aArrow,bend right=70] (d) to node[auto,swap]{\relsize{-1}$\nLeftof$} (e);
 \draw [aArrow,bend right=70] (e) to node[auto,swap]{\relsize{-1}$\nRightof$} (d);

 \draw [aArrow,bend right=90] (f) to node[auto,swap]{\relsize{-1}$\nBelow$} (g);
 \draw [aArrow,bend right=90] (g) to node[auto,swap]{\relsize{-1}$\nOntop$} (f);

 \draw[dotted] (-.6,-2.2) rectangle (12.5,5.3);

 \end{tikzpicture}
\caption{Modelo relacional del Contexto \ref{GRE3D7-stimulus-cap2}}
\label{GRE3D7-stimulus-graph}
%\end{minipage}
\end{center}
\end{figure}

Algoritmos de refinamiento para GER se basan en la siguiente idea b\'asica:
dada una escena $S$, los objetos que aparecen en $S$ son sucesivamente
clasificados de acuerdo con sus propiedades en clases m\'as y m\'as finas. 
Una descripci\'on (en alg\'un lenguaje formal de $\mathcal{L}$) de cada
clase se calcula cada vez que una clase es refinada. El procedimiento siempre
se detiene cuando el conjunto de clases se estabiliza, es decir, no se puede hacer m\'as refinamiento
con la informaci\'on disponible en la escena \footnote{Por supuesto, si s\'olo estamos interesados en una expresi\'on referencial de un objeto dado, se puede detener el procedimiento en cuanto el objetivo es el
   \'unico elemento de alguna de las clases.}

Si el elemento target est\'a en
una clase singleton, entonces la descripci\'on formal de esa clase es un
expresi\'on referencial; de lo contrario el target no puede ser un\'{i}vocamente
descripto (en $\mathcal{L}$).

Est\'a claro que una escena puede ser codificada en diferentes formas como un
modelo relacional (por ejemplo, en \ref{GRE3D7-stimulus-cap2}, podr\'{i}amos argumentar que
$e_1$ es tambi\'en \emph{leftof} $e_2$, pero no lo consideramos porque no se estan 
tocando en la imagen). El algoritmo asume que estas cuestiones se han resuelto y que el modelo codifica una representaci\'on adecuada de la escena que
queremos describir. Por otra parte, vamos a suponer que todas las relaciones son
\emph{binarias}. No vamos a considerar las relaciones de aridad mayor que
dos (relaciones de mayor aridad pueden codificarse como relaciones binarias v\'{i}a
reificaci\'on, si es necesario).

%On termination, the algorithm computes what are called the
%$\mathcal{L}$-similarity classes of the input model $\gM$.
%Intuitively, the referring expression ``\textsf{ball}'' and ``\textsf{cube}''  are more specific and then contain more information than $\top$.


Tras la resoluci\'on, el algoritmo calcula lo que se llama la
$\mathcal{L}$ - clases de semejanza del modelo de entrada de $\gM$.\\

%There is many $\mathcal{L}$, we will name $\alc$ and $\el$

%ACA VOY A PONER gramatica para generar... ALC y EL no quedaria bien aca, hay que ver lo agregamos antes o no hace falta
%In what follows, we use formulas of the $\el$ description logic
%language
En lo que sigue, se utilizan f\'ormulas de la descripci\'on de la l\'ogica $\el$
~\cite{baad:desc03} para describir las clases de refinamiendo
\footnote{N\'otese, sin embargo, que el lenguaje formal particular usado es
   independiente del algoritmo principal, y diferentes funciones
  add$_{\mathcal{L}}$($\varphi$,\RE) se pueden utilizar dependiendo
   de la l\'ogica en cuesti\'on.}. como se discuti\'o 
en~\cite{arec2:2008:Areces}, 
este lenguaje es adecuado para describir
RE conjuntivas y relacionales, que son lo que encontramos en los corpus.

  La entrada al algoritmo ser\'a un modelo $\mathcal{M} =
 \tup{\Delta, \interp{\cdot}}$, donde $\Delta$ es el dominio no vac\'io de objetos de la imagen,
 $\interp{\cdot}$ es una funci\'on de interpretaci\'on que asigna a todas las propiedades de la escena su extensi\'on.
 Por ejemplo, la escena mostrada en la Figura~\ref{GRE3D7-stimulus-cap2} podr\'ia ser representada por el modelo
 $\gM=\tup{\Delta,\interp{\cdot}}$ mostrado en la 
 Figura~\ref{GRE3D7-stimulus-graph}; donde+- $\Delta =
 \{e_1,\ldots,e_7\}$, e $\interp{\textsf{red}}$ is $\{e_2, e_4, e_5,
 e_7\}$.

Se llama extensi\'on de una f\'ormula al conjunto de objetos que la hacen v\'alida.

$\top$ es una f\'ormula que representa la descripci\'on m\'as general, cuya
interpretaci\'on incluye todos los elementos del modelo. Se podr\'ia realizar
como la ER con el sustantivo
``\textsf{cosa}''. Decimos que una f\'ormula es
\emph{subsumida} por otras f\'ormulas, cuando su extensi\'on puede ser cubierta por la
union de las extensiones de las otras f\'ormulas. Por ejemplo, en la
Figura~\ref{GRE3D7-stimulus-cap2}, $\top$ es subsumida por ``\textsf{esfera}'' y
``\textsf{cubo}'', porque $\interp{\top}$ = $\interp{\textsf{esfera}}
\cup \interp{\textsf{cube}}$.
%= $\{e_2, e_4, e_6, e_7\}$, it is $\{e_1, e_2, e_3, e_4, e_5, e_6, e_7\}$ = $\{e_1, e_3, e_5\} \cup \{e_2, e_4, e_6, e_7\}$. 
Intuitivamente la f\'ormula ``\textsf{cubo}'' o ``\textsf{esfera}'' tienen m\'as informaci\'on que $\top$, para cada elemento de $\top$, hay una f\'ormula que d\'a m\'as informaci\'on, digamos ``\textsf{cubo}'' es m\'as informativa que ``\textsf{cosa}''.\\

%In the following we will explain an example of execusion of the
%algorithm shown in Figure
%A continuaci\'on vamos a explicar un ejemplo de ejecusi\'on del
%algoritmo mostrado en la Figura~\ref{algoritmoOriginal} considerando la l\'ogica 
%$\el$ como language. Este algoritmo fue presentado en
%~\cite{arec2:2008:Areces}.
%
%\begin{figure}[h!]
%\begin{center}
%\includegraphics[width=\textwidth]{images/algoritmoOriginal.png}
%\end{center}
%\vspace*{-2em}
%\caption{Algoritmo para GER con l\'ogicas de descripci\'on}
%\label{algoritmoOriginal}
%\end{figure}

%\subsection{Ejemplo de ejecuci\'on}
%
%
%
%\textcolor{blue}{no se si poner aca un ejemplo, si poner el texto y las im\'agenes en otro apendice... o ponerlas mas chiquitas en varias columnas, asi queda feo}\\
%Vamos a ejecutar el algoritmo para la Figura~\ref{GRE3D7-stimulus-cap2},
%el algoritmo comienza con una lista fija de propiedades y relaciones, supongamos que
%esas listas son las siguientes:
%
%propiedades ordenadas (prop): \textsf{ball}, \textsf{cube}, \textsf{red}, \textsf{yellow}, \textsf{small}, \textsf{large}.\\
%relaciones ordenadas (rel): \textsf{leftof}, \textsf{rightof}, \textsf{ontopof}, \textsf{bellowof}.
%
%%\begin{figure}
%%\begin{center}	
%%\includegraphics[width=.5\textwidth]{images/22.jpg}
%%\end{center}
%%\vspace*{-1.5em}
%%\caption{Escena 3D de figuras geom\'etricas}\label{figure22}
%%\end{figure}
%
%%\begin{figure}
%%\begin{minipage}[b]{0.6\linewidth}
%%\centering
%%\begin{tikzpicture}
%%  [
%%    n/.style={circle,fill,draw,inner sep=3pt,node distance=1.4cm},
%%    aArrow/.style={->, >=stealth, semithick, shorten <= 2pt, shorten >= 2pt},
%%  ]
%% \node[n,label=above:$e_1$,label=below:{
%%    \relsize{-1}$\begin{array}{c}
%%      \nLeft\\[-2pt]
%%      \nSmall\\[-2pt] 
%%      \nYellow \\[-2pt] 
%%      \nBall\end{array}$}] (a) {};
%
%% \node[n,label=above:$e_2$,label=below:{
%%    \relsize{-1}$\begin{array}{c}
%%      \nLeft\\[-2pt]
%%      \nSmall\\[-2pt] 
%%      \nRed\\[-2pt] 
%%      \nCube\end{array}$}, right of=a] (b) {};
%
%% \node[n,label=below:$e_3$,label=above:{
%%    \relsize{-1}$\begin{array}{c}
%%      \nTop\\[-2pt]
%%      \nLeft\\[-2pt]
%%      \nSmall\\[-2pt] 
%%      \nYellow\\[-2pt] 
%%      \nBall\end{array}$}, above of=b] (c) {};
%
%% \node[n,label=above:$e_4$,label=below:{
%%    \relsize{-1}$\begin{array}{c}
%%      \nBig\\[-2pt] 
%%      \nRed\\[-2pt] 
%%      \nCube\end{array}$}, right of=b] (d) {};
%
%% \node[n,label=above:$e_5$,label=below:{
%%    \relsize{-1}$\begin{array}{c}
%%      \nBig\\[-2pt] 
%%      \nRed\\[-2pt] 
%%      \nBall\end{array}$}, right of=d] (e) {};
%
%% \node[n,label=above:$e_6$,label=below:{
%%    \relsize{-1}$\begin{array}{c}
%%      \nSmall\\[-2pt] 
%%      \nYellow\\[-2pt] 
%%      \nCube\end{array}$}, right of=e] (f) {};
%
%% \node[n,label=above:$e_7$,label=below:{
%%    \relsize{-1}$\begin{array}{c}
%%      \nSmall\\[-2pt] 
%%      \nRed\\[-2pt] 
%%      \nCube\end{array}$}, right of=f] (g) {};
%
%% \draw [aArrow,bend right=90] (b) to node[auto,swap]{\relsize{-1}$\nBelow$} (c);
%% \draw [aArrow,bend right=90] (c) to node[auto,swap]{\relsize{-1}$\nOntop$} (b);
%
%% \draw [aArrow,bend right=30] (d) to node[auto,swap]{\relsize{-1}$\nLeftof$} (e);
%% \draw [aArrow,bend right=30] (e) to node[auto,swap]{\relsize{-1}$\nRightof$} (d);
%
%% \draw [aArrow,bend right=30] (f) to node[auto,swap]{\relsize{-1}$\nLeftof$} (g);
%% \draw [aArrow,bend right=30] (g) to node[auto,swap]{\relsize{-1}$\nRightof$} (f);
%
%% \draw[dotted] (-.4,-1.7) rectangle (7.5,3.3);
%
%% \end{tikzpicture}
%%\caption{La escena como modelo relacional}\label{GRE3D7-stimulus-graph}
%%\end{minipage}
%%\end{figure}
%
%
%El algoritmo siempre termina, y devuelve ER un conjunto de f\'ormulas que describe cada elemento en el dominio (si existe esa f\'ormula). \\
%
%En el comienzo ER=$\{\top\}$ y $\interp{\top}$ = $\{e_1, e_2, e_3, e_4, e_5, e_6, e_7\}$ como se puede ver en la Figura~\ref{fig-modelo}.\\
%
%ACA
%\begin{figure}[ht]
%\begin{minipage}[b]{0.45\linewidth}
%\centering
%\includegraphics[width=\textwidth]{images/22.jpg}
%\vspace*{1cm}
%%\caption{Input scene}
%\label{GRE3D7-stimulus-22}
%\end{minipage}
%%\hspace*{-0.35cm}
%\begin{minipage}[b]{0.6\linewidth}
%\centering
%%\begin{figure}[ht]
%%\begin{center}
%\frame{\includegraphics[width=8cm]{images/modelo.png}}\\[0pt]
%\caption{Modelo de la Figura \ref{GRE3D7-stimulus-22}}
%\label{fig-modelo}
%\end{minipage}
%\end{figure}
%El primer bucle del algoritmo es en las propiedades. Para cada propiedad hace add$_\el$ ($\varphi$, RE), las propiedades at\'omicas se muestran en la Figura~\ref{fig-modelo2}.
%
%\begin{figure}[ht]
%\begin{center}
%\frame{\includegraphics[width=8cm]{images/modelo2.png}}\\[0pt]
%\caption{Propiedades proposicionales en cuadro rojo, las del primer ciclo del algoritmo}
%\label{fig-modelo2}
%\end{center}
%\end{figure}
%
%La f\'ormula $\varphi$ se a\~nadir\'a a ER si su interpretaci\'on tiene al menos un elemento, a continuaci\'on, para cada f\'ormula
 %$\psi$ en ER la conjunci\'on
%$\varphi  \wedge \psi$ no necesita estar subsumida in ER, la $\interp{\varphi \cup \psi}$ no tiene que ser vac\'io, y su interpretaci\'on tiene que ser distinta de $\interp{\psi}$. Luego las f\'ormulas subsumidas se borran.
%
%La primer propiedad es \textsf{ball}, ER = \{$\top$, \textsf{ball}\}, se ven los elementos de ``ball'' en un recuadro en la Figura~\ref{fig-modelo3}.
%
%\begin{figure}[ht]
%\begin{center}
%\frame{\includegraphics[width=8cm]{images/modelo3.png}}\\[0pt]
%\caption{El cuadro indica cuales son ``ball''}
%\label{fig-modelo3}
%\end{center}
%\end{figure}
%
%La siguiente propiedad es \textsf{cube}, ER = \{$\top$, \textsf{ball}, \textsf{cube}\}, pero ahora la $\interp{\textsf{ball}}$ = $\{e_1, e_3, e_5\}$, $\interp{\textsf{cube}}$ = $\{e_2, e_4, e_6, e_7\}$, quedando las particiones como se muestra en la Figura~\ref{fig-modelo4}
%\begin{figure}[ht]
%\begin{center}
%\frame{\includegraphics[width=8cm]{images/modelo4.png}}\\[0pt]
%\caption{Cuadros indicando ``ball'' y ``cube''}
%\label{fig-modelo4}
%\end{center}
%\end{figure}
%Ahora podemos borrar $\top$, porque es subsumida (esta cubierta por) las otras dos f\'ormulas. La siguiente propiedad es  \textsf{red}, $\interp{\textsf{red}}$ es: $\{e_2, e_4, e_5, e_7\}$, haciendo la intersecci\'on con la $\interp{.}$ de cada f\'ormula en ER obtenemos, $\{e_5\}$ y $\{e_2, e_4, e_7\}$, ER = $\{\textsf{ball}, \textsf{cube}, \textsf{ball} \wedge \textsf{red}, \textsf{cube} \wedge \textsf{red}\}$, las particiones actuales se pueden ver en la Figura~\ref{fig-modelo9}.
%\begin{figure}[ht]
%\begin{center}
%\frame{\includegraphics[width=8cm]{images/modelo9.png}}\\[0pt]
%\caption{Cuadros indicando ``ball'', ``cube'' y ``red''}
%\label{fig-modelo9}
%\end{center}
%\end{figure}
%
%Siguiendo con \textsf{yellow}, tenemos, $\interp{\textsf{yellow}}$ = $\{e_1, e_3, e_6\}$ y obtenemos ER = $\{\textsf{ball} \wedge \textsf{yellow}, \textsf{cube} \wedge \textsf{yellow}, \textsf{ball} \wedge \textsf{red}, \textsf{cube} \wedge \textsf{red}\}$. 
%Note que aqu\'i ya borramos la f\'ormula \textsf{ball} porque estaba subsumida, y la f\'ormula \textsf{cube} tambi\'en. Se muestran particiones en Figura~\ref{fig-modelo10}.
%
%\begin{figure}[ht]
%\begin{center}
%\frame{\includegraphics[width=8cm]{images/modelo10.png}}\\[0pt]
%\caption{Cuadros indicando ``ball'', ``cube'', ``red'' y ``yellow''}
%\label{fig-modelo10}
%\end{center}
%\end{figure}
%
%Haciendo lo mismo con \textsf{small} tenemos ER = $\{\textsf{ball} \wedge \textsf{yellow} \wedge \textsf{small}, \textsf{cube} \wedge \textsf{yellow} \wedge \textsf{small}, \textsf{ball} \wedge \textsf{red}, \textsf{cube} \wedge \textsf{red}, \textsf{cube} \wedge \textsf{red} \wedge \textsf{small}\}$, como se puede ver en Figura~\ref{fig-modelo11}.
%\begin{figure}[ht]
%\begin{center}
%\frame{\includegraphics[width=8cm]{images/modelo11.png}}\\[0pt]
%\caption{Cuadros indicando ``ball'', ``cube'', ``red'', ``yellow'', ``small'' y ``large''}
%\label{fig-modelo11}
%\end{center}
%\end{figure}
%
%La siguiente propiedad es \textsf{large} as\'i, tenemos ER = $\{\textsf{ball} \wedge \textsf{yellow} \wedge \textsf{small}, \textsf{cube} \wedge \textsf{yellow} \wedge \textsf{small}, \textsf{ball} \wedge \textsf{red}, \textsf{cube} \wedge \textsf{red} \wedge \textsf{large}, \textsf{cube} \wedge \textsf{red} \wedge \textsf{small}\}$. Aqu\'i no podemos agregar \textsf{large} a la f\'ormula $\textsf{red} \wedge \textsf{cube}$ porque su interpretaci\'on tiene un solo elemento, y la condici\'on dice que es necesario tener m\'as de uno.
%
%Hasta ahora ER = $\{\textsf{ball} \wedge \textsf{yellow} \wedge \textsf{small}, \textsf{cube} \wedge \textsf{yellow} \wedge \textsf{small}, \textsf{ball} \wedge \textsf{red}, \textsf{cube} \wedge \textsf{red} \wedge \textsf{large}, \textsf{cube} \wedge \textsf{red} \wedge \textsf{small}\}$ 
%y tenemos las siguientes extensiones: $\{e_1, e_3\}, \{e_6\}, \{e_5\}, \{e_4\}, \{e_2, e_7\}$ respectivamente. 
%Hay dos f\'ormulas que a\'un pueden ser refinadas, $\textsf{ball} \wedge \textsf{yellow} \wedge \textsf{small}$ y $\textsf{cube} \wedge \textsf{red} \wedge \textsf{small}$ 
%debido a que tienen m\'as de un elemento cada una, por lo que entran en el ciclo, while del algoritmo 1, en la l\'inea 4. Ahora es el turno de las relaciones, la primera de ellas es \textsf{leftof}, para cada f\'ormula $\varphi$ en ER trataremos de hacer add$_\el$ ($\exists \textsf{leftof}.\varphi$, RE). Notar que $\psi$ solo puede ser $\textsf{ball} \wedge \textsf{yellow} \wedge \textsf{small}$ o $\textsf{cube} \wedge \textsf{red} \wedge \textsf{small}$ porque esos son los que su interpretaci\'on tiene m\'as de un elemento. 
%\begin{figure}[ht]
%\begin{center}
%\frame{\includegraphics[width=8cm]{images/modelo15.png}}\\[0pt]
%\caption{Cuadros indicando ``ball'', ``cube'', ``red'', ``yellow''...}
%\label{fig-modelo15}
%\end{center}
%\end{figure}
%
%
%No hay
%%because those are the ones that its interpretation have more than one element. There is not 
%$\varphi$ y $\psi$ que puedan ser aplicadas. Continuando con \textsf{rightof} agregamos $\textsf{cube} \wedge \textsf{yellow} \wedge \textsf{small} \wedge \exists \textsf{rightof}. \textsf{cube} \wedge \textsf{red} \wedge \textsf{small}$, y asi con \textsf{topof} agregamos $\textsf{small} \wedge \textsf{red} \wedge \textsf{cube} \wedge \exists \textsf{ontop}. \textsf{small} \wedge \textsf{yellow} \wedge \textsf{ball}$ y el algoritmo termina con ER = $\{\textsf{ball} \wedge \textsf{yellow} \wedge \textsf{small}, \textsf{cube} \wedge \textsf{yellow} \wedge \textsf{small}, \textsf{ball} \wedge \textsf{red}, \textsf{cube} \wedge \textsf{red} \wedge \textsf{large}, \textsf{cube} \wedge \textsf{red} \wedge \textsf{small}, \textsf{cube} \wedge \textsf{yellow} \wedge \textsf{small} \wedge \exists \textsf{rightof}. \textsf{cube} \wedge \textsf{red} \wedge \textsf{small}, \textsf{small} \wedge \textsf{red} \wedge \textsf{cube} \wedge \exists \textsf{ontop}. \textsf{small} \wedge \textsf{yellow} \wedge \textsf{ball}\}$, 
%aqu\'i todos los elementos est\'an en una clase singleton y no se puede hacer ning\'un refinamiento m\'as. 
%%can be applied to $cube \wedge red \wedge small$ but there is no formula which interpretation has more than one element to be apply with this one. The same happen for the other relations, so the algorithm ends.
%%its interpretation is $\{e_7\}$ with $\psi$ is $cube \wedge yellow \wedge small$, the others combinations can't be apply because they don't do true the preconditions. The following relation is rightof, 
%
%%leftof, rightof, ontopof, bellowof
%
%%At this point we already have the target in a singleton set. So the formula for it is ``red and ball'', and also for s6 which formula is ``yellow cube''.\\
%%As we show this algorithm depends of the order of properties and relations.\\
%\begin{figure}[ht]
%\begin{center}
%\frame{\includegraphics[width=8cm]{images/modelo16.png}}\\[0pt]
%\caption{Cuadros indicando ``ball'', ``cube'', ``red'', ``yellow''...}
%\label{fig-modelo16}
%\end{center}
%\end{figure}
%
%\begin{figure}[ht]
%\begin{center}
%\frame{\includegraphics[width=8cm]{images/modelo17.png}}\\[0pt]
%\caption{Cuadros indicando ``ball'', ``cube'', ``red'', ``yellow''...}
%\label{fig-modelo17}
%\end{center}
%\end{figure}
%
%Las expresiones referenciales encontradas son:\\
%
%$\textsf{ball} \wedge \textsf{yellow} \wedge \textsf{small}$ representa $e_1$ \\
%$\textsf{cube} \wedge \textsf{yellow} \wedge \textsf{small}$ representa $e_6$ \\
%$\textsf{ball} \wedge \textsf{red}$ representa $e_5$ \\
%$\textsf{cube} \wedge \textsf{red} \wedge \textsf{large}$ representa $e_4$ \\
%$\textsf{cube} \wedge \textsf{red} \wedge \textsf{small}$ representa $\{e_2,e_7\}$  \\
%$\textsf{cube} \wedge \textsf{yellow} \wedge \textsf{small} \wedge \exists \textsf{rightof}. \textsf{cube} \wedge \textsf{red} \wedge \textsf{small}$ representa $e_6$ \\
%$\textsf{small} \wedge \textsf{red} \wedge \textsf{cube} \wedge \exists \textsf{ontop}. \textsf{small} \wedge \textsf{yellow} \wedge \textsf{ball}$ representa $e_2$ \\
%



%\section{Aproximaciones emp\'iricas a la soluci\'on de GER}
%\label{sec:trabajos_empiricos}
%
%
%
%
%\subsection{Trabajos emp\'iricos en el \'area}
%\label{sec:trab_emp}

%http://link.springer.com/chapter/10.1007/978-3-642-15573-4_9
%http://www.jetteviethen.net/papers/DaleViethen2010chapter.pdf



\subsection{Relacional}

Various researchers have attempted to extend the IA by allowing relational descriptions
(Horacek 1996; Krahmer and Theune 2002; Kelleher and Kruijff 2006), often based
on the assumption that relational properties (like ``x is on y'') are less preferred than
non-relational ones (like ``x is white''). If a relation is required to distinguish the target
x, the basic algorithm is applied iteratively to y. It seems, however, that these attempts
were only partly successful. One of the basic problems is that relational descriptions—
just like references to sets, but for different reasons—do not seem to fit in well with an
incremental generation strategy. In addition, it is far from clear that relational properties
are always less preferred than non-relational ones (Viethen and Dale 2008). Viethen
and Dale suggest that even in simple scenes, where objects can easily be distinguished
without relations, participants still use relations frequently (in about one third of the
trials).

\subsection{Otros Algoritmos}

Luego se propusieron extensiones del algoritmo Incremental, por ejemplo 
Theune y Krahmer propusieron una extensi\'on que permite la generaci\'on de referencia teniendo en cuenta la prominencia discurso del target \cite{Krahmer:2010:EMN:1880370}; \cite{krahmer}; 
y un segundo algoritmo que permite producir expresiones referenciales que contienen relaciones con otros objetos. El enfoque Theune y de Krahmer funciona asignando una puntuaci\'on de relevancia a todos los objetos de acuerdo con el enfoque / tema distinci\'on por \cite{hajicova-1993} y la teor\'ia de centrado \cite{Grosz:1995:CFM:211190.211198}. Alteran el criterio de \'exito del algoritmo y s\'olo permiten que se detenga cuando hay un distractor que es tanto o m\'as relevante que el target.
No todas las propiedades tienen la misma relevancia. Las diferencias cualitativas que existen entre diferentes propiedades se discutieron por primera vez en la literatura GER por van Deemter (2000, 2006). Se\~nal\'o que la conveniencia de otras propiedades sobre las
propiedades vagas como peque\~no y grande dependen del contexto en el que se utilizan, mientras que, por ejemplo, el color de un objeto es absoluta. 
%Considere dos descripciones de un dominio de los animales ... (van Deemter, 2002), van Deemter considera integridad l\'ogica del ia en t\'erminos de los operadores booleanos de negaci\'on y disyunci\'on. \'el la extendi\'o a
%ser capaz de generar expresiones referenciales que contienen propiedades, tales como Ejemplo (2.3) negado, y las descripciones de conjuntos de objetos, tales como Ejemplo (2.4), o incluso (2,5), que contiene una disyunci\'on l\'ogica de propiedades. Sus algoritmo procede
%en etapas, tratando m\'as y disyunciones m\'as largos de propiedades, si las propiedades at\'omicas
%y disyunciones m\'as cortos no son suficientes para distinguir el conjunto target .. que funciona en referencia a conjuntos fue tomada adem\'as por Gatt y van Deemter (2005, 2006), que han presentado los algoritmos m\'as maduros a la fecha. Utilizaron un procedimiento similar al de las otras cosas en que sus algoritmos se basan en el procesamiento incremental de un orden de preferencia de las propiedades. Sus algoritmos agregan una gran cantidad de maquinaria compleja para el procedimiento b\'asico para asegurar que las propiedades
%se eligen de manera que maximiza la coherencia dentro del conjunto de objetos descritos por las expresiones referenciales. Por ejemplo, su enfoque intentar\'a utilizar propiedades del mismo tipo para todos los referentes de un conjunto. As\'i, ser\'ia producir
%descripciones tales como ejemplos
%(2.6) or (2.7) rather than Example (2.8) or.. 


%Theune y Krahmer propusieron una extensi\'on que permite la generaci\'on de referencia con la subsiguiente ia teniendo en cuenta la prominencia discurso del referente objetivo (Krahmer y Theune, 1998; Theune, 2000; Krahmer y Theune, 2002), y un segundo uno que permite la IA para producir expresiones referenciales que contener relaciones binarias a otros objetos (Theune, 2000; Krahmer y Theune, 2002). Voy a volver a su extensi\'on relacional en la Secci\'on 2.3. Enfoque Theune y de Krahmer funciona asignando una puntuaci\'on de relevancia a todos los objetos de acuerdo con la enfoque / tema distinci\'on por Hajicova (1993) y el centrado Theory (Grosz et al., 1995). Alteran el criterio de \'exito del algoritmo y s\'olo permiten que se detenga cuando aqu\'i hay izquierda distractor que es tan o m\'as relevante que el referente de destino.
%No todas las propiedades son las mismas. Las Diferencias cualitativas que existen entre diferentes propiedades se discutieron por primera vez en la literatura reg por van Deemter (2000, 2006). Se\~nal\'o que la conveniencia de propiedades orgradable vagos como peque\~nos y grandes depende del contexto en el que se utilizan, mientras que, Por ejemplo, el color de un objeto es absoluta. Considere dos descripciones de un dominio de los animales ... (van Deemter, 2002), van Deemter considera integridad l\'ogica del ia en t\'erminos de los operadores booleanos de negaci\'on y disyunci\'on. \'el la extendi\'o a ser capaz de generar expresiones que se refieren que contienen propiedades, tales como Ejemplo (2.3) negado, y las descripciones de conjuntos de objetos, tales como Ejemplo (2.4), o incluso (2,5), que contiene una disyunci\'on l\'ogica de propiedades. Sus algoritmo procede en etapas, tratando m\'as y disyunciones m\'as largos de propiedades, si las propiedades at\'omicas y disyunciones m\'as cortos no son suficientes para distinguir el conjunto de destino .. que funciona en referencia a conjuntos fue tomada adem\'as por Gatt y van Deemter (2005, 2006), que han presentado los algoritmos m\'as maduros en este espacio para la fecha. Utilizaron un procedimiento similar al de las otras cosas en que sus algoritmos se basan en el procesamiento incremental de un orden de preferencia de las propiedades. Sus algoritmos agregar una gran cantidad de maquinaria compleja para el procedimiento b\'asico para asegurar que las propiedades se eligen de manera que maximiza la coherencia dentro del conjunto de objetos descritos por las expresiones que se refieren. Por ejemplo, su enfoque intentar\'a utilizar propiedades del mismo tipo para todos los referentes de un conjunto. As\'i, ser\'ia producir Ejemplos descripciones tales como (2,6) o (2,7) en lugar de Ejemplo (2.8) o ..

ESTo hay que pasarlo a algun lugar donde se discutan cosas de SOBREESPECIFICACION no se bien donde todavia

Paraboni et al. en su paper \cite{acl-Paraboni15} estudian la sobreespecificaci\'on de las ERs, en particular de las ERs relacionales.
  Se enfocan en qu\'e propiedad agregar a las descripci\'on del landmark. Toma como precisas:
que se ha decidido que se va a sobreespecificar la descripci\'on del landmark
y que existe una propiedad de alto poder de discriminaci\'on del landmark.
La hip\'otesis que quieren probar es la siguiente

\begin{displayquote}h1: Dado el objetivo de sobre-especificar una descripci\'on relacional usando una propiedad p extra para el landmark, p deber\'ia corresponder a la propiedad m\'as discriminatoria que est\'a disponible en el contexto. 
\end{displayquote}
La salida de la mayor\'ia de los algoritmos GER es la m\'inima ER que distingue al target \shortcite{dale91:gener}; \shortcite{krahmer}; \shortcite{graph}; pero esto no es lo que hacen los humanos, los humanos son redundantes, generan ER sobreespecificadas \shortcite{Engelhardt_Bailey_Ferreira_2006}; \shortcite{arts}; \shortcite{factors-overspec}; \shortcite{engelhardt2011over}. La discriminaci\'on de una propiedad (calculada como 1 si es el \'unico objeto que tiene la propiedad en el contexto considerado o 0 si no la tiene) juega un rol central en la tarea de desambiguaci\'on. En el art\'iculo buscan probar que la discriminaci\'on tambi\'en es importante cuando no hay nada para desambiguar, como es el caso de la sobreespecificaci\'on.


%Supongamos que queremos obtener una expresión referecial para el objeto se\~nalado por la flecha en la Figura\ref{GRE3D7-stimulus7} del GRE3D7 


\begin{figure}[ht]
\centering
\includegraphics[width=0.6\textwidth]{images/7.jpg}
\caption{Ejemplo de contexto del corpus GRE3D7}
\label{GRE3D7-stimulus7}
\end{figure}

Seg\'un \cite{pechmann} el color de un objeto es una propiedad preferida en ER minimales y sobreespecificadas. Y, en particular, en expresiones sobre-especificadas, \cite{time-course} muestran evidencia emp\'irica de que color es preferida antes que la propiedad de tama\~no.
Filtrando las ER que no son relacionales de 3 corpus seleccionados, \cite{gre3d3}, \cite{gre3d7} y Stars2 y habiendo una propiedad o relaci\'on discriminativa del landmark (siempre la hay en stars2), esta propiedad o relaci\'on es preferida para sobreespecificar la descripci\'on del landmark.
Usa el algoritmo incremental con relaciones \cite{incremental}, el cual da siempre una ER minimal, incluye relaciones como \'ultimo recurso cuando no pueden distinguir al target con las propiedades at\'omicas. Otro algoritmo que luego de hacer lo que hace el anterior, selecciona 1 propiedad m\'as para el landmark y \'esta, es la propiedad m\'as frecuente en la parte de entrenamiento de un corpus del dominio. Propone un algoritmo igual al primero al que luego le agrega una propiedad m\'as al landmark, la de mayor poder discriminativo. En caso de empate entre varias propiedades, selecciona la m\'as frecuente del corpus de entrenamiento y en caso de no haber propiedades discriminatorias, no selecciona ninguna, quedando como el primer algoritmo. Compara \cite{dice}, y Accuracy para los 3 algoritmos, para 2 de 3 corpus consigue que el tercer algoritmo mejore ambos, pero no para el GRE3D7. 
Explica que por la naturaleza del corpus GRE3D7, ya que solo el 50\% de los landmarks m\'as cercanos tiene una propiedad con alto nivel discriminativo, en vez en el corpus Stars2 todas tienen una propiedad con alto nivel discriminativo. Concluye que la propiedad o relaci\'on m\'as discriminativa ser\'a la propiedad preferida para el landmark cuando se ha decidido hacer sobreespecificaci\'on.

Dicen que como trabajo futuro se podr\'ia usar grados de discernibilidad, y esto es lo que usamos en esta tesis, luego en la Secci\'on \ref{sec:learning} usaremos la discernibilidad calculada como 1 sobre la cantidad de objetos que tienen la propiedad, en el contexto y la usaremos para calcular las probabilidades de uso de las palabras que le daremos al algoritmo.

%En nuestro trabajo para calcular las prob de uso de las palabras en el GRE3D7 , una de las propiedades fue la discriminacion, nosotros no encontramos tal correlación,  quizás por el mismo motivo q el dice (naturaleza del corpus)
%
%Me pregunto porque lo acotó a solo 1 propiedad y solo a la descripción del landmark
%Me gustaría probar q las prob de uso en el Start2, tienen fuerte correlación con la discriminacion.
%Me hubiera gustado q otro baseline en su trabajo fuera nuestro algoritmo...(me parece q tenia miedo)
%Nosotros tenemos una probabilidad de uso que nos sirve para agregar sobreespecificacion a la descripción del target y de los landmarks , no tenemos restricción de cantidad, podemos agregar varias.
%Los n\'umeros no son directamente comparables, ya q el filtro solo las relacionales y uso el total del corpus y nosotros usamos la mitad del corpus (solo verde, azul)


La investigaci\'on presentada en \cite{viet:gene11} se basa en dos premisas fundamentales: que la investigaci\'on
en la generaci\'on autom\'atica de expresiones referenciales debe esforzarse por lograr
sistemas que den salidas tan similar a la humana como sea posible; y que, para ello, debemos
esforzarnos para modelar el comportamiento humano como se puede observar en corpora.

La adopci\'on de estas premisas sirve para dos fines: en primer lugar, mejora la adecuaci\'on
de la salida de algoritmos de GER para el objeto target imitando la capacidad humana
de producir referencias adecuadas; y en segundo lugar, el estudio de corpus de datos producidos por humanos
 y algoritmos en desarrollo que pueden replicar estos datos podr\'ian
acercarnos a la comprensi\'on de que es lo que hacen los humanos cuando dan una ER.

Se\~nala que el cl\'asico algoritmo de GER y la mayor parte de sus descendientes no se basaron ni evaluaron contra datos producidos por humanos. Ellos se basaron en una visi\'on bastante minimalista de lo que se necesita
para que una expresi\'on referencial sea \'optima, concentr\'andose en la eficiencia computacional 
 y descripciones breves como sus principales preocupaciones.

 Existe un peque\~no n\'umero de enfoques que 
 se basaron en observaciones del comportamiento de referencia general humana
que obtuvieron a partir de experimentos psicoling\"u\'isticos, pero de nuevo no fueron evaluados
contra datos humanos.

Los algoritmos que se presentaron a los desaf\'ios de evaluaci\'on \cite{gatt-balz-kow:2008:ENLG} y \cite{reg2009}
fueron probados en el TUNA-Corpus, y algunos de ellos
tambi\'en tuvieron en cuenta los patrones que se encontraban en el conjunto del desarrollo. Pero hay una serie de preocupaciones en torno a la pregunta de si el TUNA-Corpus, y la forma de salida de los sistemas que se compar\'o 
en los desaf\'ios eran ideales para una evaluaci\'on de la adecuaci\'on descriptiva de GER.
%saque esto porque no se entiende
%A partir de los desaf\'ios que se describen en m\'as detalle y para evaluarlos en una serie de datos m\'as grande
%que contiene m\'as de una expresi\'on referencial para cada elemento est\'imulo.
Aborda tres \'areas principales en las que los corpus se puede utilizar para
promover el objetivo de la semejanza humana en la investigaci\'on sobre generaci\'on de expresiones referenciales:
evaluaci\'on, la recolecci\'on de corpus y an\'alisis y modelizaci\'on estad\'istica de
datos de corpus.
Comenz\'o con el an\'alisis del estado del arte de la investigaci\'on en
la generaci\'on de descripciones distintivas, trabaj\'o con relaciones espaciales y en el trabajo us\'o datos de corpus. Examin\'o una serie de opciones metodol\'ogicas que tienen que hacerse cuando se trabaja con los corpus de GER. Aqu\'i, explor\'o diferentes opciones para los desaf\'ios de recopilaci\'on de corpus, que se centran en torno al equilibrio que se necesita
entre el control de los par\'ametros experimentales tanto como sea necesario
y mantener la configuraci\'on de lo m\'as natural posible. Discuti\'o una serie de conceptos
que son de importancia para el an\'alisis de corpora de GER, tales como la naturalidad de diferentes
propiedades de los objetos, y las nociones de minimalidad y cuestiones de sobre-especificaci\'on de expresiones referenciales. Por \'ultimo, analiz\'o diferentes maneras en que la salida de un sistema se puede
comparar con los datos de corpora, bajo la premisa de que el objetivo de la comparaci\'on es
para evaluar si el sistema podr\'ia tener un modelo adecuado del comportamiento humano para la generaci\'on de expresiones referenciales.
Realiz\'o una investigaci\'on en las tres \'areas donde se puede emplear corpora en GER: Evaluaci\'on de semejanza humana, recopilaci\'on y an\'alisis de corpus, y modelado de datos de corpus. Realiz\'o un experimento de evaluaci\'on con tres de los algoritmos cl\'asicos, (1989) Algoritmo Greedy de Dale (Greedy), Dale y Haddock (1991b), Algoritmo Relacional (ra) Y Dale y Reiter (1995) Algoritmo Incremental (IA), Se pusieron a prueba en cuanto a su capacidad de
replicar las expresiones referenciales se encuentran en un grupo relativamente peque\~no de corpus de expresiones referenciales
en un dominio visual de im\'agenes puestas en una grilla.
%de rejilla de cajones de armarios ling. 
En el an\'alisis de este experimento tuvo dos resultados principales: (1) que identic\'o en particular tres 
fen\'omenos que todav\'ia plantean importantes retos para los algoritmos GER con el objetivo de replicar
el comportamiento humano, y (2) que proporciona una plataforma para la discusi\'on de una serie de
dificultades que se presentan para la evaluaci\'on basada en corpus de GER. Esto result\'o en una serie
de criterios para el dise\~no de los dos corpus que el trabajo en el resto de la tesis.
Los tres fen\'omenos en las expresiones referenciales producidas por humanos que los
algoritmos probados no fueron capaces de replicar satisfactoriamente sobre-especificaci\'on,
relaciones espaciales, y comportamiento de voluntarios espec\'ificos. Ambos
Greddy y la IA fueron capaces de generar algo de la redundancia que se encontr\'o en el corpus.
 Ni Greedy ni el IA estaban dise\~nados para ser capaz de generar expresiones referenciales que contengan relaciones entre entidades, pero el ra fue dise\~nado para incluirlas. Sorprendentemente, el
ra no s\'olo fallo en generar cualquiera de las descripciones contenidas en el corpus de evaluaci\'on; sino tambi\'en que las descripciones que se gener\'o parec\'ian m\'as como enigmas cuyo objetivo era confundir a un oyente, m\'as que ayudar en
los intentos de se\~nalar el objetivo referente. 

Una valoraci\'on te\'orica de otras aproximaciones dise\~nados para manejar las relaciones estableci\'o que ninguno de ellos incluir\'ia una relaci\'on, si no es absolutamente necesaria para distinguir el target.
La tercera observaci\'on que el experimento dejo a la vista fue que la gente no siempre hace lo mismo en la misma situaci\'on. De hecho, incluso la misma persona podr\'ia describir el mismo target de diferente manera en distintas
circunstancias. 

 None of the algorithms tested were intended to take such inter- and
intra-speaker variation into account, and only very recently have implementations
of the
ia
begun to model speaker-preferences to some degree.

No se pretend\'ia que ninguno de los algoritmos de la prueba tomara en cuenta las variaciones entre-hablantes, ni del mismo hablante. Hay implementaciones del IA que han comenzado a agregar modelo de preferencias de hablante en alg\'un grado.

Los temas generales con evaluaci\'on basada en corpus que esta experiencia de evaluaci\'on dej\'o
al descubierto fueron (1) la interdependencia estrecha entre algoritmos y la
representaci\'on subyacente de conocimiento que utilizan, (2) el no-determinismo de la generaci\'on del lenguaje natural, (3) la cuesti\'on de c\'omo comparar la salida algoritmos con gold-standar, y (4) el dominio espec\'ifico de los algoritmos de GER.\\
La discusi\'on de estos temas ha dado lugar a la siguiente lista de Evaluaci\'on en GER basado en corpus:

1. Si el corpora est\'a destinado para ser reutilizado para la evaluaci\'on comparativa de diferentes
algoritmos, una representaci\'on subyacente del dominio debe ser proporcionada para ser usada por todos los algoritmos.

2. Si queremos confiar en los resultados, el corpus debe contener tantos casos como sea posible de tantos direntes hablantes como sea posible para cada escenario referencial. 
Esto es cierto si un algoritmo es evaluado en t\'erminos de ser capaz de generar una expresi\'on referencial que suene natural, o si est\'a probado por su probabilidad de pertenecer a un modelo espec\'ifico de conducta humana de referencia, mediante la comprobaci\'on de si se puede generar todas las descripciones en un corpus.

3. Si la probabilidad de un algoritmo de ser un modelo de la conducta humana de referencia
se eval\'ua, deben utilizarse m\'etricas basadas en recuento (RECALL) y precisi\'on (PRESITION). En este
caso, el conjunto completo de las descripciones que el algoritmo proporciona para cada
escenario de referencia en virtud de cualquier ajuste de par\'ametro debe ser comparado con el
conjunto de las descripciones contenidas en el corpus para el mismo escenario referencial.
Si la capacidad m\'as orientada a la aplicaci\'on para generar una referencia es similar a la humana
se ha de evaluar, s\'olo una descripci\'on por escenario.
Esto debe hacerse utilizando m\'etricas basadas en la precisi\'on para probar c\'omo muchos de las
descripciones dadas por los algoritmos est\'an contenidas en el corpus.
4. Algoritmos que son juzgados en un dominio espec\'ifico c, no se debe asumir como
f\'acilmente adaptable a otros dominios. Idealmente, los corpus que abarcan muchos diferentes
deben estar disponibles para las pruebas de los algoritmos que generalizan en distintos tipos de dominios.
Realiz\'o 2 corpus el GRE3D3 y el GRE3D7 descriptos en \ref{sec:corpus2}

Trabaj\'o con \'arboles de decisi\'on ...
\textcolor{blue}{Aca agregar otras personas que usaron corpus para generacion de ER, Ivandre... }

%http://www.lrec-conf.org/proceedings/lrec2012/pdf/152_Paper.pdf
En el trabajo \ref{ivandre-work-corpus} presentan 2 alternativas para aprender la seleccio\'on de atributos de una expresi\'on referencial a partir de corpus. Toma los caracter\'isticas de aprendizaje como el conjunto de valores enteros que representan el poder discriminativo de cada atributo (es decir, el n\'umero de distractores que cada atributo elimina por cada atributo que tiene el target, ejemplos son color, tama\~no, etc.) 


Kelleher and Kruiff 2006  hay que agregar aca, porque en parte intro logica se usa
Estudia la generaci\'on de ER locativas en escenas din\'amicas, reduce la complejidad achicando el modelo.


Gardent 2002 genera plurales, generalizando el algoritmo incremental con disyuncion de propiedades (LEER PAPER), hay que agregar aca, porque en parte intro logica se usa y es interesante por lo de plurales

Van Deemter 2001 extiende algoritmo incremental para plurales con negacion, conj y disy de propiedades (LEER PAPER)




\section{M\'etricas de evaluaci\'on/comparaci\'on con corpus}
\label{sec:metricas_evaluacion}
Jordan y Walker usaron 25 veces la validaci\'on cruzada en 393 expresiones referenciales
de 13 de la Coco di\'alogos para probar diferentes combinaciones de caracter\'isticas. Ellos
medido la precisi\'on absoluta, siendo \'esta la proporci\'on de expresiones referenciales
generadas que son id\'enticas a las descripciones de referencia humanos producidos a partir de
el corpus. En el aislamiento, la intencional uencias factores desempe\~naron mejor (42,4\%
exactitud) que los otros dos conjuntos de caracter\'isticas (conjuntos de contraste: 30,4\% y conceptual
pactos: 28,9\%) y la combinaci\'on de los tres tipos de caracter\'isticas hicieron signicativa no inexactitud aumento (43,2\%). Sin embargo, lo que tuvo el mayor impacto fue cuarto,
independiente de la teor\'ia, el tipo de caracter\'isticas que registran informaci\'on al juicio espec\'ifico c, tales
como el juicio
Identificaci\'on
, El participante-diada, el hablante actual y el atributo exacta
valores de la referente de destino. En el aislamiento, esta colecci\'on de caracter\'isticas logra el 54,5\%
exactitud, y combin\'andolos con todos los otros tres tipos de caracter\'isticas s\'olo aumentaron
esta actuaci\'on al 59,9\%. Estos resultados apoyan leve a Jordan intencional
en
influencias modelo sobre los otros dos modelos, pero m\'as fuertemente sugieren que ninguno
de los modelos de capturar la variaci\'on en los datos muy bien
incursi\'on en el uso de la m\'aquina de aprendizaje para
reg
fue hecha por Stoia et al.
(2006). Ellos apuntan a la construcci\'on de un sistema de di\'alogo para un agente situado dando
instrucciones en un mundo virtual en 3D. Sin embargo, este enfoque no se centr\'o por lo
tanto en la selecci\'on de contenidos como en determinar la mejor forma de referencia a utilizar.
Utilizaron un aprendiz m\'aquina para entrenar a los \'arboles de decisi\'on que decidieron que determinador
utilizar, qu\'e tipo de cabeza para incluir en el sintagma nominal (por ejemplo, un pronombre o una
nombre com\'un) y si desea o no utilizar una frase modi sustantivo. La sem\'antica
contenido del modificador no estaba en cuesti\'on. Las funciones disponibles para la decisi\'on
alumnos de los \'arboles eran una mezcla de la historia del di\'alogo, contexto visual y tipo sem\'antico
informaci\'on sobre el referente objetivo. Entrenaron \'arboles de decisi\'on separados para de-
terminer, sustantivo principal y la elecci\'on modificador y les aplicaron secuencialmente, con cada \'arbol que tenga acceso a la salida del \'arbol anterior. Para el entrenamiento y autom\'atico
%evaluaci\'on que utilizan un conjunto de 1242 expresiones referenciales de una colecci\'on de di\'alogo
%Logues entre dos compan\~eros de conversaci\'on que llevaban a cabo la instrucci\'on
%tarea en el mismo mundo virtual como el sistema se emplea en adelante. Esta
%evaluaci\'on autom\'atica encontr\'o que los \'arboles de decisi\'on fueron capaces de igualar la humana datos en el 31\% de todos los casos. Como no estaban interesados ​​tanto en la semejanza humana
%de su sistema, pero sobre todo en su cacia e, tambi\'en realizaron un intr\'inseca
%Evaluaci\'on humana en la que se pidi\'o a los participantes para comparar la salida del sistema
%a las expresiones que se refieren humanos producidos en una l\'inea de base y al azar. El humano
%evaluadores juzgados 62,6\% de las expresiones que se refieren generados por el sistema para


Un n\'umero de los sistemas presentados a los desaf\'ios GER de evaluaci\'on basados
en el TUNA Corpus se basaron en los an\'alisis emp\'iricos del conjunto de entrenamiento. La mayor\'ia
de estos sistemas se basaban en la
ia y se utiliza un simple recuento de frecuencia de la
propiedades en el conjunto de entrenamiento para informar el orden en que Propie- del referente de destino
propie- deben ser juzgados (Kelleher, 2007; Spanger et al, 2007;. Fabbrizio et al., 2008;
Kelleher y Namee, 2008; de Lucena y Paraboni, 2008; Gerv como et al, 2008?.;
de Lucena y Paraboni, 2009). Un equipo, que yo era parte de, que se utiliza en frecuencia
funciones basadas en costos en el algoritmo gr\'afico-Based (Theune et al, 2007; Krahmer.
et al., 2008; Brugman et al., 2009). Bohnet (2007, 2008, 2009) nearest- combinado
vecino de aprendizaje con un enfoque brevedad completa, con el fin de elegir el m\'as corto
refiri\'endose expresi\'on que mejor se ajuste a los datos de entrenamiento para un determinado objetivo; y
utiliza un \'arbol de decisi\'on aprendido de los datos de entrenamiento para determinar din\'amicamente el
orden de preferencia para la
ia
. En 2008 y 2009, Bohnet adapta su brevedad completo al-
goritmo para que coincida con los participantes individuales, pero encontr\'o que la informaci\'on del participante
no se proporcion\'o de forma fiable en los datos de prueba. Fabbrizio et al. (2008) present\'o la
\'unico otro enfoque que intent\'o capturar las preferencias de c-altavoz espec\'ifico en
la brevedad completo y el algoritmo incrementales. Su enfoque completo brevedad recogi\'o
las descripciones m\'as cortas que estaba bien con m\'as frecuencia o m\'as recientemente utilizado por el
mismo orador, y su versi\'on de la
ia
basada en frecuencia usada altavoz espec\'i c
\'ordenes de preferencias. King (2008) y Herv\'e? Como y Gerv? As (2009) utilizaron evolutiva
programaci\'on para la tarea de selecci\'on de contenido, pero ambos se encontraron con muy limitado
el \'exito.

Investigaci\'on REG Pre-2000 dio poca o ninguna atenci\'on a la evaluaci\'on emp\'irica de algoritmos. M\'as recientemente, sin embargo, los estudios de evaluaci\'on REG han comenzado a realizar
m\'as y m\'as a menudo. Al parecer, la mayor\'ia de ellos se basaban en el supuesto de
(debatidas en la Secci\'on 7) que los algoritmos REG deber\'ian tratar de generar expresiones que son
\'optimamente similares a los producidos por hablantes o escritores humanos, incluso aunque-
importante- este supuesto rara vez se hace expl\'icito. El m\'etodo dominante en
el momento est\'a, por consiguiente, para medir la similitud entre las expresiones generadas
y los de un corpus adecuadas de expresiones que se refieren. REG lleg\'o tarde a basada corpus-
evaluaci\'on (en comparaci\'on con otras partes de la lingü\'istica computacional) porque los datos adecuados
conjuntos son dif\'iciles de conseguir. En esta secci\'on, se discuten cu\'ales son los criterios de un conjunto de datos debe cumplir
para que sea adecuado para la evaluaci\'on REG, y estudiar qu\'e colecciones est\'an actualmente
disponible. Adem\'as, se discute c\'omo uno es para determinar el rendimiento de un REG
algoritmo en un determinado conjunto de datos. Veremos que aunque mucho se ha trabajado en
los \'ultimos a\~nos, todav\'ia hay cuestiones abiertas importantes, particularmente con respecto a la relaci\'on
entre las m\'etricas autom\'aticas y juicios humanos

Para medir la performance de los algoritmos podemos usar m\'etricas autom\'aticas o m\'etricas manuales, las m\'etricas autom\'aticas son aquellas que se calculan mediante un algoritmo y las manuales en las cuales les requerimos a personas que evaluen las expresiones referenciales.


\subsubsection{M\'etricas autom\'aticas}


Si hay corpus disponible, una m\'etrica autom\'atica de evaluaci\'on es comparar la ER dada por el sistema para un target en un contexto dado con la ER (gold standar) dada por una persona para el mismo target y contexto.
Esta comparaci\'on de ER puede estar dada a distintos niveles, podemos comparar si son iguales, si solo difieren en el orden de las palabras, si difieren en las palabras pero no en la cantidad de palabras que contienen, etc. En lo que sigue nombramos algunas m\'etricas de evaluaci\'on autom\'atica.\\

La exactitud (Accuracy) se define como el porcentaje de coincidencias exactas entre cada RE producida por un ser humano y la producida por el sistema para la misma escena y target. Se considera que es una m\'etrica demasiado estricta.

El coheficiente Dice es una m\'etrica de comparaci\'on de conjuntos, el valor va entre 0 y 1, 1 indica un perfecto match entre los conjuntos. Para dos conjuntos A y B, Dice se calcula como sigue:\\

$Dice(A,B) = \frac{2\times|A \cap B|}{|A|+|B|}$\\

\textsc{masi} de \cite{masi}~es una adaptaci\'on de el coheficiente Jaccard el cual varia en favor de la similaridad cuando un conjunto es un subconjunto de otro, como Dice varia entre 0 y 1, 1 indica match perfecto. Se calcula como sigue:\\
%which biases it in favor of similarity where one set
%is a subset of the other. Like Dice, it ranges between
%0 and 1, where 1 indicates a perfect match. It is computed as follows:\\

$\textsc{masi}(A,B) = \delta \times \frac{|A \cap B|}{|A \cup B|}$ \\


donde $\delta$ es un coheficiente definido como sigue:\\


 \begin{equation}
     \delta  = \left\{
	       \begin{array}{ll}
		 0      & if A \cap B = \emptyset \\
		 1 & if A = B  \\
		 \frac{2}{3}     & if A \subset B ~or~ B \subset A\\
		 \frac{1}{3}     & otherwise
	       \end{array}
	     \right.
 \end{equation}

Intuitivamente significa que se prefieren aquellas descripciones producidas por el sistema las cuales no incluyen atributos que los humanos no incluyeron.
%Intuitively, this
%means that those system-produced descriptions are
%preferred which do not include attributes that are
%omitted by a human.  

Teniendo corpus disponible, y teniendo en cuenta que nuestro algoritmo produce diferentes ER en cada ejecuci\'on tambi\'en podemos comparar automaticamente ambas distribuciones de ER.


Cobertura: todas las ER que estan en el corpus fueron producidas por el sistema?







\subsubsection{M\'etricas manuales}

Se trata de poner a jueces a evaluar las ER, esto se puede hacer de distintas maneras, y se pueden evaluar distintas cosas, por ejemplo, que tan natural es la ER, si ayuda o no a identificar al target pretendido. 

Hicimos una evaluaci\'on manual en la que pedimos a 2 jueces que elijan cual ER les es m\'as natural, se les mostro el contexto, y se les present\'o 2 ER una dada por el sistema y otra proveniente del corpus dada por un humano.


\section{Notas finales y linkeo del cap\'itulo}

En este cap\'itulo presentamos las premisas en las que se basa la teor\'ia estandar, vimos experimentos que muestran que dadas ciertas condiciones las personas no siguen esas premisas, \cite{keysar:Curr98}. Tamb\'en vimos que algo com\'un es que las personas den ER sobreespecificadas, \cite{arts}, \cite{do-speakers}, los experimentos indican que por lo menos la tercera parte de las expresiones son sobreespecificadas, y que los oyentes no juzgan estas ER como peores que la minimales. Otras conclusiones encontradas son que la sobreespecificaci\'on permite la identificaci\'on m\'as r\'apida del target y que la informaci\'on adicional (arriba, abajo) es m\'as \'util que la (izquierda, derecha). \cite{Lu_sasha2015} hicieron un experimento de aprendizaje de palabras en una lengua extranjera, y concluyen que las personas que tuvieron a su disposici\'on ER sobreespecificadas, aprendieron m\'as que aquellas a las que se les di\'o ER minimales.  Vimos los diferentes tipos de ER, los diferentes tipos de algoritmos, dimos los algoritmos m\'as importantes del \'area, vimos sus diferencias y pudimos compararlos entre ellos, en el pr\'oximo Cap\'itulo veremos una introducci\'on a la simulaci\'on.
