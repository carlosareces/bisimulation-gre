\chapter{Definiendo optimalidad de una ER}
\label{sec:metodologia}

\section{Propiedades de una ER: relacional, minimal, sobreespecificada, underespecificada}

Cuando la ER no es relacional solo contiene propiedades del objeto mismo. Ej.: color, tama�o. Note que quiz\'as el tama�o sea con respecto a los dem�s objetos ``la m�s peque�a'', pero al no incluir una descripci�n de otro objeto no la llamamos relacional.\\

Una ER es {\it relacional}, cuando contiene relaciones con otros objetos, entonces aqu\'{i} se incluye en desaf\'{i}o de describir al siguiente objeto. Ej.: ``La que esta al lado del cubo rojo''.\\

Se dice que una ER es minimal, cuando incluye la m\'inima cantidad de propiedades con la cual puede ser identificada un�vocamente en el contexto dado.\\

Una ER es sobreespecificada cuando contiene m�s informaci�n de la m�nima necesaria para identificarla un�vocamente de los dem�s objetos en el contexto dado.\\

Underespecificada es cuando la expresi�n dada, no alcanza a ser una ER, ya que no alcanza para identificar al objeto del resto de los obejetos en la imagen. En estos casos se debe dar otra expresi�n o corregir la anterior agregando m�s propiedades o relaciones.


\section{Propiedades de un algoritmo: determin\'{i}stico, no-determin\'{i}stico, referencial, incluye negaciones, identifica plurales, usa disyunciones, conjunciones, genera sobrerespecificaci�n}

En esta parte definiremos el problema en general, cuando tenemos muchas personas, tenemos muchas ER, que pueden ser distintas. Los algoritmos deber�an dar la salida como muchas personas dar�an.

Un algoritmo es determin\'{i}stico si siempre dado un input (una escena, con un objeto target), da siempre la misma ER de salida. En cambio es no-determin\'{i}stico si considera la posibilidad de dar distintas salidas para distintas ejecusiones simulando el comportamiento de distintas personas, o incluso el de la misma persona en distintos momentos.

Un algoritmo es referencial si permite generar ER relacionales, en cuyo caso adem�s de generar las relaciones correspondientes deber� generar una ER para el objeto con el cual el objeto target tiene la relaci�n. Notar que las relaciones pueden ser con muchos objetos, es decir de una aridad mayor que 1. Ej.: ``el restaurante que esta en la esquina de Gral. Paz y Col�n''.

\textcolor{blue}{OBVIO! - Un algoritmo que incluye negaciones es uno que puede incluir negaciones}
\textcolor{blue}{identifica plurales, usa conjunciones disyunciones}

La sobreespecificaci�n es la caracter�stica de dar m�s atributos o relaciones de las necesarias para identificar al objeto target. Ser�a interesante que un algoritmo diera la sobreespecificaci�n que las personas dan. Ej.: ``el restaurante que esta en la esquina de Gral. Paz y Col�n, al frente del correo''.


\section{Problemas de corpus existente y/o de anotaci�n}

\textcolor{blue}{listado de corpus existentes, quizas traer aca los del paper}

\section{Comparando la salida del sistema con ER hechas por humanos}

Una manera de evaluar un algoritmo de GER ser�a comparar un gran n�mero de corridas del algoritmo con la frecuencia de probabilidades de las ocurrencias de las ER en el corpus de ER hechas por humanos. Es decir supongamos que el 90\% de las personas dijo ``la pelota roja'' ser�a interesante encontrar que el algoritmo genere el 90\% de las veces ``la pelota roja'', al mismo tiempo que tambi�n es importante que el algoritmo sea capaz de generar otras ER, las dadas en el 10\% restante y con una distribuci�n de probabilidad similar. 

\textcolor{blue}{agregar ejemplo paper de colling aca}

\section{Problemas en la evaluaci�n de GRE}
\section{Corpus seleccionados}
\section{Optimalidad de expresiones referenciales}

\subsection{Grice...}
\subsection{Paraboni?}
\subsection{Naturalidad Dale, Jette}

