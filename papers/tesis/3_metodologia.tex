\chapter{Definiendo optimalidad de una ER}
\label{sec:metodologia}

The concept of reference is difficult to pin down exactly (Searle 1969; Abbott 2010).
Searle therefore suggests that the proper approach is ``to examine those cases which con-
stitute the center of variation of the concept of referring and then examine the borderline
cases in light of similarities and differences from the paradigms'' (Searle 1969, pages
26-27). The ``paradigms'' of reference in Reiter and Dale (2000) are
definite descriptions
whose primary purpose it is to
identify their referent
. The vast majority of recent REG
research subscribes to this view as well. Accordingly these paradigmatic cases will also
be the main focus of this survey, although we shall often have occasion to discuss other
types of expressions. However, to do full justice to indefinite or attributive descriptions,
proper names, and personal pronouns would, in our view, require a separate, additional
survey.

\section{Problemas de corpus existente y/o de anotaci�n}

\textcolor{blue}{listado de corpus existentes, quizas traer aca los del paper}

\section{Comparando la salida del sistema con ER hechas por humanos}

Una manera de evaluar un algoritmo de GER ser�a comparar un gran n�mero de corridas del algoritmo con la frecuencia de probabilidades de las ocurrencias de las ER en el corpus de ER hechas por humanos. Es decir supongamos que el 90\% de las personas dijo ``la pelota roja'' ser�a interesante encontrar que el algoritmo genere el 90\% de las veces ``la pelota roja'', al mismo tiempo que tambi�n es importante que el algoritmo sea capaz de generar otras ER, las dadas en el 10\% restante y con una distribuci�n de probabilidad similar. 

\textcolor{blue}{agregar ejemplo paper de colling aca}

\section{Problemas en la evaluaci�n de GER}
\section{Corpus seleccionados}
\section{Optimalidad de expresiones referenciales}

\subsection{Grice...}
\subsection{Paraboni?}
\subsection{Naturalidad Dale, Jette}

