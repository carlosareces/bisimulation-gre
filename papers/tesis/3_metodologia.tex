\chapter{Definiendo optimalidad de una ER}
\label{sec:metodologia}

%The concept of reference is difficult to pin down exactly (Searle 1969; Abbott 2010).
%Searle therefore suggests that the proper approach is ``to examine those cases which con-
%stitute the center of variation of the concept of referring and then examine the borderline
%cases in light of similarities and differences from the paradigms'' (Searle 1969, pages
%26-27). The ``paradigms'' of reference in Reiter and Dale (2000) are
%definite descriptions
%whose primary purpose it is to
%identify their referent
%. The vast majority of recent REG
%research subscribes to this view as well. Accordingly these paradigmatic cases will also
%be the main focus of this survey, although we shall often have occasion to discuss other
%types of expressions. However, to do full justice to indefinite or attributive descriptions,
%proper names, and personal pronouns would, in our view, require a separate, additional
%survey.

El concepto de referencia es dificil de precisar con exactitud (Searle 1969; Abbott 2010). Searle, por tanto,
sugiere que el enfoque correcto es `` para examinar los casos que constituyen el centro de la variaci\'on
del concepto de referencia y luego examinar los casos l\'imite a la luz de las similitudes y diferencias
de los paradigmas'' (Searle 1969, p\'aginas 26-27). Los Paradigmas de referencia en Reiter y Dale (2000)
son descripciones cuyo prop\'osito principal es identificar su referente. La gran mayor\'ia de
reciente investigaci\'on GER suscribe este punto de vista tambi\'en. En consecuencia estos casos paradigm\'aticos tambi\'en ser\'an
el objetivo principal de este estudio, a pesar de que se suele tener ocasi\'on de discutir otros tipos de expresiones.
Sin embargo, para hacer justicia a las indefinidas o descripciones atributivas, nombres propios, y los pronombres personales
ser\'ia, a nuestro juicio, requieren un estudio independiente, adicional.

\section{Problemas de corpus existente y/o de anotaci\'on}

\textcolor{blue}{listado de corpus existentes, quizas traer aca los del paper}

\section{Comparando la salida del sistema con ER hechas por humanos}

Una manera de evaluar un algoritmo de GER ser�a comparar un gran n�mero de corridas del algoritmo con la frecuencia de probabilidades de las ocurrencias de las ER en el corpus de ER hechas por humanos. Es decir supongamos que el 90\% de las personas dijo ``la pelota roja'' ser�a interesante encontrar que el algoritmo genere el 90\% de las veces ``la pelota roja'', al mismo tiempo que tambi�n es importante que el algoritmo sea capaz de generar otras ER, las dadas en el 10\% restante y con una distribuci�n de probabilidad similar. 

\textcolor{blue}{agregar ejemplo paper de colling aca}

\section{Problemas en la evaluaci�n de GER}
\section{Corpus seleccionados}
\section{Optimalidad de expresiones referenciales}

\subsection{Grice...}
\subsection{Paraboni?}
\subsection{Naturalidad Dale, Jette}

