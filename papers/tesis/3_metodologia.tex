\chapter{Definiendo optimalidad de una ER}
\label{sec:metodologia}



\section{Problemas de corpus existente y/o de anotaci�n}

\textcolor{blue}{listado de corpus existentes, quizas traer aca los del paper}

\section{Comparando la salida del sistema con ER hechas por humanos}

Una manera de evaluar un algoritmo de GER ser�a comparar un gran n�mero de corridas del algoritmo con la frecuencia de probabilidades de las ocurrencias de las ER en el corpus de ER hechas por humanos. Es decir supongamos que el 90\% de las personas dijo ``la pelota roja'' ser�a interesante encontrar que el algoritmo genere el 90\% de las veces ``la pelota roja'', al mismo tiempo que tambi�n es importante que el algoritmo sea capaz de generar otras ER, las dadas en el 10\% restante y con una distribuci�n de probabilidad similar. 

\textcolor{blue}{agregar ejemplo paper de colling aca}

\section{Problemas en la evaluaci�n de GRE}
\section{Corpus seleccionados}
\section{Optimalidad de expresiones referenciales}

\subsection{Grice...}
\subsection{Paraboni?}
\subsection{Naturalidad Dale, Jette}

