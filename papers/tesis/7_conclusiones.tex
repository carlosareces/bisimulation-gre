\chapter{Conclusiones}
\label{sec:conclusiones}

%\subsection{Conclusiones de la aproximaci\'on}

%In this article we extend~\cite{arec2:2008:Areces} algorithm to generate REs similar to those produced by humans. The modifications 
%proposed are based on two observations. First, it has been argued that no fixed ordering of properties is able to generate all the REs produced by humans and, second, humans frequently overspecify their REs~\cite{Engelhardt_Bailey_Ferreira_2006,Arts_Maes_Noordman_Jansen_2011}. We tested 
%the proposed algorithm on the GRE3D7 corpus and found that it is able to generate a large proportion of the overspecified REs found in the corpus without generating trivially redundant referring expressions.

%\cite{viet:gene11} trains decision trees that are able to achieve a 65\% average accuracy on the GRE3D7 corpus. 
%The approach based on decision trees is able to generate overspecified relational descriptions, but they might fail to be referring 
%expressions. Indeed, as the decision trees does not verify the extension of the generated expression over a model of the scene, the 
%generated descriptions might not uniquely identify the target.  As we have already discussed,
%our algorithm ensures termination and it always finds a referring expression if one exists.  Moreover, it achieves an average of 75.03\% of accuracy over the scenes used in our tests. 

%Different algorithm for the generation of overspecified and distinguishing referring expressions has been proposed in recent years 
%(see, e.g.,~\cite{delucena-paraboni:2008:ENLG,ruud-emiel-mariet:2012:INLG2012}.  But, to our knowledge, they have not been evaluated on the 
%GRE3D7 corpus and, hence, comparison is difficult. \cite{delucena-paraboni:2008:ENLG} and \cite{ruud-emiel-mariet:2012:INLG2012} algorithm
%have been evaluated on the TUNA-AR corpus~\cite{gatt-balz-kow:2008:ENLG} where they have achieved a 33\% and 40\% accuracy respectively. 
%As the TUNA-AR corpus includes only propositional REs, it would be interesting future work to evaluate how these algorithms perform in corpora with relational REs such as GRE3D7. 

%A second theme worth discussing is how our algorithm deals with overspecification. As we described in Section~\ref{sec:overspecification} the generation of overspecified REs is performed in two steps. In the first iteration, the probability of including a property in the RE depends only on its \puse. It does not matter whether the property actually eliminate any distractor; hence, the resulting RE may be overspecified. After all properties had a chance of being included this way, if the resulting RE is not distinguishing, then the algorithm enters a second phase in which it makes sure that the RE identifies the target uniquely.  This model is inspired by the work of~\cite{keysar:Curr98} on egocentrism and natural language production.  Keysar~et al.\ put forwards the proposal that when producing language, considering the hearers point of view is not done from the outset but it is rather an afterthought~\cite{keysar:Curr98}. They argue that adult speakers produce REs egocentrically, just like children do, but then adjust the REs so that the addressee is able to identify the target unequivocally. The first, egocentric, step is a heuristic process based in a model of saliency of the scene that contains the target. 

%Our definition of \puse\ is intended to capture the saliences of the properties for different scenes and targets. The \puse\ of a property changes according to the scene as we discussed in Section~\ref{subsec:learning}. This is in contrast with previous work where the saliency of a property is constant in a domain. Keysar et al.~argue that the reason for generate-and-adjust procedure may have to do with information processing limitations of the mind:if the heuristic that guides the egocentric phase is well tunned, it succeeds with a suitable RE in most cases and seldom requires adjustments. Interestingly, we observe a similar behavior with our algorithm:when \puse\ values learn from the domain are used, the algorithm is not only much more accurate but also much faster. 

%As future work we plan to evaluate our algorithm on more complex domains like those provided by Open Domain Folksonimies~\cite{pacheco-duboue-dominguez:2012:NAACL-HLT}. We also plan to explore corpora obtain from interaction, such as the GIVE Corpus~\cite{GarGarKolStr10} where it it is common to observe multi shot REs. Under time pressure subjects will first produce an underspecified expression that includes salient properties of the target (e.g., ``the red button'').  And then, in a following utterance, they add additional properties (e.g., ``to the left of the lamp'') to make the expression a proper RE  identifying the target uniquely.

En esta tesis extendimos el algoritmo~\cite{arec2:2008:Areces} para generar ER similares a las producidas por los humanos. Las modificaciones
propuestas se basan en dos observaciones. En primer lugar, se ha argumentado que hay orden fijo de propiedades que es capaz de generar todas las ER producidas por los seres humanos y, en segundo lugar vimos que los seres humanos con frecuencia sobreespecifican sus ER\cite{Engelhardt_Bailey_Ferreira_2006, Arts_Maes_Noordman_Jansen_2011}. Probamos
el algoritmo propuesto en el corpus GRE3D7 y se encontr\'o que es capaz de generar una gran proporci\'on de las REs sobreespecificadas que se encuentran en el corpus sin generar expresiones referenciales trivialmente redundantes.

\cite{viet:gene11} entren\'o \'arboles de decisi\'on que son capaces de lograr una precisi\'on media del 65\% en el corpus GRE3D7.
El enfoque basado en \'arboles de decisi\'on es capaz de generar descripciones relacionales sobreespecificadas, 
pero que podr\'ian no ser expresiones referenciales.
De hecho, como los \'arboles de decisi\'on no verifican la extensi\'on de la expresi\'on generada m\'as de un modelo de la escena, la
descripci\'on generada podr\'ia no identificar de forma \'unica el objetivo. Como ya hemos comentado,
nuestro algoritmo asegura la terminaci\'on y siempre encuentra una expresi\'on referencial, si existe. Por otra parte, logra un promedio de 75,03 \% de precisi\'on en las escenas utilizadas en nuestras pruebas.

Algoritmos diferentes para la generaci\'on de expresiones referenciales sobreespecificadas y distintivas se ha propuesto en los \'ultimos a\~nos
(Ver, por ejemplo,~\cite{delucena-paraboni:2008:ENLG, ruud-emiel-mariet:2012:INLG2012}). Pero, hasta donde sabemos, no han sido evaluados en el
GRE3D7 corpus y, por lo tanto, la comparaci\'on es dif�cil.  \cite{delucena-paraboni:2008:ENLG} y \cite{ruud-emiel-mariet:2012:INLG2012}
 algoritmo
han sido evaluados en el corpus TUNA-AR~\cite{gatt-balz-kow:2008:ENLG} han logrado una precisi\'on de 33 \% y 40 \% respectivamente.
Como el corpus TUNA-AR s\'olo incluye ER proposicionales, ser�a interesante el trabajo futuro para evaluar c\'omo estos algoritmos realizan en corpora con ER relacionales como GRE3D7.

Un segundo tema vale la pena discutir es c\'omo nuestro algoritmo con sobreespecificaci\'on funciona. Como describimos en la Secci\'on~\ref{sec:overspecification} la generaci\'on de ER sobreespecificadas se realiza en dos pasos. En la primera iteraci\'on, la probabilidad de incluir una propiedad en la RE depende s\'olo de su \puse. No importa si la propiedad en realidad elimina cualquier distractor; Por lo tanto, la RE resultante puede ser sobreespecificada. Despu\'es de todas las propiedades tuvieron la oportunidad de ser inclu\'idas de esta manera, si la ER resultante no identifica al target un\'ivocamente, entonces el algoritmo entra en una segunda fase en la que se asegura que la ER identifica el objetivo un\'ivocamente. Este modelo est\'a inspirado en la obra de~\cite{keysar:Curr98} en el egocentrismo y el lenguaje natural de producci\'on. Keysar~et al \ dijeron que al producir el lenguaje, teniendo en cuenta el punto de vista de los oyentes no se hace desde el principio, pero es m\'as bien una idea de \'ultimo momento~\cite{keysar:Curr98}. Argumentan que los hablantes adultos producen ER egoc\'entricamente, al igual que hacen los ni\~nos, pero luego ajustar las ER para que el destinatario sea capaz de identificar el objetivo de forma inequ�voca. El primer paso es egoc\'entrico, es un proceso heur\'istico basado en un modelo de la prominencia de la escena que contiene el target.

Nuestra definici\'on de \puse\ pretende capturar las prominencias de las propiedades de diferentes escenas y objetivos. El \puse\ de una propiedad cambia de acuerdo a la escena como discutimos en la Secci\'on~\ref{sec:learning}. Esto est\'a en contraste con el trabajo anterior, donde el prominencia de una propiedad es constante en un dominio. Keysar et al~argumentan que la raz\'on para generar y ajustar procedimiento puede tener que ver con las limitaciones de procesamiento de informaci\'on de la mente:. Si la heur\'istica que gu\'ia la fase egoc\'entrica est\'a bien sintonizada, tiene \'exito con la primera ER adecuada en la mayor\'ia de los casos y rara vez requiere ajustes. Curiosamente, se observa un comportamiento similar con nuestro algoritmo: cuando \puse \ valores aprenden de dominio se utilizan, el algoritmo no s\'olo es mucho m\'as preciso, pero tambi\'en mucho m\'as r\'apido.

Como trabajo futuro tenemos la intenci\'on de evaluar nuestro algoritmo de dominios m\'as complejos como los proporcionados por Abiertas dominio Folksonimies~\cite{pacheco-duboue-dominguez:2012:NAACL-HLT}. Tambi\'en planeamos explorar corpus obtienen de la interacci\'on, como el DAR Corpus~\cite{GarGarKolStr10} donde es com\'un observar ER tiro m\'ultiples. Bajo tiempo de los sujetos de presi\'on se producir\'a primero una expresi\'on underspecified que incluye las propiedades m\'as destacadas de la meta (por ejemplo, ``el bot\'on rojo''). Y luego, en un siguiente enunciado, a\~naden propiedades adicionales (por ejemplo, ``a la izquierda de la l\'ampara'') para que la expresi\'on de un RE adecuada identificaci\'on de la meta \'unica.
