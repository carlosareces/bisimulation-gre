\section{Discussion and Conclusions} \label{sec:discussion}

In this article we extend~\shortcite{arec2:2008:Areces} algorithm to generate REs similar to those produced by humans. The modifications 
proposed are based on two observations. First, it has been argued that no fixed ordering of properties is able to generate all the REs produced by humans and, second, humans frequently overspecify their REs~\cite{Engelhardt_Bailey_Ferreira_2006,Arts_Maes_Noordman_Jansen_2011,viet:gene11}. We tested 
the proposed algorithm on the GRE3D7 corpus and found that it is able to generate a large proportion of the overspecified REs found in the corpus without generating trivially redundant referring expressions.

\shortcite{viet:gene11} trains decision trees that are able to achieve a 65\% average accuracy on the GRE3D7 corpus. 
The approach based on decision trees is able to generate overspecified relational descriptions, but they might fail to be referring 
expressions. Indeed, because the  method does not verify the extension of the generated expression over a model of the scene, the 
generated descriptions might not uniquely identify the target.  As we have already discussed,
our algorithm ensures termination and it always finds a referring expression if one exists.  Moreover, it achieves an average of 75\% of accuracy over the scenes used in our tests. 

Different algorithms for the generation of overspecified and distinguishing referring expressions have been proposed in recent years 
(see, e.g.,~\cite{delucena-paraboni:2008:ENLG,ruud-emiel-mariet:2012:INLG2012}.  But, to our knowledge, they have not been evaluated on the 
GRE3D7 corpus and, hence, comparison is difficult. \shortcite{delucena-paraboni:2008:ENLG} and \shortcite{ruud-emiel-mariet:2012:INLG2012} algorithm
have been evaluated on the TUNA-AR corpus~\cite{gatt-balz-kow:2008:ENLG} where they have achieved a 33\% and 40\% accuracy respectively. 
As the TUNA-AR corpus includes only propositional REs, it would be interesting future work to evaluate how these algorithms perform in corpora with relational REs such as GRE3D7. 

A second theme worth discussing is how our algorithm deals with overspecification. As we described in Section~\ref{sec:overspecification} the generation of overspecified REs is performed in two steps. In the first iteration, the probability of including a property in the RE depends only on its \puse. It does not matter whether the property actually eliminate any distractor; hence, the resulting RE may be overspecified. After all properties had a chance of being included this way, if the resulting RE is not distinguishing, then the algorithm enters a second phase in which it makes sure that the RE identifies the target uniquely.  This model is inspired by the work of~\shortcite{keysar:Curr98} on egocentrism and natural language production.  Keysar~et al.\ put forwards the proposal that when producing language, considering the hearers point of view is not done from the outset but it is rather an afterthought. They argue that adult speakers produce REs egocentrically, just like children do, but then adjust the REs so that the addressee is able to identify the target unequivocally. The first, egocentric, step is a heuristic process based in a model of saliency of the scene that contains the target. 

Our definition of \puse\ is intended to capture the saliences of the properties for different scenes and targets. The \puse\ of a property changes according to the scene as we discussed in Section~\ref{subsec:learning}. This is in contrast with previous work where the saliency of a property is constant in a domain. Keysar et al.~argue that the reason for this generate-and-adjust procedure may have to do with information processing limitations of the mind: if the heuristic that guides the egocentric phase is well tunned, it succeeds with a suitable RE in most cases and seldom requires adjustments. Interestingly, we observe a similar behavior with our algorithm: when \puse\ values learned from the domain are used, the algorithm is not only more accurate but also much faster. 

Besides testing our algorithm over the rest of the scenes in the GRE3D7 corpus, as future work we plan to evaluate our algorithm on more complex domains like those provided by Open Domain Folksonomies~\cite{pacheco-duboue-dominguez:2012:NAACL-HLT}. We will also explore corpora obtained through interaction, such as the GIVE Corpus~\cite{GarGarKolStr10} where it is common to observe multi shot REs. Under time pressure subjects will first produce an underspecified expression that includes salient properties of the target (e.g., ``the red button'').  And then, in a following utterance, they add additional properties (e.g., ``to the left of the lamp'') to make the expression a proper RE  identifying the target uniquely.

The source code and the documentation for the algorithm are distributed under the GNU Lesser GPL and can be obtained at \url{http://code.google.com/p/bisimulation-gre}.
