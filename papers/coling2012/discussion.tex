\section{Discussion}
\label{sec:discussion}

In this paper we extend Areces et al.~\shortcite{arec2:2008:Areces} algorithm to generate REs similar to those produced by humans. The extensions done to the algorithm are based on two observations, on the one hand, there is no ordering of properties that is able to generate all the REs produced by humans and humans frequently overspecify their REs~\cite{Engelhardt_Bailey_Ferreira_2006,Arts_Maes_Noordman_Jansen_2011}. We obtain an algorithm that is able to generate a large proportion of the overspecified REs found in corpora without generating bad referring expressions for our domain.

Viethen~\shortcite{viet:gene11} trains decision trees that are able to achieve a 65\% average accuracy on the GRE3D7 corpus, the same corpus that we use for our experiments. The decision trees are able to generate overspecified relational descriptions. Viethen proposal has the problem that the generated descriptions may not be referring expressions, that is, the decision trees do not consider a complete model of the scene that is being described and hence cannot make sure that the generated description are distinguishing.

Algorithms that generate overspecified and distinguishing referring expressions have been proposed~\cite{delucena-paraboni:2008:ENLG,ruud-emiel-mariet:2012:INLG2012}. All these algorithms have been not been evaluated on the GRE3D7, so our results are not directly comparable to them. They have been evaluated on the TUNA-AR corpus~\cite{gatt-balz-kow:2008:ENLG} where~\cite{delucena-paraboni:2008:ENLG} achieves a 33\% accuracy and the~\cite{ruud-emiel-mariet:2012:INLG2012} achives a 40\%. The TUNA-AR corpus includes only propositional REs, it would be interesting to evaluate how these algorithms perform in corpora of relation REs such as GRE3D7. 

Our algorithm is able to generate relational referring expressions which may include redundant information. It achieves a XX\% average accuracy on the GRE3D7 corpora.
Our short term plans of future work include evaluating our algorithm on a more complex domain such as Open Domain Folksonimies~\cite{pacheco-duboue-dominguez:2012:NAACL-HLT}. 

