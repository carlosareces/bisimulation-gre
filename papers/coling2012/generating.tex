\section{Generation of referring expressions}\label{sec:gre}

In linguistics, a \emph{referring expression} (RE) is an expression that 
unequivocally identifies the intended target to the interlocutor, from a set of possible distractors.  
For example, if we intend to identify a certain animal $d$ from a set of pets, the expression 
``the dog'' will be an RE if $d$ is the only dog in the set, and if we are confident
that our interlocutor will identify $d$ as a dog. 

The generation of referring expressions (GRE)  is a key task of most natural 
language generation (NLG) systems (see, e.g.,~\cite[Section 5.4]{dale2000}). 
Depending on the information available to the NLG system, certain objects might 
not be associated with an identifier which can be easily recognized by the user. 
In those cases, the system will have to generate a, possibly complex, description that contains 
enough information so that the interlocutor will be able to identify the intended referent.

The generation of referring expressions is a well developed field in automated natural language generation.
Building upon Dale and Reiter's foundational work~\cite{dale89cooking,Dale1995},
various proposals have investigated the generation of different kinds of referring expressions 
such as relational expressions (``the blue ball next to the cube''~\cite{dale91:_gener_refer_expres_invol_relat}),
reference to sets (``the two small cubes''~\cite{Stone2000}), or more expressive logical connectives (``the 
blue ball not on top of the cube''~\cite{deemter02:_gener_refer_expres}).

REs involving relations, in particular, have
received increasing attention recently; especially in the context of
spatial referring expressions in situated generation (e.g., \cite{kelleher06:_increm_gener_of_spatial_refer}), where it is
particularly natural to use expressions involving spatial relations such as ``the ball on top of the cube.''  However, the classical algorithm by~\shortcite{dale91:_gener_refer_expres_invol_relat} was shown to be unable to generate satisfying REs in practice (see the 
analysis over the \emph{cabinet corpus} in~\cite{viethen06:_algor_for_gener_refer_expres}).  Furthermore, the Dale
and Haddock algorithm and many of its successors (such as~\cite{kelleher06:_increm_gener_of_spatial_refer}) are vulnerable to
the problem of \emph{infinite regress}, where the algorithm enters an infinite loop, jumping back
and forth between descriptions for two related individuals, as in ``the book on the table which supports a book on the
table \ldots''

\shortcite{arec2:2008:Areces,arec:usin11} have proposed new, low complexity algorithms for the generation 
of relational REs (including references to sets) that eliminate the risk of infinite regression. 
These algorithms are based on different variations of the partition refinement algorithms of~\shortcite{paig:thre87}.
The information provided by a given scene is interpreted as a relational model.  The algorithm, then, classifies the 
objects in the model  into sets that can 
be described by the same description.  This classification is successively \emph{refined} (while possible) till the target 
is the only element fitting the description of its class.  The existence of such RE will 
depend, on the one hand, on the information available in the input scene, and on the other 
on the expressive power of the formal language used to describe the elements of the different 
classes in the refinement (e.g., whether the REs can contain negations, or relational information). 

The idea of using a formal language to describe the information that an RE should convey has been discussed 
already in~\cite{Krahmer2003,gardent07:_gener_bridg_defin_descr}.  In~\cite{arec2:2008:Areces,arec:usin11} the 
combination of partition refinement algorithms and different formal languages is used to classify existing 
GRE approaches in an expressiveness hierarchy.  For instance, the classical Dale and Reiter algorithms
compute purely conjunctive formulas; \cite{deemter02:_gener_refer_expres} extends this language by
adding the other propositional connectives, whereas~\cite{dale91:_gener_refer_expres_invol_relat} extends it by
allowing existential quantification.

The refinement algorithms presented in~\cite{arec2:2008:Areces,arec:usin11} effectively
compute REs for all individuals in the domain, at the same time.  The algorithm always terminates, and on termination it returns a formula of 
the formal language chosen that uniquely describes the target, if such a formula exists (i.e., if the 
formal language chosen is expressive enough to uniquely identify the target in the input model). \shortcite{arec2:2008:Areces}
show that the refinement algorithm using the description language \el as formal language is capable of generating 67\% of 
the relational REs in the~\cite{viethen06:_algor_for_gener_refer_expres} dataset, when all possible orders of the relations in the domain are considered. This is in sharp contrast with the analysis 
done in~\cite{viethen06:_algor_for_gener_refer_expres} over the cabinet corpus, of algorithms based in Dale and Reiter's original proposal.    

As mentioned, refinement algorithms require an 
ordered list of the properties that can be used to described the objects in the scene, and the coverage results reported over Viethen and 
Dale's cabinet corpus means that \emph{some ordering} produces a reasonably wide coverage.  In other words, it has been shown that refinement algorithms have the capacity of producing REs similar to those produced by human subjects, provided a suitable ordering over relations appearing 
in the input scene is available, but it is unclear which of all possible orders should be used.  In this article we directly address this issue.  

Our goal is actually twofold. First we show how we can add non-determinism to the refinement algorithms, by replacing the fixed ordering 
over the properties of the input scene by a \emph{probability of use} for each property, and modifying the algorithm accordingly.  
In this way, each call to the algorithm can produce different REs for the same input scene and target.  We will then show that given suitable corpora of REs (like the GRE3D7 corpora discussed in~\cite{viet:gene11}) we can estimate these probability of use so that REs are generated with a probability distribution that matches those found in the corpora.  

The rest of the paper is structured as follows. In Section~\ref{sec:algorithm} we introduce the technical details of the 
refinement algorithms presented in~\cite{arec2:2008:Areces,arec:usin11} and show how to introduce non-determinism using 
the probability of use of the properties in the input scene.  In this section, we assume that these probabilities are provided as 
input to the algorithm. In Section~\ref{sec:learning}, we show how to estimate the 
probability of use of a property from training data. Given corpora consisting of pairs (scene, target) together with the REs used to 
describe the target in each case, we propose a method to compute the probability of use of each property for each scene, and use a machine learning approach to generalize these properties to new targets and scenes not appearing in the corpora. 

Preliminary testing of the resulting algorithms shows that there is still one factor missing to property account for the REs found in corpora: overspecification.  Refinement algorithms only allow a mild form of over-specification in the REs produced.  We discuss this in Section~\ref{sec:overspecification} and propose a modification that let the algorithm generate overspecified, but non trivially redundant RE.  The modification proposed is inspired by the work of~\shortcite{keysar:Curr98}, on the egocentric basis of language.  
Section~\ref{sec:evaluation} presents a quantitative evaluation of the resulting algorithm and discusses interesting examples. We show that when trained with scenes from the GRE3D7 corpora the algorithm can generate REs with a probability distribution that, in certain scenes, coincides with an up to 84.49\% of accuracy with the probability distribution of REs used by humans for that scene. 

In Section~\ref{sec:discussion} we discuss previous work, applications and further motivations for our proposal, focusing on recent work discussing the role of non-determinism and over-specification in the generation of referring expressions. Section~\ref{sec:conclusion} concludes the article and points to future work.

