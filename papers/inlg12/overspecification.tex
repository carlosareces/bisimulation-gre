\section{Non determinism and Overspecification}\label{sec:overspecification}

\fxnote{Lu working here}

Both in psycholinguistics and computational linguistics, researchers have addressed aspects of the problem of generation of referring expressions, but we still have an incomplete understanding of how the human production of referring expressions works. This paper is part of a larger effort which rests on the assumption that for a better, more complete understanding of this process, it is important to bridge the gap between the two disciplines. Such a bridge has both theoretical and methodological advantages. Psycholinguistics has important insights to offer in the human production of referring expressions and a methodology involving carefully constructed and controlled experiments; computational linguistics a well-established approach involving corpus analysis and computational modeling.

Recent studies have suggested that redundant information is frequent in the referring expressions produced by people (e.g., Arts, 2004; P. E. Engelhardt, Bailey, & Ferreira, 2006), that is people generate \emph{overspecified} referring expressions. This is consistent with the Incremental Algorithm~\cite{dale95}, which predicts overspecification in specific situations. In fact, the IA makes some precise predictions concerning overspecification: assuming that colour is preferred over size, then a generated description can include colour as an overspecified property, but not size (after all, if colour were suficient to distinguish a target, the IA would select colour and then terminate, so it would not consider less preferred properties). This is an interesting prediction that has never been
tested experimentally

Perhaps the most striking property of most computational algorithms that is problematic from a psycholinguistic point of view is their \emph{determinism}: they always generate the same referring expression in a particular situation or condition. For example, in a situation where there is no other object of the same category as the target object---say, a single car---most algorithmic models either always generate minimally specified expressions---the car---or always generate overspecified expressions---the red car. But given this specific situation, they would not generate a minimally specified expression in some cases and an overspecified expression in others. This contrasts with the results from experiments with human speakers, which show that they produce various types of referring expressions in a specific condition. For example,~\cite{dale10} showed that even when referring to simple objects in simple scenes, different speakers used a large variety of referring expressions to refer to the same object, while the same speaker was likely to vary their choice of referring expression considerably in very similar (or even isomorphic) scenarios. The results of experimental studies are normally reported averaged across participants, so they do not report whether individual human speakers are deterministic. However, closer examination of the data of individual participants reveals that their responses vary substantially, even within a single experimental condition. For example,~\cite{deemter12} examined the data of~\cite{Fukumura10}, who conducted experiments that investigated the choice between a pronoun and a name for referring to a previously mentioned discourse entity. The clear majority (79\%) of participants in their two main experiments behaved non-deterministically, that is, they produced more than one type of referring expression (i.e., both a pronoun and a name) in at least one of the conditions.

Consider the Incremental Algorithm (IA) once again~\cite{dale95}. The original, deterministic version always generates the cup to refer to a black cup in the
presence of a blue ashtray and yellow candle. The reason is that it assumes a fixed preference order, causing it to check the category of the object (cup) before its colour (red), and since cup rules out both distractors, colour is not tried. Research by~\cite{pechmann89} suggests that speakers do produce overspecified expressions
such as the red cup in this situation. To account for this, the IA could be revised slightly, so that colour is selected first when it is a discriminating feature. But this would still not fully account for Pechmann's results, because he showed that although overspecified expressions (e.g., the black cup) are produced most frequently, minimally specified expressions are produced on one quarter of the trials. To account for this, the ia would need to incorporate some form of non-determinism. One possibility would be to include a random process by which the algorithm checks colour before type three-quarters of the time and type before colour in the remaining quarter (both across speakers and within a single speaker). If we assume that the decision about which property is checked first is a probabilistic, non-deterministic process, then the algorithm makes interesting predictions that are relevant to psycholinguists. For example, a non-deterministic version of the Incremental Algorithm makes exact, quantitative predictions about when overspecification occurs. Although several psycholinguistic studies have shown that overspecification is common, it remains unclear under exactly what conditions it occurs and psycholinguistic models do not make clear predictions concerning this issue. We therefore believe that the algorithm provides an important step towards a better understanding of the possible psychological mechanisms involved in overspecification.

In this paper we address the concrete questions: ``Why and how do speakers overspecify their references?'' and ``Why and how do speakers produce different referring expressions in isomorphic scenarios?''. In other words, in this paper we propose and test a non-deterministic algorithm that is able to generate overspecified referring expressions.  


%%%% p_use p_disc

The probability of use of a property p_i is the probability that p_i is included in a referring expression without considering how discriminating this property is in the referring context. This property is related with ‘salience’; in general, if the salience of a property is high, its use probability is high due to the fact that is a property that ‘comes to mind’ easily and the person may decide to include it in the referring expression before even considering its discriminatory power in context. For instance, bright colors or unusual sizes are properties with high probability of use because they catch the speaker attention very easily. However, as we define it, the probabilty of use is not the same as salience, since probability of use is defined independently of the particular context in which the referring expression is produced, while salience is in general defined with respect to the context. 

p_use (p_i) = (# uses of p_i) / (# of RE)
 
The probability of discrimination of a property p_i is the probability that p_i is understood by the audience (i.e., the interpreter of the referring expression). By understood we mean that the interpreter is able to identify the target as an object that exhibits the property p_i. The property of discrimination is affected by different factors: The first factor is the certainty that the property p_i is true of the target (for instance, gradable properties are not certain because people might have different opinions of whether something is small or middle sized). The second factor is the certainty that the audience will know about the property p_i (for instance, a technical property will have a low probability of discrimination). The third factor is lexical, does the audience how the property p_i, is realized in the target language?

p_disc = \sum ( 1 \ length RE) \ #RE donde aparece p_i

