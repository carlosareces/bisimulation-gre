\documentclass[jair,twoside,11pt,theapa]{article}
\usepackage{jair, theapa, rawfonts}
\usepackage{amsmath}
\usepackage{latexsym}
%\usepackage[draft]{fixme}
\usepackage{xspace}
\usepackage{relsize}
\usepackage{graphicx}
\usepackage[algoruled,vlined]{algorithm2e}
\usepackage{tikz}
\usetikzlibrary{automata}

\newcommand{\tup}[1]{\langle #1 \rangle}
\newcommand{\puse}{\ensuremath{\textsf{p}_\textit{use}}}
\newcommand{\randomuse}{\ensuremath{\textsf{rnd}_\emph{use}}\xspace}
\newcommand{\incuse}{\ensuremath{\textsf{inc}_\emph{use}}\xspace}
\newcommand{\RE}{\textsf{RE}\xspace}
\newcommand{\REL}{\textsf{REL}\xspace}
\newcommand{\IR}{\textrm{I}\!\textrm{R}}
\newcommand{\gM}{\mathcal{M}}
\newcommand{\el}{\ensuremath{\mathcal{EL}}\xspace}
\newcommand{\interp}[1]{|\!|#1|\!|}


%\jairheading{1}{1993}{1-15}{6/91}{9/91}
%\ShortHeadings{Minimizing Conflicts: A Heuristic Repair Method}
%{Minton, Philips, Johnston, \& Laird}
%\firstpageno{25}


\begin{document}

\title{Probabilistic Refinement Algorithms
for the Generation of Referring Expressions}

\author{\name Romina Altamirano \email ialtamir@famaf.unc.edu.ar \\
       \name Carlos Areces \email areces@famaf.unc.edu.ar \\
       \name Luciana Benotti \email benotti@famaf.unc.edu.ar \\ 
       \addr FaMAF Universidad Nacional de C\'ordoba, \\
       Haya de la Torre y Medina Allende
       C\'ordoba, Capital, Argentina
}

% For research notes, remove the comment character in the line below.
% \researchnote

\maketitle


\begin{abstract}
We propose an algorithm for the generation of referring expressions (REs) that adapts the approach of~\shortcite{arec2:2008:Areces,arec:usin11} 
to include overspecification and probabilities learned from corpora.  After introducing the algorithm, we discuss how probabilities required as 
input can be computed for any given domain for which a suitable corpus of REs is available, and how the probabilities can be adjusted for new scenes in the domain using a machine learning approach.  
We exemplify how to compute probabilities over the GRE3D7 corpus of~\shortcite{viet:gene11} \textit{and the TUNA corpus of ~\shortcite{gatt-balz-kow:2008:ENLG}}.
The resulting algorithm is able to generate different referring expressions for the same target with a frequency similar to that observed in corpora. 
%Moreover, the most frequently generated referring expressions not in the corpus are also semantically natural, indicating that the algorithm 
%can generalize from the learning data. \\
We empirically evaluate the new algorithm over the GRE3D7 corpus, and show that the probability distribution of the generated referring expressions 
matches the one found in the corpus with high accuracy. \textit{We also compare our results with the results of~\shortcite{graph08} the winners of the NLG Challenge on Attribute Selection for Generating Referring Expressions (ASGRE). We provide error analysis and conclusion for both corpus. }

\end{abstract}



%1. Generation of Referring Expresions
%2. Adding non-determisnism
%p_use es la probabilidad de usar la propiedad si elimina distractores. 
%3. Learning probabilities from corpus
%4. Adding overspecification
%5. Evaluation
%6. Discussion of Non Determinism and Overspecification
%7. Conclusions

\section{Generation of referring expressions}\label{sec:gre}

In linguistics, a \emph{referring expression} (RE) is an expression that 
unequivocally identifies the intended target to the interlocutor, from a set of possible distractors.
%  
%For example, if we intend to identify a certain animal $d$ from a set of pets, the expression 
%``the dog'' will be an RE if $d$ is the only dog in the set, and if we are confident
%that our interlocutor will identify $d$ as a dog. 
%
The generation of referring expressions (GRE)  is a key task of most natural 
language generation (NLG) systems~\cite[Section 5.4]{dale2000}. 
Depending on the information available to the NLG system, certain objects might 
not be associated with an identifier which can be easily recognized by the user. 
In those cases, the system will have to generate a, possibly complex, description that contains 
enough information so that the interlocutor will be able to identify the intended referent.

The generation of referring expressions is a well developed field in automated natural language generation.
Building upon GRE foundational work~\cite{winograd,dale89cooking,Dale1995},
various proposals have investigated the generation of different kinds of referring expressions 
such as relational expressions (``the blue ball next to the cube''~\cite{dale91:_gener_refer_expres_invol_relat}),
reference to sets (``the two small cubes''~\cite{Stone2000}), or more expressive logical connectives (``the 
blue ball not on top of the cube''~\cite{deemter02:_gener_refer_expres}).
%
REs involving relations, in particular, have
received increasing attention recently.  However, the classical algorithm by~\shortcite{dale91:_gener_refer_expres_invol_relat} was shown to be unable to generate satisfying REs in practice (see the 
analysis over the \emph{cabinet corpus} in~\cite{viethen06:_algor_for_gener_refer_expres}).  Furthermore, the Dale
and Haddock algorithm and many of its successors (such as~\cite{kelleher06:_increm_gener_of_spatial_refer}) are vulnerable to
the problem of \emph{infinite regress}, where the algorithm enters an infinite loop, jumping back
and forth between descriptions for two related individuals, as in ``the book on the table which supports a book on the
table \ldots''

\shortcite{arec2:2008:Areces,arec:usin11} have proposed low complexity algorithms for the generation 
of relational REs 
%(including references to sets) 
that eliminate the risk of infinite regression. 
These algorithms are based on variations of the partition refinement algorithms of~\shortcite{paig:thre87}.
The information provided by a given scene is interpreted as a relational model whose 
objects are classified into sets that fit the same description.  
This classification is successively \emph{refined}  till the target 
is the only element fitting the description of its class.  The existence of an RE 
depends on the information available in the input scene, and on the expressive power of the formal 
language used to describe elements of the different classes in the refinement. 
%
%The idea of using a formal language to describe the information that an RE should convey has been discussed 
%already in~\cite{Krahmer2003,gardent07:_gener_bridg_defin_descr}.  In~\cite{arec2:2008:Areces,arec:usin11} the 
%combination of partition refinement algorithms and different formal languages is used to classify existing 
%GRE approaches in an expressiveness hierarchy.  For instance, the classical Dale and Reiter algorithms
%compute purely conjunctive formulas; \cite{deemter02:_gener_refer_expres} extends this language by
%adding the other propositional connectives, whereas~\cite{dale91:_gener_refer_expres_invol_relat} extends it by
%allowing existential quantification.
%
Refinement algorithms %presented in~\cite{arec2:2008:Areces,arec:usin11} 
effectively compute REs for all individuals in the domain, at the same time. The algorithms always terminate returning a formula of 
the formal language chosen that uniquely describes the target (if the 
formal language is expressive enough to identify the target in the input model). 
%\shortcite{arec2:2008:Areces}
%show that the refinement algorithm using the description language \el  is capable of generating 67\% of 
%the relational REs in the~\cite{viethen06:_algor_for_gener_refer_expres} dataset, when all possible orders of the relations in the domain are considered. This is in sharp contrast with the analysis 
%done in~\cite{viethen06:_algor_for_gener_refer_expres} over the cabinet corpus, of algorithms based in Dale and Reiter's original proposal.    
%
Refinement algorithms require an 
ordered list of properties that can be used to described the objects in the scene, and the naturalness of the generated REs strongly depends on this ordering. 
%coverage results reported over Viethen and 
%Dale's cabinet corpus means that \emph{some ordering} produces a reasonably wide coverage.  In other words, it has been shown that refinement algorithms have the capacity of producing REs similar to those produced by human subjects, provided a suitable ordering over relations appearing 
%in the input scene is available, but it is unclear which of all possible orders should be used.  In this article we directly address this issue.  
The goal of this paper is twofold. First we show how we can add non-determinism and overspecification to the refinement algorithms, by replacing the fixed ordering 
over properties of the input scene by a \emph{probability of use} for each property, and modifying the algorithm accordingly.  
In this way, each call to the algorithm can produce different REs for the same input scene and target.  We will then show that given suitable corpora of REs (like the GRE3D7 corpora discussed in~\cite{viet:gene11}) we can estimate these probabilities of use so that REs are generated with a probability distribution that matches the one found in corpora.  

%The rest of the paper is structured as follows. In Section~\ref{sec:algorithm} we introduce the technical details of the 
%refinement algorithms presented in~\cite{arec2:2008:Areces,arec:usin11} and show how to introduce non-determinism using 
%the probability of use of properties in the input scene.  In this section, we assume that these probabilities are provided as 
%input to the algorithm. In Section~\ref{sec:learning}, we show how to estimate the 
%probability of use of a property from training data. Given corpora consisting of pairs (scene, target) together with the REs used to 
%describe the target in each case, we propose a method to compute the probability of use of each property for each scene, and use a machine learning approach to generalize these properties to new targets and scenes not appearing in the training set. 

%Testing of the resulting algorithms shows that there is still one factor missing to properly account for the REs found in corpora: overspecification.  Refinement algorithms only allow a mild form of overspecification in the REs produced.  We discuss this in Section~\ref{sec:overspecification} and propose a modification that let the algorithm generate overspecified, but non trivially redundant RE.  The modification proposed is inspired by the work of~\shortcite{keysar:Curr98} on the egocentric basis of language.  
%Section~\ref{sec:evaluation} presents a quantitative evaluation of the resulting algorithm. We show that when trained with scenes from the GRE3D7 corpora the algorithm can generate REs with a probability distribution that, in certain scenes, coincides with an up to 84.49\% of accuracy with the probability distribution of REs used by humans for that scene. 

%In Section~\ref{sec:discussion} we discuss related work, motivations and future lines of research, focusing on the role of non-determinism and overspecification in the generation of referring expressions. 


\section{Adding probabilities to a symbolic algorithm} \label{sec:algorithm}

In this section we present a modification of the algorithm
in~\cite{arec2:2008:Areces} where the fixed order of the properties in
the input scene is replaced by a finite probability distribution.  The
required changes are fairly straightforward (see
Figures~\ref{fig:algo1} and~\ref{fig:algo2}), but the behavior of the
resulting algorithm is strikingly changed. To start with, the
algorithm is now non-deterministic: two runs of the algorithm with the
same input might result in different REs for the objects in the scene.
Putting it differently, we can now generate a distribution probability
for the REs generated by the algorithm by repeatedly running it over
the same input.  As we will empirically show in
Section~\ref{sec:evaluation}, given a corpus of REs for a given scene,
it is possible to compute a suitable probability distribution for the
probability of use of the relations in the scene, in such a way that
the probability distribution of the REs generated by the model
simulates the one found in the corpus.

We consider unary properties encoded as binary relations including one
additional `dummy' element in the model (e.g., we encode the fact that
$e_1$ is \emph{blue} saying that it is related to the dummy element by
the \emph{blue} binary relation).

%In order to understand how the algorithm works, and the differences with the original proposal, {\it MOVED we need first to introduce some 
%basic notions.  The input to the algorithm will be a relational model $\mathcal{M} = \tup{\Delta, \interp{\cdot}}$,
%where $\Delta$ is the non-empty domain of objects in the scene, and $\interp{\cdot}$ is an 
%interpretation function that assigns to all properties in the scene its intended extension.  For example, 
%the scene shown in Figure~\ref{GRE3D7-stimulus} could be represented by the model $\gM=\tup{\Delta,\interp{\cdot}}$ shown in Figure~\ref{GRE3D7-stimulus-graph}; where $\Delta = \{e_1,\ldots,e_7\}$, and $\interp{\emph{green}}$, for example, is $\{e_3,e_4,e_6\}$.}


\begin{figure}[ht]
\begin{minipage}[b]{0.45\linewidth}
\centering
\includegraphics[width=\textwidth]{images/3.jpg}
\vspace*{.15cm}
%\caption{Input scene}
\label{GRE3D7-stimulus}
\end{minipage}
\hspace*{-0.35cm}
\begin{minipage}[b]{0.6\linewidth}
\centering
\begin{tikzpicture}
  [
    n/.style={circle,fill,draw,inner sep=3pt,node distance=1.4cm},
    aArrow/.style={->, >=stealth, semithick, shorten <= 2pt, shorten >= 2pt},
  ]
 \node[n,label=above:$e_1$,label=below:{
    \relsize{-1}$\begin{array}{c}
      \nLeft\\[-2pt]
      \nSmall\\[-2pt] 
      \nBlue \\[-2pt] 
      \nBall\end{array}$}] (a) {};

 \node[n,label=above:$e_2$,label=below:{
    \relsize{-1}$\begin{array}{c}
      \nLeft\\[-2pt]
      \nBig\\[-2pt] 
      \nBlue\\[-2pt] 
      \nCube\end{array}$}, right of=a] (b) {};

 \node[n,label=below:$e_3$,label=above:{
    \relsize{-1}$\begin{array}{c}
      \nTop\\[-2pt]
      \nLeft\\[-2pt]
      \nSmall\\[-2pt] 
      \nGreen\\[-2pt] 
      \nBall\end{array}$}, above of=b] (c) {};

 \node[n,label=above:$e_4$,label=below:{
    \relsize{-1}$\begin{array}{c}
      \nSmall\\[-2pt] 
      \nGreen\\[-2pt] 
      \nCube\end{array}$}, right of=b] (d) {};

 \node[n,label=above:$e_5$,label=below:{
    \relsize{-1}$\begin{array}{c}
      \nBig\\[-2pt] 
      \nBlue\\[-2pt] 
      \nBall\end{array}$}, right of=d] (e) {};

 \node[n,label=above:$e_6$,label=below:{
    \relsize{-1}$\begin{array}{c}
      \nBig\\[-2pt] 
      \nGreen\\[-2pt] 
      \nCube\end{array}$}, right of=e] (f) {};

 \node[n,label=below:$e_7$,label=above:{
    \relsize{-1}$\begin{array}{c}
      \nTop\\[-2pt]
      \nSmall\\[-2pt] 
      \nBlue\\[-2pt] 
      \nCube\end{array}$}, above of=f] (g) {};

 \draw [aArrow,bend right=90] (b) to node[auto,swap]{\relsize{-1}$\nBelow$} (c);
 \draw [aArrow,bend right=90] (c) to node[auto,swap]{\relsize{-1}$\nOntop$} (b);

 \draw [aArrow,bend right=30] (d) to node[auto,swap]{\relsize{-1}$\nLeftof$} (e);
 \draw [aArrow,bend right=30] (e) to node[auto,swap]{\relsize{-1}$\nRightof$} (d);

 \draw [aArrow,bend right=90] (f) to node[auto,swap]{\relsize{-1}$\nBelow$} (g);
 \draw [aArrow,bend right=90] (g) to node[auto,swap]{\relsize{-1}$\nOntop$} (f);

 \draw[dotted] (-.5,-1.4) rectangle (7.1,3.1);

 \end{tikzpicture}
\caption{Scene as a relational model}
\label{GRE3D7-stimulus-graph}
\end{minipage}
\end{figure}

%It is clear that a scene can be encoded in different ways as a relational model (for example, we could argue that $e_1$ is also \emph{leftof} $e_2$, and so on).  The algorithm assumes that these issues have been resolved and that the model encodes a suitable representation of the scene we want to describe.  Moreover, we will assume that all relations are \emph{binary}.  We will not consider relations of arity greater than two (relations of higher arity can be encoded as binary relations via reification, if necessary), and unary
%properties can be encoded as binary relations including one additional `dummy' element in the model (e.g., we encode the fact that $e_1$ is \emph{blue} saying that it is related to the dummy element by the \emph{blue} binary relation).

%On termination, the algorithm computes what are called the $\mathcal{L}$-similarity classes of the input model $\gM$. Intuitively, if two elements in the model belong to the same $\mathcal{L}$-similarity class, then $\mathcal{L}$ is not expressive enough to tell them appart (i.e, no formula in $\mathcal{L}$ can distinguish them). 

%In what follows, we will use formulas of the $\el$ description logic language~\cite{baad:desc03} to describe refinement classes.  As discussed in~\cite{arec2:2008:Areces}, this language is suitable for conjunctive relational RE, which are the ones we will find in the corpora used for our evaluation\footnote{Notice, though, that the particular formal language used is independent of the main algorithm, and different add$_{\mathcal{L}}$(R,$\varphi$,\RE) functions can be used depending on the language involved.}. For a detail description of $\el$, we refer to~\cite{baad:desc03}.  For this paper, we only need to know that the interpretation of the formula $\psi \sqcap \exists$R.$\varphi$ is the set of all elements that satisfy $\psi$ and that are related by relation R to some element that satisfy $\varphi$. For example, the interpretation of the formula \emph{ball} $\sqcap \exists$\emph{leftof}.\emph{cube} is the set of all balls that are on the left of some cube.  

We are now ready to describe Algorithms~\ref{algo:bisim-l}
and~\ref{algo:bisim-add-el}. Algorithm~\ref{algo:bisim-l} takes as
input a model $\gM$ and a list Rs of pairs (R,R.\puse) that links each
relation R to some probability of use R.\puse. I.e., if $\REL$ is the
set of all relation symbols in the model (i.e., the \emph{signature}
of the model) then Rs $\in (\REL \times [0,1])^*$. Moreover, we assume
Rs to be ordered by R.\puse.

%\begin{figure}[t]
%\small
%\centering
%\begin{algorithm}[H]
%\dontprintsemicolon
%\caption{Computing $\mathcal{L}$-similarity classes}\label{algo:bisim-l}
%\KwIn{\footnotesize A model $\gM$ and a list Rs $\in (\REL \times [0,1])^*$
% of relation symbols with their \puse\ values, ordered by \puse}
%\KwOut{\footnotesize A set of formulas \RE such that
%$\{\interp{\varphi} \mid \varphi \in \RE\}$ is the set of
%$\mathcal{L}$-similarity classes of $\gM$}

%$\RE \leftarrow \{\top\}$\tcp*[f]{\footnotesize the most general description $\top$ applies to all elements in the scene}

%\For{\em (R,R.\puse) $\in$ Rs}{
%	R.\randomuse = Random(0,1)\tcp*[f]{\footnotesize R.\randomuse is the probability of using R} \;
%        R.\incuse = (1 $-$ R.\puse) / MaxIterations\tcp*[f]{\footnotesize R.\puse\ are incremented by R.\incuse in each loop}
%}

%\Repeat{\em $\forall$((R,R.\puse) $\in$ Rs).(R.\puse $\ge$ 1)\tcp*[f]{\footnotesize R.\puse\ are incremented until they reach 1}}{
%  \While(\tcp*[f]{\footnotesize while some class has at least two elements}){\em $\exists (\varphi \in$ \RE)$.(|\interp{\varphi}|>1)$}{
%      \RE' $\leftarrow$ \RE \tcp*[f]{\footnotesize make a copy for future comparison} \;
%      \For{\em (R, R.\puse) $\in$ Rs}{
%          \If(\tcp*[f]{\footnotesize R will be used in the expression}){\em R.\randomuse $\le$ R.\puse}{
%              \For{\em $\varphi \in$ \RE}{
%                  add$_\mathcal{L}$(R, $\varphi$, \RE)\tcp*[f]{\footnotesize refine all classes using R}}
%                  }\;
%              \If(\tcp*[f]{\footnotesize the classification has changed}){\em \RE $\not =$ \RE'}{exit\tcp*[f]{\footnotesize exit for-loop to try again highest R.\puse}}
%              }
%     \If(\tcp*[f]{\footnotesize the classification has stabilized}){\em \RE $=$ \RE'}{exit\tcp*[f]{\footnotesize exit while-loop to increase R.\puse}}
%  }
%  \For{\em (R,R.\puse) $\in$ Rs}{
%    R.\puse $\leftarrow$ R.\puse $+$ R.\incuse\tcp*[f]{\footnotesize increase R.\puse}
%  }
%}
%\end{algorithm}
%\vspace*{-.5cm}\caption{Main algorithm, dealing with probabilities}\label{fig:algo1}
%\end{figure}


%\begin{figure}[t]
%\small
%\centering
%\begin{algorithm}[H]
%\dontprintsemicolon
%\caption{add$_\el$(R, $\varphi$, \RE)} \label{algo:bisim-add-el}

%\For{\em $\psi \in$ \RE with $|\interp{\psi}| > 1$}{
%  \If(\tcp*[f]{\footnotesize informative: smaller than the original?}){\em $\psi \sqcap \exists$R.$\varphi$ is not subsumed in \RE\ {\bf and} \tcp*[f]{\footnotesize non-redundant: can't be obtained from \RE?}\\
%    \em \ \ \ $\interp{\psi \sqcap \exists \mbox{\em R}.\varphi} \neq \emptyset$ {\bf and} \tcp*[f]{\footnotesize non-trivial: has elements?}\\
%     \ \ \ $\interp{\psi \sqcap \exists \mbox{\em R}.\varphi} \neq \interp{\psi}$ }{
%    add $\psi \sqcap \exists \mbox{R}.\varphi$ to $\RE$ \tcp*[f]{\footnotesize add the new class to the classification} \;
%    remove subsumed formulas from $\RE$ \tcp*[f]{\footnotesize remove redundant classes}
%  }
%}
%\end{algorithm}
%\vspace*{-.5cm}\caption{Refinement function for the \el-language}\label{fig:algo2}
%\end{figure}

The set $\RE$ will contain the formal description of the refinement
classes and it is initialized by the most general description $\top$.
For each R, we first compute R.\randomuse, a random number in [0,1].
If R.\randomuse $\le$ R.\puse\ then we will use R to refine the set of
classes.  The value of R.\puse\ will be incremented by $R.\incuse$ in
each main loop, to ensure that all relations are, at some point,
considered by the algorithm.  This ensures that a referring expression
will be found if it exist; but gives higher probability to expressions
using relations with a high R.\puse.
 
While $\RE$ contains descriptions that can be refined (i.e., classes
with at least two elements) we will call the refinement function
add$_\mathcal{L}$(R,$\varphi$,$\RE$) successively with each relation
in Rs. A change in one of the classes, can trigger changes in
others. For that reason, if $\RE$ changes, we exit the for-loop to
start again with the relations of higher R.\puse. If the after trying
to refine the set with all relations in Rs, the set $\RE$ has not
changed, the we have reach a stable state (i.e., the classes described
in $\RE$ cannot be further refined, using the current R.\puse\
values). We will then increment all the R.\puse\ values and start the
procedure again.

Algorithm~\ref{algo:bisim-add-el} coincides with the one described
in~\cite{arec2:2008:Areces}.  It will refine each of the descriptions
in $\RE$ using the relation R and the other descriptions already in
$\RE$, under certain conditions. The new description should be
\emph{non-redundant} (the new class cannot be obtained as the union of
classes already represented in $\RE$), \emph{non-trivial} (the new
class is not empty), and \emph{informative} (the new class should not
coincide with the original class).  If all this conditions are met,
the new description is added to $\RE$, and redundant descriptions
possible created by the addition of the new description are
eliminated.

Suppose fixed an input model $\gM$ and values for Rs, and fix also
some target element $t$.  Assume also that $t$ indeed has an
$\el$-referring expression.  Upon termination,
Algorithm~\ref{fig:algo1} will compute an $\el$ formula $\varphi$ such
that $\interp{\varphi} = \{t\}$, but $\varphi$ might be different in
each run of the algorithm (even though $\gM$ and Rs are fixed).  If we
repeat this experiment a statistically significant number of times, we
can define an estimate of the probability distribution of the REs
generated by the algorithm for $t$, given $\gM$ and Rs. In
Section~\ref{sec:evaluation} we will show that given a corpus of REs
for $\gM$, it is possible to define R.\puse\ values so that this
probability distribution matches with good accuracy the probability
distribution of REs found in the corpus.

\subsection{Learning to describe new objects from corpora}
\label{sec:learning}

In the previous section we presented an algorithm that assumes that
each relation R used in a referring expressions has a known
probability of use R.\puse. In this section, we describe how to
calculate these probabilities from corpora.  The general set up is the
following: we assume available a corpus of REs associated to different
scenes that are typical of the domain in which the GRE algorithm will
have to operate.  We show first how to calculate R.\puse\ values for
those scenes for which a corpus of REs is available.  We then show how
to generalize these values to other scenes in the domain, using a
machine learning algorithm. \textit{First we} will exemplify the
methodology using the GRE3D7 corpus which we introduce in the Section
~\ref{sec:learningGRE} and then we will show how to do the same with
the TUNA-corpus that we describe in Section~\ref{sec:learningTUNA}.

\subsubsection{Calculating \puse\ when a corpus for the scene is available}

Suppose we want to automatically generate REs for target $t$ in a
given scene, and that we do have available a corpus $C$ of REs for $t$
in that scene (this is exactly the kind of information we find in the
GRE3D7 corpus \textit{and in the TUNA-corpus}).  We use the REs in $C$
to define the relational model used by the algorithm.  Then we
estimate the value of \puse\ for each of the relations in the model as
the percentage of REs in which the relation appears.  I.e.,
\begin{equation}\label{eq1}
R.\puse = \frac{\# \mbox{ of REs in $C$ in which R appears}}{\# \mbox{ of REs in $C$}}.
\end{equation}

In the case of the TUNA corpus the calculation is not neccesary
because we have only one RE for each scene.

This estimation is overly simplified and, for example, it does not
differentiate between the properties of a target and the properties of
a landmark object used in a relational RE to complete the description
of the target.  But it is extremely easy to compute, and we will see
in Section~\ref{sec:evaluation} that it already produces natural REs
that match those found in the corpus.

To clarify the computation of R.\puse\ and the model $\gM$ associated
to each scene we list the required steps in detail, and discuss how we
carried them out in the GRE3D7 corpus:

\begin{enumerate}
\item Tokenize the referring expressions and call the set of tokens
  $T$. In particular, multi-word expressions like ``on top of''
  should be matched to a single token like \emph{ontop}.

\item Remove hyperonyms from $T$. E.g., if both \emph{cube} and
  \emph{thing} appear in $T$, delete \emph{thing}.

\item If the set of tokens obtained in the previous steps contains
  synonyms normalize them to a representative in the synonym class,
  and call the resulting set $\REL$; it will be the signature of the
  model $\gM$ used by the algorithm. E.g., the tokens \emph{little}
  and \emph{small} are both represented by the token \emph{small}.

\item For each scene, define $\gM$ such that the interpretation
  $\interp{\cdot}$ ensures that all the REs in the corpus are REs in
  the model.  E.g., the $\el$ formulas corresponding to the REs in
  Table~\ref{corpus-distribution} should all denote the target in the
  model $\gM$ depicted in
  Figure~\ref{GRE3D7-stimulus-graph}.

\item For each R $\in \REL$ compute R.\puse\ using~(\ref{eq1}) if we
  ave many RE for each scene or we assign 1 to R.\puse\ if R occurs in
  the RE, we assign 0 otherwise.

\end{enumerate}

Steps 1-5 above are easy to carry out (actually, the tokenization and
normalization steps were already done in the GRE3D7 corpus).  Starting
from the scene in Figure~\ref{GRE3D7-stimulus} and the corresponding
corpus shown in Table~\ref{corpus-distribution}, the resulting
signature and their associated \puse\ are listed in the first three
columns of Table~\ref{probability-of-use}.

Notice that the values R.\puse\ obtained in this way should be
interpreted as the probability of using R to describe the target in
model $\gM$, and we could argue that they are correlated to the
\emph{saliency} of R in the model.  For that reason, for example, the
value of \emph{ball}.\puse\ is 1, while the value of
\emph{cube}.\puse\ is 0.178.  These probabilities will not be useful
to describe different targets in different scenes.  We will see how we
can use them to obtain values for new targets and scenes using a
machine learning approach in the next section.  Not surprisingly,
using these values for R.\puse\ the REs generated most often by the
algorithm can be found in the corpus.  More interestingly, as we
discuss in Section~\ref{sec:evaluation} the algorithm generates REs
with a distribution that matches the one found in the corpus and, as
Table~\ref{results-algo-fig3} shows, even the generated REs not found
in the corpus are natural.


\subsubsection{Calculating \puse\ for scenes without corpora for the
  target} \label{subsec:learning}

If there is no corpora that describes the target we can estimate the
\puse~from corpora on a different scenes in the same domain.

We use simple features to obtain the function, all the features can be
extracted automatically from the relational model and are listed in
Table~\ref{features}.

\begin{small}
\begin{table}[h!]
\begin{center}
\begin{tabular}{|l|p{10cm}|}
\hline
target & whether the target element has the property. \\
\#rel-prop & number of properties and relations that the target has.\\
\#rel & number of the relations that the target has. \\
landmark & whether a landmark of the target has the property, an object is a landmark if there is a direct relation in the model 
between them (for the GRE3D7).\\
location-has & whether the RE may use the location of the target in the figure (for TUNA corpus).\\
discrimination & 1 over the number of objects in the model that have the property.  \\
\hline
\end{tabular}
\caption{Features used for learning the \puse ~for each token in the signature of the scenes \textit{(GRE3D7 and TUNA-corpus)} \label{features}}
\end{center}
\end{table}
\end{small}

Our feature set is intentionally simplistic in order for it to be
domain independent. As a result there are some complex relations
between characteristics of the scenes that it is not able to
capture. The most important characteristic of the GRE3D7 domain is
that we are not able to learn, and has an impact in our performance,
the properties of type size (namely, small and large) are used much
more when the target cannot be uniquely identified with taxonomical
(ball and cube) and absolute (green and blue) properties only.  In
other words, in the GRE3D7 corpus the size is used more often (90.2\%)
of the time when the resulting RE is not overspecified than when it is
(34\%). It may not be possible to learn this characteristic from the
GRE3D7 data since even with the domain dependent features defined
in~\cite[Chapter 6]{viet:gene11}, it could not be learned by decision
trees. As a result we can see in Table~\ref{probability-of-use} that
for Fig 13 the value estimated for ``large'' are not close to the
value calculated from corpora.  \textit{In the case of the TUNA-corpus
  we show that we couldn't learn the dependency of dimension-x and
  dimension-y, it mean, when a person adds dimension-x is highly
  probably that he includes dimension-y in his referring expression.}

\begin{table}[h!]
\begin{center}
\begin{tabular}{|l|c|c|c|c|}
\hline
Token & Model Fig 3 \puse & Learned Fig 3\puse & Model Fig 13 \puse & Learned Fig 13 \puse \\
\hline
ball & 1.0 & 1.0 & 1.0 & 1.0 \\
cube & 1.0 & 1.0 & 1.0 & 1.0 \\
green & 0.978 & 0.993 & 1.0 & 0.9875 \\
small & 0.257 & 0.346 & 0.0428 & 0.1993 \\
on-top & 0.178 & 0.179 & 0 & 0\\ 
blue & 0.15 & 0.124 & 0.064 & 0.1353 \\
large & 0.107 & 0.03 & 0.307 & 0.7378 \\
left & 0.007 & 0.002 & 0 & 0.0024 \\
top & 0.007 & 0 & 0 & 0 \\
right & 0 & 0.001 & 0.064 & 0.0005 \\
left-of & 0 & 0 & 0 & 0 \\
right-of & 0 & 0 & 0.064 & 0.1023 \\
below-of & 0 & 0 & 0 & 0 \\
\hline
\end{tabular}
\caption{Probabilities of use of the tokens from the corpora in Table~\ref{GRE3D7-stimulus}  \textit{(GRE3D7)} \label{probability-of-use}}
\end{center}
\end{table}

The learning was done with the machine learning toolkit
WEKA~\cite{Hall:WEK09}, training on all minus one (the one for that we
are learning) for all the scenes of the GRE3D7 \textit{and the
  TUNA-corpus}.  We use linear regression to learn the function of
\puse\ for each word in the signature.  For a given scene, we replace
the variables of the obtained function by the values of the features
in the scene that we want to describe.

Using linear regression we are able to learn interesting
characteristics of the domain. To start with, it learns known facts
such that the saliency of a color depends strongly on whether the
target object is of that color, and it does not depend on its
discrimination power in the model. Moreover, it learns that the on-top
relation is used more frequently than the horizontal relations
(left-of and right-of) which confirms a previous finding reported
in~\cite{viet:gene11}. Finally, it learned a surprising fact of the
GRE3D7 corpus (not found by previous work), that is that size is used
more frequently in an overspecified manner when the target and
landmark share the size. Size was used in overspecified REs in 49\% of
the descriptions for scenes where target and landmark shared the size,
and 25\% of the time when target and landmark did not share the
size. This can be explained by the observation that if landmark and
target share a property, this property is more salient.



\subsection{Generating overspecified descriptions}\label{sec:overspecification}

As it stand, Algorithm~\ref{algo:bisim-l} allows very little overspecification in the REs it
generates.  A relation with a low \puse\ might be enough 
(by itself or in combination with some of the relations already considered) to 
identify the target. Once this relation is added, we obtain an RE, but a shorter, 
more specific RE might be possible, by eliminating some of the previous refinements. 
Hence, the resulting RE might be overspecified. This is the same kind of overspecification 
that the original incremental algorithm allows.  But it has been argued~\cite{Engelhardt_Bailey_Ferreira_2006,Arts_Maes_Noordman_Jansen_2011} that 
a much higher degree of overspecification is usually found in corpora, and this 
is indeed what can be seen in the GRE3D7 corpus.  As we can see in Table~\ref{corpus-distribution}, 
the target is described 16.43\% of the times as a ``small green ball'' when ``green 
ball'' is already an RE.  Using the \puse\ values learnt from the calculus explained
in the previous section, Algorithm~\ref{algo:bisim-l} cannot simulate this behavior. 

Because the fundamental idea of the algorithm is semantics, handling overspecification in 
a natural way is difficult. If two properties have the same interpretation in a given 
model, then once the first has been considered the second will not refine the classes 
obtained so far, and hence the algorithm won't include it in the generated descriptions. 
On the other hand, if we disregard the condition the informativeness constrain (i.e., 
the fact that the addition of a relation indeed refine the class, eliminating some of 
the elements it contains), then we run the risk of generating descriptions like ``the green 
green ball.''

As a compromise, we consider the following variation of Algorithm~\ref{algo:bisim-l} were 
we disregard the informativeness constrain (i.e., we allow the inclusion of new relations 
in the description, even if they do not refine the associated class) \emph{but only during the 
first loop of the algorithm}.  That is, during the first loop over the elements in the 
input list Rs, we will allow the inclusion of all relations that do not trivialize the 
description (i.e., the associated class is not empty).  Because this is done only during 
the first loop, we know that repeated properties will not appear in the generated REs.  
In the remaining loops, additional properties will be added only if they are informative. 
%The modified algorithm is shown in Figure~\ref{fig:algo3}.

%%\newcommand{\nBlue}{\mathit{blue}\xspace}
%%\newcommand{\nGreen}{\mathit{green}\xspace}
%%\newcommand{\nSmall}{\mathit{small}\xspace}
%%\newcommand{\nBig}{\mathit{big}\xspace}
%%\newcommand{\nBall}{\mathit{ball}\xspace}
%%\newcommand{\nCube}{\mathit{cube}\xspace}
%%\newcommand{\nOntop}{\mathit{ontop}\xspace}
%%\newcommand{\nTop}{\mathit{top}\xspace}
%%\newcommand{\nBelow}{\mathit{below}\xspace}
%%\newcommand{\nRightof}{\mathit{rightof}\xspace}
%%\newcommand{\nLeftof}{\mathit{leftof}\xspace}
%%\newcommand{\nLeft}{\mathit{left}\xspace}

%\begin{figure}[ht]
%\begin{minipage}[b]{0.42\linewidth}
%\centering
%\includegraphics[width=\textwidth]{images/3.jpg}
%\vspace*{-.25cm}
%\vspace*{-.4cm}\caption{Input scene}
%\label{GRE3D7-stimulus}
%\end{minipage}
%\hspace*{-0.2cm}
%\begin{minipage}[b]{0.6\linewidth}
%\centering
%\begin{tikzpicture}
%  [
%    n/.style={circle,fill,draw,inner sep=3pt,node distance=1.4cm},
%    aArrow/.style={->, >=stealth, semithick, shorten <= 2pt, shorten >= 2pt},
%  ]
% \node[n,label=above:$e_1$,label=below:{
%    \relsize{-1}$\begin{array}{c}
%      \nLeft\\[-2pt]
%      \nSmall\\[-2pt] 
%      \nBlue \ \  
%      \nBall\end{array}$}] (a) {};

% \node[n,label=above:$e_2$,label=below:{
%    \relsize{-1}$\begin{array}{c}
%      \nLeft\\[-2pt]
%      \nBig\\[-2pt] 
%      \nBlue \ \  
%      \nCube\end{array}$}, right of=a] (b) {};

% \node[n,label=below:$e_3$,label=above:{
%    \relsize{-1}$\begin{array}{c}
%      \nTop \ \       \nLeft \ \ 
%      \nSmall \ \       \nGreen \ \ 
%      \nBall\end{array}$}, above of=b] (c) {};

% \node[n,label=above:$e_4$,label=below:{
%    \relsize{-1}$\begin{array}{c}
%      \nSmall\\[-2pt] 
%      \nGreen\\[-2pt] 
%      \nCube\end{array}$}, right of=b] (d) {};

% \node[n,label=above:$e_5$,label=below:{
%    \relsize{-1}$\begin{array}{c}
%      \nBig\\[-2pt] 
%      \nBlue\\[-2pt] 
%      \nBall\end{array}$}, right of=d] (e) {};

% \node[n,label=above:$e_6$,label=below:{
%    \relsize{-1}$\begin{array}{c}
%      \nBig\\[-2pt] 
%      \nGreen\\[-2pt] 
%      \nCube\end{array}$}, right of=e] (f) {};

% \node[n,label=below:$e_7$,label=above:{
%    \relsize{-1}$\begin{array}{c}
%      \nTop \ \ 
%      \nSmall\ \ 
%      \nBlue \ \  
%      \nCube\end{array}$}, above of=f] (g) {};

% \draw [aArrow,bend right=90] (b) to node[auto,swap]{\relsize{-1}$\nBelow$} (c);
% \draw [aArrow,bend right=90] (c) to node[auto,swap]{\relsize{-1}$\nOntop$} (b);

% \draw [aArrow,bend right=30] (d) to node[auto,swap]{\relsize{-1}$\nLeftof$} (e);
% \draw [aArrow,bend right=30] (e) to node[auto,swap]{\relsize{-1}$\nRightof$} (d);

% \draw [aArrow,bend right=90] (f) to node[auto,swap]{\relsize{-1}$\nBelow$} (g);
% \draw [aArrow,bend right=90] (g) to node[auto,swap]{\relsize{-1}$\nOntop$} (f);

% \draw[dotted] (-.65,-1.2) rectangle (7.1,2.1);

% \end{tikzpicture}
%\vspace*{-.4cm}\caption{Scene as a relational model}
%\label{GRE3D7-stimulus-graph}
%\end{minipage}
%\end{figure}

On termination, the algorithm computes what are called the $\mathcal{L}$-similarity classes of the input model $\gM$. Intuitively, if two elements in the model belong to the same $\mathcal{L}$-similarity class, then $\mathcal{L}$ is not expressive enough to tell them appart (i.e, no formula in $\mathcal{L}$ can distinguish them). 

The algorithm we discuss uses formulas of the $\el$ description logic language~\cite{baad:desc03} to describe refinement classes\footnote{Notice, though, that the particular formal language used is independent of the main algorithm, and different add$_{\mathcal{L}}$(R,$\varphi$,\RE) functions can be used depending on the language involved.}. 
For a detailed description of $\el$, we refer to~\cite{baad:desc03}.  
The interpretation of the $\el$ formula $\psi \sqcap \exists$R.$\varphi$ is the set of all elements that satisfy $\psi$ and that are related by relation R to some element that satisfy $\varphi$. 
For example, the interpretation of the formula \emph{ball} $\sqcap \exists$\emph{leftof}.\emph{cube} is the set of all balls that are on the left of some cube.  

%%We are now ready to describe Algorithms~\ref{algo:bisim-l} and~\ref{algo:bisim-add-el-over}. 

%\begin{figure}[!t]
%\small
%\centering
%\begin{algorithm}[H]
%\dontprintsemicolon
%\caption{Computing $\mathcal{L}$-similarity classes}\label{algo:bisim-l}
%\KwIn{\footnotesize A model $\gM$ and a list Rs $\in (\REL \times [0,1])^*$
% of relation symbols with their \puse\ values, ordered by \puse}
%\KwOut{\footnotesize A set of formulas \RE such that
%$\{\interp{\varphi} \mid \varphi \in \RE\}$ is the set of
%$\mathcal{L}$-similarity classes of $\gM$}

%$\RE \leftarrow \{\top\}$\tcp*[f]{\footnotesize the most general description $\top$ applies to all elements in the scene}

%\For{\em (R,R.\puse) $\in$ Rs}{
%	R.\randomuse = Random(0,1)\tcp*[f]{\footnotesize R.\randomuse is the probability of using R} \;
%        R.\incuse = (1 $-$ R.\puse) / MaxIterations\tcp*[f]{\footnotesize R.\puse\ are incremented by R.\incuse in each loop}
%}

%\Repeat{\em $\forall$((R,R.\puse) $\in$ Rs).(R.\puse $\ge$ 1)\tcp*[f]{\footnotesize R.\puse\ are incremented until they reach 1}}{
%  \While(\tcp*[f]{\footnotesize while some class has at least two elements}){\em $\exists (\varphi \in$ \RE)$.(\#\interp{\varphi}>1)$}{
%      \RE' $\leftarrow$ \RE \tcp*[f]{\footnotesize make a copy for future comparison} \;
%      \For{\em (R, R.\puse) $\in$ Rs}{
%          \If(\tcp*[f]{\footnotesize R will be used in the expression}){\em R.\randomuse $\le$ R.\puse}{
%              \lFor{\em $\varphi \in$ \RE}{
%                  add$_\mathcal{EL}$(R, $\varphi$, \RE)\tcp*[f]{\footnotesize refine all classes using R}}
%                  }\;
%              \If(\tcp*[f]{\footnotesize the classification has changed}){\em \RE $\not =$ \RE'}{exit\tcp*[f]{\footnotesize exit for-loop to try again highest R.\puse}}
%              }
%     \If(\tcp*[f]{\footnotesize the classification has stabilized}){\em \RE $=$ \RE'}{exit\tcp*[f]{\footnotesize exit while-loop to increase R.\puse}}
%  }
%  \lFor{\em (R,R.\puse) $\in$ Rs}{
%    R.\puse $\leftarrow$ R.\puse $+$ R.\incuse\tcp*[f]{\footnotesize increase R.\puse}
%  }
%}
%\end{algorithm}

%\begin{algorithm}[H]
%\dontprintsemicolon
%\caption{add$_\el$(R, $\varphi$, \RE)} \label{algo:bisim-add-el-over}


%\begin{figure}[t]
%\small
%\centering
%\begin{algorithm}[H]
%\dontprintsemicolon
%\caption{add$_\el$(R, $\varphi$, \RE)} \label{algo:bisim-add-el-over}

%\If(\tcp*[f]{\footnotesize are we in the first loop?}){\em FirstLoop?}{
%    Informative $\leftarrow$ TRUE \tcp*[f]{\footnotesize allow overspecification}}
%\lElse(\tcp*[f]{\footnotesize informative: smaller than the original?}) {Informative $\leftarrow$ $\interp{\psi \sqcap \exists \mbox{\em R}.\varphi} \neq \interp{\psi}$} 
%\For{\em $\psi \in$ \RE with $\#\interp{\psi} > 1$}{
%  \If{\em $\psi \sqcap \exists$R.$\varphi$ is not subsumed in \RE\ {\bf and} \tcp*[f]{\footnotesize non-redundant: can't be obtained from \RE?}\\
%    \em \ \ \ $\interp{\psi \sqcap \exists \mbox{\em R}.\varphi} \neq \emptyset$ {\bf and} \tcp*[f]{\footnotesize non-trivial: has elements?}\\
%     \ \ \  \emph{Informative}}{
%    add $\psi \sqcap \exists \mbox{R}.\varphi$ to $\RE$ \tcp*[f]{\footnotesize add the new class to the classification} \;
%    remove subsumed formulas from $\RE$ \tcp*[f]{\footnotesize remove redundant classes}
%  }
%}
%\end{algorithm}
%\vspace*{-.5cm}\caption{Refinement algorithm with probabilities and overspecification for the \el-language}\label{fig:algo3}

%\end{figure}

Algorithm~\ref{algo:bisim-l} takes as input a model and a list Rs of pairs (R,R.\puse) that links each relation R $\in \REL$, the set of all relation symbols in the model,  to some probability of use R.\puse. 
I.e., if $\REL$ is the set of all relation symbols in the model then Rs $\in (\REL \times [0,1])^*$. We assume Rs to be ordered by R.\puse. 

The set $\RE$ will contain the formal description of the refinement classes and it is initialized by the most general description $\top$.  
For each R, we first compute R.\randomuse, a random number in [0,1].  If R.\randomuse $\le$ R.\puse\ then we will use R to refine the set of classes.  The value of R.\puse\ will be incremented by R.$\incuse$ in each main loop, to ensure that all relations are, at some point, considered by the algorithm.  This ensures that a referring expression will be found if it exists; but gives higher probability to expressions using relations with a high R.\puse. 

While $\RE$ contains descriptions that can be refined (i.e., classes with at least two elements) we will call the refinement function add$_\mathcal{L}$(R,$\varphi$,$\RE$) successively with each relation in Rs. A change in one of the classes, can trigger changes in others. For that reason, if $\RE$ changes, we exit the \textbf{for} loop to start again with the relations of higher R.\puse. If after trying to refine the set with all relations in Rs, the set $\RE$ has not changed, then we have reached a stable state (i.e., the classes described in $\RE$ cannot be further refined with the current R.\puse\ values). We will then increment all the R.\puse\ values and start the procedure again. 

Algorithm~\ref{fig:algo1} almost coincides with the one in~\cite{arec2:2008:Areces}.  The \textbf{for} loop will refine each descriptions in $\RE$ using the relation R and the other descriptions already in $\RE$, under certain conditions. The new description should be \emph{non-redundant} (it cannot be obtained from classes already in $\RE$), \emph{non-trivial} (it is not empty), and \emph{informative} (it does not coincide with the original class).  If these conditions are met, the new description is added to $\RE$, and redundant descriptions created by the new description are eliminated. The \textbf{if} statement at the beginning of Algorithm in Figure~\ref{fig:algo2} disregards the informativity test during the first loop of the algorithm allowing overspecification.    

Suppose fixed an input model $\gM$ and values for Rs, and fix also some target element $t$.  Assume also that $t$ indeed has an $\el$-referring expression.  Upon termination, Algorithm~\ref{fig:algo1} will compute an $\el$ formula $\varphi$ such that $\interp{\varphi} = \{t\}$, but $\varphi$ might be different in each run of the algorithm (even though $\gM$ and Rs are fixed).  If we repeat this experiment a statistically significant number of times, we can define an estimate of the probability distribution of the REs generated by the algorithm for $t$, given $\gM$ and Rs. In Section~\ref{sec:evaluation} we will show that given a corpus of REs for $\gM$, it is possible to define R.\puse\ values so that this probability distribution matches with good accuracy the probability distribution of REs found in the corpus.  



%%% Local Variables: 
%%% mode: latex
%%% TeX-master: "journal"
%%% End: 

\section{Learning to describe new objects from corpora}
\label{sec:learning}

In the previous section we presented an algorithm that assumes that each property used in the referring expressions has a known probability of use. In this section, we describe how to calculate these probabilities from corpora when there is one available, and how to estimate them if there is no such corpora. This section starts by describing the corpora that we use for calculating the probabilities of use. Then we present the method used in the calculation. And finally we propose a methodology for estimating the probability of use of properties when describing objects for which no corpora is available. 

\subsection{A corpus of referring expressions}

The corpus we used is known as the GRE3D7 and consists of 4480 referring expressions that describe 16 object in a 3D scene. Each scene contained a small number of simple objects (cubes and balls), and the individual descriptions were elicited in the absence of a preceding discourse. The stimulus scenes were designed in a way that encourage the use of relations between objects, but did not require them. For a detailed description of the collection produre see~\cite[Chapter 5]{viet:gene11}. A sample stimulus used in the corpus collection is shown in Figure~\ref{GRE3D7-stimulus}. 


\subsection{Calculating the probability of use}



\subsection{Learning to describe new objects}

\section{Generating overspecified descriptions}
\label{sec:overspecification}

\section{Evaluation}
\label{sec:evaluation}

In this section we present a quantitative evaluation of the algorithm proposed in this paper. In the GRE area there was a common assumption that there is a gold standard ordering for a given domain~\cite{Dale1995}. However, this assumption has been dropped after empirical studies such as those presented in~\cite{arec2:2008:Areces,viet:gene11}. It has been observed that not only there is no single ordering of properties that covers all human-produced descriptions in a given domain but, in fact, it is not even the case that each speaker consistently uses just one ordering. In this section we show that the non-deterministic algoritm presented in the previous section is able to generate a distribution of REs similar to that observed in corpora, even when no corpus specific for a target object is available. 

Using \puse learned as described in Section~\ref{sec:learning} and running our algorithm 10000 times, we obtain 21 referring expressions for Figure~\ref{GRE3D7-stimulus}. The algorithm generates 8 of the 12 different kinds of REs observed in the 140 ocurrences of the corpora. We also generate other 13 REs for the target, not present in the corpora, but natural sounding as can be observed in Table~\ref{results-algo-fig3} that only represent a 1,64\% of the utterances generated by the algorithm. Hence, 98,36\% of the utterances generated by the algorithm appear in the corpora. In the table we list all the REs found in the corpus for Figure~\ref{GRE3D7-stimulus} and all the RE generated by our algorithm using the learned \puse for the same figure. For each RE, we indicate the number of times it appears in the corpus (\#Cor), the proportion of the corpus it frequency represents (\%Cor), the number of times it is generated by our algorithm (\#Alg) and the proportion of the generated REs its frequency represents (\%Alg). Finally, the accuracy (\%Acc) column compares the REs in the corpus with respect to the REs generated by the algorithm. The accuracy is the proportion of perfect matches between the algorithm output and the human REs from the corpus. The accuracy metric has been used in previous work for comparing the output of a REG algorithm with the REs found in corpora~\cite{sluis07:eval,viet:gene11} and is considered an strict comparison metric for this task. 

\begin{table}[h!]
\begin{center}
\begin{tabular}{|l|c|c|c|c|c|}
\hline
RE & \#Cor & \%Cor & \#Alg & \%Alg & \%Acc \\
\hline
ball,green & 91 & 65 & 5887 &58,52 & 58,52 \\
ball,green,small & 23 & 16,43 & 2270 & 22,58 & 16,43 \\
ball,green,small,on-top(blue,cube,large) & 8 & 5,71 & 0 & 0 & 0 \\
ball,green,on-top(blue,cube) & 5 & 3,57 & 516 & 5,10 & 3,57 \\
ball,green,on-top(blue,cube,large) & 5 & 3,57 & 0 & 0 & 0 \\
ball,green,on-top(blue,cube),small & 2 & 1,43 & 1151 & 11,51 & 1,43 \\
ball,on-top(cube) & 1 & 0,71 & 20 & 0,20 & 0,20 \\
ball,green,small,on-top(blue,cube,large,left) & 1 & 0,71 & 0 & 0 & 0 \\
ball,on-top(cube,large),small	& 1 & 0,71 & 1 & 0,01 & 0,01 \\
ball,green,on-top & 1 & 0,71 &	26 & 0,26 & 0,26 \\
ball,on-top(cube), small & 1 & 0,71 & 18 & 0,18 & 0,18 \\
ball,green,on-top(cube) & 1 & 0,71 & 0 & 0 & 0 \\
ball,green,left	& 0 & 0 & 36 & 0,36 & 0 \\
ball,on-top(blue,cube) & 0 & 0 & 32 & 0,32 & 0 \\
ball,on-top & 0 & 0 & 22 & 0,22 & 0 \\
ball,green,small,on-top & 0 & 0 & 14 & 0,14 & 0 \\
ball,green,left,on-top(blue,cube),small & 0 &  0 & 7 & 0,07 & 0 \\
\hline
Total & 140 & 100 & 10000 & 100 & 80,6 \\
\hline
\end{tabular}
\caption{Referring expressions produced by the algorithm for Figure~\ref{GRE3D7-stimulus}\label{results-algo-fig3}}
\end{center}
\end{table}

In order to put our results in perspective we compare the accuracy obtained for several of the figures in the corpus using the probability of used inferred (column Learned \puse) and the probability of used directly extracted from corpora (column Extracted \puse) as explained in Section~\ref{sec:learning}. We show the accuracy results in Table~\ref{results-algo-all}. The random baseline (column Random) is calculated in by producing random probabilities of use and then running the algorithm 10000 using these random probabilities. Not only the intersection between the randomly generated REs and the corpus is lower but also many of  the REs generated in this way are not naturally sounding (e.g. ``small on the top of a blue cube that is below of something that is small''). 

\begin{table}[h!]
\begin{center}
\begin{tabular}{|l|c|c|c|c|}
\hline
Figure & Random & Learned \puse & Extracted \puse \\
\hline
Fig 1 & 5.38\% & 80.60\% & 84.27\% \\
Fig 3 & 3.29\% & 80.74\% & 83.19\% \\
Fig 6 & 12.76\%	& 84.00\% & 90.71\% \\
Fig 8 & 15.19\%	& 73.62\% & 85.14\% \\
Fig 10 & 7.94\%	& 60.85\% & 81.77\% \\
Fig 12 & 37.94\% & 84.15\% & 75.00\% \\
Fig 13 & 3.48\%	& 53.81\% & 93.57\% \\
Fig 21 & 7.92\%	& 82.00\% & 92.55\% \\
\hline
Average	& 11.74\% & 74.97\% & 85.77\% \\
\hline
\end{tabular}
\caption{Percentage on intersection between the REs generated using random probabilities, learned probabilities and probabilities directly extracted from the figure corpora\label{results-algo-all}}
\end{center}
\end{table}


\section{Discussion}
\label{sec:discussion}

In this paper we extend Areces et al.~\shortcite{arec2:2008:Areces} algorithm to generate REs similar to those produced by humans. The extensions done to the algorithm are based on two observations, on the one hand, there is no ordering of properties that is able to generate all the REs produced by humans and humans frequently overspecify their REs~\cite{Engelhardt_Bailey_Ferreira_2006,Arts_Maes_Noordman_Jansen_2011}. We obtain an algorithm that is able to generate a large proportion of the overspecified REs found in corpora without generating bad referring expressions for our domain.

Viethen~\shortcite{viet:gene11} trains decision trees that are able to achieve a 65\% average accuracy on the GRE3D7 corpus, the same corpus that we use for our experiments. The decision trees are able to generate overspecified relational descriptions. Viethen proposal has the problem that the generated descriptions may not be referring expressions, that is, the decision trees do not consider a complete model of the scene that is being described and hence cannot make sure that the generated description are distinguishing.

Algorithms that generate overspecified and distinguishing referring expressions have been proposed~\cite{delucena-paraboni:2008:ENLG,ruud-emiel-mariet:2012:INLG2012}. All these algorithms have been not been evaluated on the GRE3D7, so our results are not directly comparable to them. They have been evaluated on the TUNA-AR corpus~\cite{gatt-balz-kow:2008:ENLG} where~\cite{delucena-paraboni:2008:ENLG} achieves a 33\% accuracy and the~\cite{ruud-emiel-mariet:2012:INLG2012} achives a 40\%. The TUNA-AR corpus includes only propositional REs, it would be interesting to evaluate how these algorithms perform in corpora of relation REs such as GRE3D7. 

Our algorithm is able to generate relational referring expressions which may include redundant information. It achieves a 75\% average accuracy on the GRE3D7 corpora. Our algorithm not only achieves a good correlation with the distribution of REs found in corpora but the way in which it generates overspecification is 


Our short term plans of future work include evaluating our algorithm on a more complex domain such as Open Domain Folksonimies~\cite{pacheco-duboue-dominguez:2012:NAACL-HLT}. 


\section{Analisis of Errors}
\label{sec:error}

We analize and clasify the errors founded in the TUNA-corpus, the clasification is the following:

Was not RE is when the RE contains attibutes that are not in the model (except for other considered separatelly), or wen the person did a mistake wen give the RE. In this category also put when a person name twice a characteristic like a color.\\ 
Ok means that the system generates the same RE that the person generates with the higher frequency.\\
Contain other means that the RE generated for the human contain other in the attributes-set, and the system couldn't generates because has not other in the model.\\
Sys doesn't generates is when the system in the 100 execusion not generates the RE exactly like a person does.\\
Sys renerates in another position not the first one frequency and taking into account until position 20.\\

In the table we show the diferentes clases of errors with quantities and porcentages for furniture and people separatelly.

\begin{table}[h!]
\begin{center}
\begin{tabular}{|l|c|c|}
\hline
Clasification & count & porcentage \\
\hline
Was not RE	&	16	&	20,00	\%	\\
Ok	&	34	&	42,50	\%	\\
Contain other	&	6	&	7,50	\%	\\
Sys doesn't generate	&	6	&	7,50	\%	\\
Sys generates in another position	&	18	&	22,50	\%	\\
\hline
Total	&	80	&	100	\%	\\
\hline
\end{tabular}
\caption{Percentage average adquired for People}
\end{center}
\end{table}


\begin{table}[h!]
\begin{center}
\begin{tabular}{|l|c|c|}
\hline
Clasification & count & porcentage \\
\hline

Was not RE	&	4	&	5,88	\%	\\
Ok	&	13	&	19,12	\%	\\
Contain other	&	6	&	8,82	\%	\\
Sys doesn't generate	&	17	&	25,00	\%	\\
Sys generates in another position	&	28	&	41,18	\%	\\
\hline
Total	&	68	&	100	\%	\\

\hline
\end{tabular}
\caption{Percentage average adquired for People}
\end{center}
\end{table}






\acks{

 This work was partially supported by grants ANPCyT-PICT-2008-306, ANPCyT-PICT-2010-688, the FP7-PEOPLE-2011-IRSES Project
``Mobility between Europe and Argentina applying Logics to Systems'' (MEALS)
and the Laboratoire Internationale Associ\'e ``INFINIS''.
}

\appendix
\section*{Appendix A. Probability Distributions for N-Queens}


\vskip 0.2in
\bibliography{journal}
\bibliographystyle{theapa}

\end{document}






