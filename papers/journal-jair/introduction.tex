\section{Introduction} \label{introduction}

In linguistics, a \emph{referring expression} (RE) is an expression that 
unequivocally identifies the intended target to the interlocutor, from a set of possible distractors.  
For example, if we intend to identify a certain animal $d$ from a set of pets, the expression 
``the dog'' will be an RE if $d$ is the only dog in the set, and if we are confident
that our interlocutor will identify $d$ as a dog. 

The generation of referring expressions (GRE)  is a key task of most natural 
language generation (NLG) systems~\cite[Section 5.4]{dale2000}. 
Depending on the information available to the NLG system, certain objects might 
not be associated with an identifier which can be easily recognized by the user. 
In those cases, the system will have to generate a, possibly complex, description that contains 
enough information so that the interlocutor will be able to identify the intended referent.

