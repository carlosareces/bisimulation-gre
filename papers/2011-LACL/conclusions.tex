\section{Conclusions}\label{sec:conclusions}

%\fxnote{\tiny  I would like to rewrite the conclusions also. It reads to weak
%and too vague. Empezar repitiendo cual es el punto principal del paper y cual
%fue la contribucion.  Lo que hay que decir ya esta dicho en lo que esta escrito,
%pero falta decirlo more 'to the point' y sin dar tantas vueltas.}

The content determination phase during the generation of referring
expressions identifies which `properties' will be used to refer to a
given target object or set of objects. What is considered as a
`property' is specified in different ways by each of the many
algorithms for content determination existing  in the literature. In
this article, we put forward that this issue can be addressed by
deciding when two elements should be considered to be equal, that
is, by deciding which discriminatory power we want to use. Formally,
the discriminatory power we want to use in a particular case can be
specified syntactically by choosing a particular formal language, or
semantically, by choosing a suitable notion of simulation.  It is
irrelevant whether we choose first the language (and obtain the
associated notion of simulation afterwards) or vice versa.

We maintain that having both at hand is extremely useful. Obviously,
the formal language will come handy as representation language for
the output to the content determination problem.  But perhaps more
importantly, once we have fixed the expressivity we want to use, we
can  rely on model theoretical results defining the adequate notion
of sameness underlying each language, which indicates what can and
cannot be said (as we discussed in \sect{technical}). Moreover, we
can transfer general results from the well-developed fields of
computational logics and graph theory as we discuss
in~\sect{simulation} and~\sect{krahmer}, where we generalized known
algorithms into \emph{families} of GRE algorithms for different
logical languages. %These notions were crucial also in
%\sect{combining} to devise new heuristics.


%Each language has its own expressive power, and
%this induces an adequate notion of sameness. Naturally this last
%notion becomes crucial when trying to uniquely refer to an element
%in a model.


%We have discussed making the notion of expressiveness involved an
%explicit parameter of the GRE problem, unlike usual practice. Hence
%we talk of ``$\+L$-GRE problem'', for a logical language $\+L$. We
%considered various possible choices for $\+L$, though we did not
%argue for any of them. Instead, we tried to make explicit the
%trade-off involved in the selection of a particular $\+L$. This, we
%believe, depends heavily on the given context.


An explicit notion of expressiveness also provides a
cleaner interface, either between the content determination and
surface realization modules or between two collaborating content
determination modules. An instance of the latter was exhibited in
\sect{combining}.

As a future line of research, one may want to avoid sticking to a
fixed $\+L$ but instead favor an incremental approach in which
features of a more expressive language $\+L_1$ are used only when
$\+L_0$ is not enough to distinguish certain element.

%\fixme{Podemos generalizar el algoritmo para conjuntos? Mencionar algoritmo de Piazza}

%To finish this section, observe that the algorithms introduced here
%can also be used to compute referring expressions for \emph{sets} of
%elements. Let $\+L$ be \EL or \ELAN. For any $v\in\Delta$ we define
%the $\+L$-class of $v$ as
%%\fixme{$[v]_{\+L}$ pide demasiado. alcanza con que sean todos los $u$ tq $v \simul{\+L} u$, no?}
%$$
%[v]_{\+L}=\{u\in\Delta\mid u\in\simset_{\+L}(v)\wedge
%v\in\simset_{\+L}(u)\}.
%$$
%A set $T\subseteq\Delta$ has an $\+L$-RE iff $T=[u]_{\+L}$ for some
%$u\in\Delta$. In case $T=[u]_{\+L}$ for some $u$ then for any
%$v\in[u]_{\+L}$, $F(v)$ is a $\+L$-RE for $T$.
%
%Since computing the $\+L$-classes of $\Delta$ is polynomial in
%$\size{\Delta}$, Theorem \ref{thm:complexity-EL-GRE} implies the
%following:


%\fixme{ARREGLAR} To finish this section, observe that the algorithms
%introduced here can also be used to compute referring expressions
%for some \emph{sets} of elements. Let $\+L$ be \EL or \ELAN. A set
%$T\subseteq\Delta$ has an $\+L$-RE iff $T=\simset_{\+L}(u)$ for some
%$u\in\Delta$. In case $T=\simset_{\+L}(u)$ for some $u$ then $F(u)$
%is a $\+L$-RE for $T$. In fact, $F(v)$ also is, for any $v$ such
%that $u\in\simset_{\+L}(v)\wedge v\in\simset_{\+L}(u)$.

%Hence by Theorem \ref{thm:complexity-EL-GRE} we have:

%\begin{cor}
%The problem of generating the \EL and \ELAN referring expressions of
%sets of elements given a finite model
%%$\gM=\tup{\Delta,\interp{\cdot}}$
%can be solved in polynomial time.
%\end{cor}
