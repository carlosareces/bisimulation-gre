\section{On the Size of Referring Expressions} \label{sec:size}

The expressive power of a language $\+L$ determines if there is an
$\+L$-RE for an element $u$. It also influences the \emph{size} of
the \emph{shortest} $\+L$-RE (when they exist). Intuitively, with
more expressive power we are able to `see' more differences and
therefore have more resources at hand to build a shorter formula.

A natural question is, then, whether we can characterize the relative size
of the $\+L$-REs for a given $\+L$. That is, if we can give (tight) upper
bounds for the size of the shortest $\+L$-REs for the elements of an arbitrary
model $\+M$, as a function of the size of $\+M$.

%\fixme{Incluir ejemplo de S/U y FO (Kams Result).}

For the case of one of the most expressive logics considered in this
article, \EPFOL, the answer follows from algorithm
\instFun{makeRE}{\EPFOL} in \sect{krahmer}. Indeed, if an
\EPFOL-RE exists, it is computed by \instFun{buildF}{\EPFOL}
from a model $H$ that is not bigger than the input model. It is easy
to see that this formula is linear in the size of $H$ and, therefore
the size of any \EPFOL-RE is $O(\size{\Delta} +
\size{\interp{\cdot}})$. It is not hard to see that this upper bound
holds for \FOL-REs too.% (cf.~\sect{krahmer} for details).

% Although \instFun{buildFormula}{\EL} also returns a formula that is linear
% in the size of the tree-model $H$, $H$ could be, in principle, exponentially
% larger than the input model. We can use this to give an exponential upper
% bound for the size of the shortest \EL-RE, but is it tight?

One is tempted to conclude from Theorem~\ref{thm:complexity-EL-GRE}
that the size of the shortest \EL-RE is $O(\size{\Delta} \times
\size{\interp{\cdot}})$, but there is a pitfall.
Theorem~\ref{thm:complexity-EL-GRE} assumes that formulas are
represented as a DAG and it guarantees that this DAG is polynomial
in the size of the input model. One can easily reconstruct
 (the syntax tree of) the formula from the DAG, but this, in principle, may lead
 to a exponential blow-up --the result will be an exponentially larger formula,
 but composed of only a polynomial number of different subformulas.
%
As the following example shows, it is indeed possible to obtain an \EL-formula
that is exponentially larger when expanding the DAG representation
generated by Algorithm~\ref{alg:schematic-GRE}.


%Is there an {\em intrinsic} gain in the succinctness of \ALC over
%\EL (or \ELAN over \EL)?
%More formally, is there a sequence of finite pointed models
%$(\+M_i,v_i)_{i\in\NN}$ such that for all $i$, if $n_i$ is the size
%of $\+M_i$, the shortest \EL-RE for $v_i$ has size at least say
%$2^{n_i}$ but $v_i$ has a \ALC-RE of size, say $O(n_i)$? We
%conjecture that the answer is yes, though the models $\+M_i$ seem to
%be far from simple to define.

%Of course, neither Algorithm \ref{alg:schematic-GRE} nor the more
%efficient implementations of it leading to the time complexity
%bounds of Theorem \ref{thm:complexity-EL-GRE} ensures us to obtain
%the shortest (or one of the shortest, in case there is more than
%one) \EL-RE. Forcing the algorithm to obtain the shortest referring
%expression (or at least one which is not is exaggeratedly large),
%seems to be much harder and still needs to be studied. For now, the
%following examples illustrate that the non-deterministic choices of
%$u$, $v$ and $w$ in the guard of the while loop of Algorithm
%\ref{alg:schematic-GRE} are quite sensitive to the size of the
%produced \EL-REs.

%We show that, over the same input $\gM=\tup{\Delta,\interp{\cdot}}$
%where $\size{\Delta}=n$, the $\EL$-REs output by Algorithm
%\ref{alg:schematic-GRE} may be of size greater than $2^n$ or of size
%$O(n^2)$, depending on these non-deterministic choices.

%Before going to the examples, it is important to make the following
%remark. All the computed $\EL$-RE $F(v)$ of Algorithm
%\ref{alg:schematic-GRE} may be stored in a polynomially bounded
%directed acyclic graph. This means that although $F(v)$ may have
%exponential size, the number of distinct subformulas of $F(v)$ is a
%polynomial of the size of the input $\+M$.
%Therefore, the time
%complexity bound of Theorem \ref{thm:complexity-EL-GRE} does not
%suppose a textual output of the referring expressions, but a
%compacted one (represented in a DAG).

%Fist, let us see an execution of Algorithm~\ref{alg:schematic-GRE}
%which outputs a \EL-RE of exponential size.

\begin{example}\label{ex:bad-length}
Consider a language with only one binary relation $r$, and let
$\gM=\tup{\Delta,\interp{\cdot}}$ where $\Delta=\{1,2,\dots, n\}$
and $(i,j)\in\interp{r}$ iff $i<j$.
Algorithm~\ref{alg:schematic-GRE} initializes $F(j)=\top$ for all
$j\in \Delta$. Suppose the following choices in the execution: For
$i=1,\dots, n-1$, iterate $n-i$ times picking $v=w=n-i+1$ and
successively $u=n-i,\dots, 1$. It can be shown that each time a
formula $F(j)$ %($1\leq j<n$)
 is updated, it changes from $\phi$ to
$\phi\wedge\diam\phi$ and hence it doubles its size.
Since $F(1)$ is updated $n-1$ many times, the size of $F(1)$ is
greater than $2^n$.

\iffullversion Suppose that in the first $n-1$ iterations we
successively choose $v=w=n$ and $u=n-1,n-2,\dots,1$. That is, we
discover that $n$ does not simulate $n-1,n-2,\dots,1$. We end up
with $F(1)=\dots=F(n-1)=\top\wedge\diam\top$ and
$S(1)=\dots=S(n-1)=W\setminus\{n\}$. Suppose that in the following
$n-2$ iterations we successively choose $v=w=n-1$ and
$u=n-2,n-3,\dots,1$, namely, we discover that $n-1$ does not
simulate $n-2,\dots,1$. We end up with
$F(i)=(\top\wedge\diam\top)\wedge \diam(\top\wedge\diam\top)$ and
$S(i)=\Delta\setminus\{n,n-1\}$, for $1 \le i \le n-1$. Following
this choice scheme, each time $F(1)$ is updated, it changes from
$\varphi$ to $\varphi\wedge\diam\varphi$. Since $F(1)$ is updated
$n-1$ many times, the size of $F(1)$ is
$O(2^n)=O(2^{\size{\Delta}})$. \fi
\end{example}

The large $\EL$-RE of Example~\ref{ex:bad-length} is due to an
unfortunate (non-deterministic) choice of elements.
Example~\ref{ex:not-that-bad-length} shows that another execution
 leads to a quadratic RE (but notice the
shortest one is linear: $(\exists r)^{(n-1)}.\top$).

\begin{example}\label{ex:not-that-bad-length}
%Let $\gM$ be the model of Example~\ref{ex:bad-length}.
Suppose now that in the first $n-1$ iterations we successively
choose $v=w=n-i$ and $u=v-1$ for $i=0\dots n-2$. It can be seen that
for further convenient choices, $F(1)$ is of size
$O(n^2)$.%
%
%
%We end up with $F(n)=\top$ and $F(i)=\top\wedge\diam F(i+1)$ for
%$i=1,\dots,n-1$. Up to this point, the size of $F(1)$ is linear in
%$n$. Of course, the algorithm does not stop here, but the reader can
%verify that if we successively choose $u=1$ and $v=w=n,n-1,\dots,2$
%we end up with $F(1)$ of size $O(n^2)$.
\end{example}

But is it always possible to obtain an \EL-RE of size polynomial in
the size of the input model, when we represent a formula as a
string, and not as a DAG? In~\cite{FG10} it is shown that the answer
is `no': for $\+L\in\{\ALC, \EL ,\ELAN\}$, the lower bound for the
length of the $\+L$-RE is exponential in the size of the input
model\footnote{
  More precisely, there are infinite models $G_1,G_2,\dots$ such that for
  every $i$, the size of $G_i$ is linear in $i$ but the size of the minimum RE
  for some element in $G_i$ is bounded from below by a function which is exponential
  on $i$.
}, and this lower bound is tight.

%Therefore, Theorem~\ref{thm:complexity-EL-GRE} is true {\em provided
%that $\+L$-formulas are represented as a DAG} (or other similar
%structures) but not as a plain string. This contraposition with
%respect to the representation shows that those relational models
%with target object which are exclusively referenced with
%$\+L$-formulas of exponential size are somewhat redundant.

% \fxnote{\tiny Include here new results?  Cite AiML?}
% We are yet unable to answer whether the exponential bound for the
% size of the minimum \EL-RE is tight. We conjecture no polynomial
% bound can be given, though. In any case, it seems clear that not only
% existence of RE but relative lengths should be taken into account
% when considering the trade-off between expressive powers.

%Note that executions showed in examples
%
%note that in both examples, element $1$ can be described with the
%formula $(\exists r)^{(n-1)}.\top$, which is of size $O(n)$.


\iffullversion
\begin{ex}
Let $\gM$ be the model of example~\ref{ex:bad-length}. Suppose that
in the first $n-1$ iterations we successively choose $v=w=n,u=n-1$;
$v=w=n-1,u=n-2$; $\dots$; $v=w=2,u=1$. That is, we discover that
$n-i+1$ does not simulate $n-i$, for successive $i=1,2,\dots,n-1$.
We end up with $F(n)=\top$ and $F(i)=\top\wedge\diam F(i+1)$ for
$i=1,\dots,n-1$. Up to this point, the size of $F(1)$ is linear in
$n$. But this $F(1)$ is not the final value because we still have to
discover that $3,\dots,n$ do not simulate $1$. The reader may verify
that if we successively choose $u=1$ and $v=w=n,n-1,\dots,2$ we end
up with $F(1)$ of size quadratic in $n$.
\end{ex}
\fi

\iffullversion \fixme{This example is a bit too specific.} Let $\gG$
be any finite graph. If $u,v$ of $\gG$ are bisimilar then for all
formula $\varphi\in\pos$, $u \in \interp{\varphi}$ iff $v \in
\interp{\varphi}$. Observe that in this case, $F(u)$ and $F(v)$
computed by Algorithm~\ref{alg:schematic-GRE} need not necessarily
be equal.

\begin{ex}
PONER EJEMPLO DE ESTO ULTIMO.
\end{ex}
\fi
