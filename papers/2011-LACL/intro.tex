\section{Generating Referring Expressions}

The generation of referring expressions (GRE) --given a context and
an element in that context generate a grammatically correct
expression in a given natural language that uniquely represents the
element-- is a basic task in natural language generation, and one of
active research
(see~\cite{dale89cooking,dale91:_gener_refer_expres_invol_relat,Dale1995,Stone2000,deemter02:_gener_refer_expres}
among others). Most of the work in this area is focused on the
\emph{content determination} problem (i.e., finding a collection of
properties that singles out the target object from the remaining
objects in the context) and leaves the actual \emph{realization}
(i.e., expressing a given content as a grammatically correct
expression) to standard techniques\footnote{For exceptions to this
practice see, e.g.,~\cite{hora:algo97,ston:text98}}.

However, there is yet no general agreement on the basic representation of both the input
and the output to the problem; this is handled in a rather ad-hoc way by each
new proposal instead.

Krahmer~et~al.~\cite{Krahmer2003} make the case for the use of \emph{labeled
directed graphs} in the context of this problem: graphs are abstract
enough to express a large number of domains and there are many
attractive, and well-known algorithms for dealing with this type of
structures. Indeed, these are nothing other than
an alternative representation of relational models, typically used to provide
semantics for formal languages like first and higher-order logics,
modal logics, etc. Even valuations, the basic models of
propositional logic, can be seen as a single-pointed labeled graph.
It is not surprising then that they are well suited to the task.

In this article, we side with~\cite{Krahmer2003} and use labeled
graphs as input, but we argue that an important notion has been left
out when making this decision. Exactly because of their generality,
graphs do not define, by themselves, a unique notion of
\emph{sameness}. When do we say that two nodes in the graphs can or
cannot be referred uniquely in terms of their properties?  This
question only makes sense once we fix a certain level of
expressiveness which determines when two graphs, or two elements in
the same graph, are equivalent.

Expressiveness can be formalized using
structural relations on graphs (isomorphisms, etc.) or, alternatively, logical languages.
Both ways are presented in~\sect{technical}, where we also discuss
how fixing a notion of expressiveness impacts on
the number of instances of the GRE problem that have a solution; the
computational complexity of the GRE algorithms involved; and the
complexity of the surface realization problem.
We then investigate the GRE problem in terms of different notions of
expressiveness. We first explore in~\sect{simulation} how well-known algorithms
from computational logic can be applied to GRE. This is a systematization of the
approach of~\cite{AKS08}, and we are able to answer a complexity
question that was left open there. In \sect{krahmer} we take the opposite route:
we take the well-known GRE-algorithm of~\cite{Krahmer2003}, identify its
underlying expressivity and rewrite in term of other logics. We then show in~\sect{combining}
that both approaches can be combined and finally discuss in \sect{size} the size of
an RE relative to the expressiveness employed. We conclude in~\sect{conclusions} with
a short discussion and prospects for future work.

%We conclude in
%\sect{conclusions} discussing advantages and disadvantages of the
%two particular approaches considered.


%
%In~\cite{AKS08} the problem of generating referring expressions (GRE) is
%reformulated as the problem of computing a formula (in a formal
%language with a formal semantics defined in
%terms of models) that singles out the target to be referred to.  More formally,
%the current available information is represented as a model
%$\gM$ of the chosen formal language $\gL$.  Then an element,
%or a set of elements, in the domain of $\gM$ is chosen as target $T$
%for the generation of a referring expression (RE).  The concrete task then is to
%compute a formula $\varphi \in \gL$ such that the interpretation of
%$\varphi$ in the model $\gM$ coincides exactly with $T$ if such a formula exists,
%or report that such a formula does not exists.
%
%It is clear that the proposed reformulation does not solve the GRE task,
%as the result is not a natural language expression (i.e., a grammatically
%correct expression in some natural language) that refers to the target. But
%it does provide a precise definition of \emph{part} of the problem to be solved
%when generating RE.
%
%Actually, we could split the computational linguistic GRE task into two
%separate problems: 1) the problem of whether a RE \emph{exists}, and
%in that case which \emph{information} should be used to generate it; and 2)
%the problem of generating a grammatically correct expression in a
%particular natural language that exactly conveys this information.
%This last problem can be seen as a surface realization task~\findcite{}, which
%given a semantically annotated grammar will compute a sentence expressing
%the semantics (represented as a formula in the formal language) obtained
%while solving the first problem.
%
%In this article, we will focus in the first problem and use graph algorithms to solve it.
%In that respect, our work is related to~\cite{Krahmer2003} which also provide
%a graph based algorithm computing RE. The main difference is that \cite{Krahmer2003}
%algorithm is based on subgraph isomorphisms, and hence it is computationally expensive.
%We will show that depending on the formal language $\gL$ we used, we will obtained
%algorithms of different complexity.
